\chapter[Habitable Climates: The Influence of Obliquity]{Habitable Climates: The Influence of Obliquity}
\label{ch:obl}
\markright{}



\section{Introduction}
\label{obl_sec:intro}

The Earth's obliquity is remarkably stable: the angle between the
spin--axis and the normal to the orbital plane varies by no more than
a few degrees from its present value of $\sim 23.5\degr$.  This
stability is maintained by torque from the Moon
\citep{laskar_et_al1993,nerondesurgy+laskar1997}.  Even within our own
Solar System, though, the obliquity of other terrestrial planets has
varied significantly more; the analysis of \citet{laskar+robutel1993}
indicates that Mars' obliquity exhibits chaotic variations between
$\sim 0\degr$ and $\sim 60\degr$.

How does climate depend on obliquity?  How does the range of orbital
radii around a star at which a planet could support water-based life
depend on the planet's obliquity?  Has the stability of Earth's
obliquity made it a more climatically hospitable home?  The answers to
these questions will be crucial to evaluate the fraction of stars that
have potentially habitable planets.  There are now more than 280
extrasolar planets known, several of which are less than 10 times the
mass of the Earth.\footnote{See http://exoplanet.eu/.}  {\it Corot},
which has already launched, and {\it Kepler}, scheduled to launch in
less than one year, are dedicated space-based transit-detecting
observatories that will monitor a large number of stars to detect the
small decreases in stellar flux that occur when terrestrial planets
cross in front of their host stars
\citep{baglin2003,borucki_et_al2003,borucki_et_al2007}.  These
missions are expected to multiply by perhaps a thousandfold or more
the number of known terrestrial planets, depending on the distribution
of such planets around solar-type stars (\citealt{borucki_et_al2007,
borucki_et_al2003, basri_et_al2005}; although see revised predictions
in \citealt{beatty+gaudi2008}).  NASA and ESA have plans for ambitious
future missions to obtain spectra of nearby Earth-like planets in the
hope that they would reveal the first unambiguous signatures of life
on a remote world: NASA's {\it Terrestrial Planet Finder} and ESA's
{\it Darwin} \citep{leger+herbst2007}.  The design of such
observatories, and the urgency with which they will be built and
deployed, will depend on the habitability analysis of terrestrial
planets that will be found in the next 5-10 years.

Over the last 50 years, various authors have addressed how to predict
the way in which terrestrial planet habitability depends on
star-planet distance.  Several of the important initial calculations
predated the first discoveries of extrasolar planets, including
\citet{dole1964}, \citet{hart1979}, and the seminal work of
\citet{kasting_et_al1993}.  More recently, \citet[hereafter
WK97]{williams+kasting1997} and \citet{williams+pollard2003} have
tackled precisely the questions posed above, and have concluded that
obliquity variations do not necessarily render a planet non-habitable.
Here we seek to generalize these analyses to model planets that are
less close analogs to Earth than have been considered previously.
Chapter~\ref{ch:hab} \citep[hereafter SMS08]{spiegel_et_al2008}
examines how regionally and temporally habitable climates are affected
by variations in the efficiency with which heat redistribes on a
planet, and by variations in the ocean fraction.  Following in the
footsteps of that analysis, here we again consider the influence on
habitability of several unobservable planetary attributes: we
simultanously vary with obliquity both the efficiency of latitudinal
heat transport, and the distribution of land and ocean.

The remainder of this chapter is structured as follow: In
\S~\ref{obl_sec:model} we describe the energy balance model we use; in
\S~\ref{obl_sec:valid} discuss several validation tests in which our
model performs well enough to give us some confidence in its behavior
for conditions that differ from those found on Earth; in
\S~\ref{obl_sec:results} we examine the influence on regional and
seasonal habitability of various excursions from Earth-like
conditions; and in \S~\ref{obl_sec:disc+conc} we conclude by
considering why the Earth's South Pole is so cold and by examining the
suitability of global radiative balance models for habitable zone
analysis.


\section{Model}
\label{obl_sec:model}

In order to describe the temperature structure and evolution on a
planet, we use a 1-dimensional time-dependent energy balance model
based on a diffusion equation.  This type of model has been used in
previous investigations of habitable climates (WK97, SMS08), in
modeling Martian climate under changes in forcing
\citep{nakamura+tajika2002,nakamura+tajika2003}, and in studies of the
Earth's climate (\citealt{north_et_al1981} and references therein).

Our model is based on the following prognostic equation for the
planetary surface temperature (as described in SMS08):
\begin{equation}
\label{obl_eq:diffu eq1}
C \frac{\partial T[x,t]}{\partial t} - \frac{\partial}{\partial x}
\left( D(1 - x^2) \frac{\partial T[x,t]}{\partial x} \right) + I = S
(1 - A).
\end{equation}
In this equation, $x \equiv \sin \lambda$ is the sine of latitude
$\lambda$, $T$ is the temperature, $C$ is the effective heat capacity
of the surface layer, $D$ is the diffusion coefficient that determines
the efficiency of latitudinal redistribution of heat, $I$ is the
infrared emission function (energy sink), $S$ is the diurnally
averaged insolation function (energy source) and $A$ is the albedo. In
the above equation, $C$, $D$, $I$, and $A$ may be functions of $T$,
$x$, $t$, and possibly other relevant parameters.

Our prescriptions for the functions $C$, $D$, $I$, and $A$ follow
SMS08 and are largely borrowed from WK97 or the existing geophysical
literature on 1D EBMs. For simplicity and flexibility, many of our
models use very simple, physically motivated prescriptions. As
described in SMS08, we find that an infrared cooling function of the
form
\begin{equation}
I[T] = \sigma T^4 / (1 + (3/4)\tau_{\rm IR}[T])
\label{obl_eq:I}
\end{equation}
(i.e., a one-zone model combined with a simple Eddington transfer
approximation; e.g., \citealt{shu1982}), reproduces the greenhouse
effect on Earth reasonably well.  Here, $\tau_{\rm IR}$ represents the
opacity of the atmosphere to long wavelength infrared radiation.
SMS08 describes three pairs of infrared radiation functions and albedo
functions.  In this analysis, we will use two of the three:
$(I_2,A_2)$, which gives the closest match to the Earth's temperature
distribution, and $(I_3,A_3)$, which uses the standard linearized
cooling function of \citet{north+coakley1979}.  The albedo functions
$A_2$ and $A_3$ are constant and low ($\sim 0.3$) for high
temperatures, constant and high ($\sim 0.7$) for low temperatures (to
represent the high albedo of snow and ice), and vary smoothly in
between.  The $(I,A)$ functions are presented in
Table~\ref{obl_tab:one}.  One other $I$ function that we use is the
one proposed in WK97, derived from full radiative-convective
calculations, here denoted $I_{\rm WK97}$.  This function includes the
influence of $\rm CO_2$ on radiative cooling.  For $C$, we assume
various configurations of land and ocean, in each configuration using
the same land, ocean, and ice partial $C$ values as WK97.  Finally, we
adopt $D_{\rm fid} = 5.394\times 10^2 {\rm
~erg~cm^{-2}~s^{-1}~K^{-1}}\times (\Omega_p/\Omega_\Earth)^{-2}$, as
described in SMS08, where $\Omega_p$ is the angular spin frequency of
the model planet and $\Omega_\Earth$ is that of the Earth.

%\begin{deluxetable}{cll}
\begin{table}[p]
\begin{center}
%\tablewidth{0pt}
\caption[Atmospheric Models]{~~Atmospheric Models.}
%\tablehead{
%\colhead{Model}  & \colhead{IR Cooling Function}  & \colhead{Albedo Function} 
%}
%\startdata
\vspace{0.2in}
\begin{tabular}{cll}
  \tableline
  \tableline
  Model             & IR Cooling Function      & Albedo Function\\[0.1in]
  \tableline
2\tablenotemark{a}  &  $I_2[T]  =  \frac{\sigma T^4}{1 + (3/4)\tau_{\rm IR}[T]}$   &  $A_2[T] =  0.525 - 0.245 \tanh[\frac{(T-268{\rm K})}{5{\rm K}}]$   \\
3\tablenotemark{b}  &   $I_3[T]  =  A + B T$   &  $A_3[T] =  0.475 - 0.225 \tanh[\frac{(T-268{\rm K})}{5{\rm K}}]$  \\
3\tablenotemark{c}  &   $I_{\rm WK97}[T,p\rm{CO_2}]$   &  $A_2[T]$  \\
\label{obl_tab:one}
\end{tabular}
\vspace{-0.4cm}
\tablenotetext{a}{\,Model with $T$--dependent optical thickness: $\tau_{\rm IR}[T] = 0.79(T/273{\rm K})^3$.}
\tablenotetext{b}{\,Linearized model: $A = 2.033\times 10^5{\rm~erg~cm^{-2}~s^{-1}}$, $B = 2.094\times 10^3 {\rm~erg~cm^{-2}~s^{-1}~K^{-1}}$.}
\tablenotetext{c}{\,WK97 cooling function; $A_2$ albedo: The functional form is quite complicated and is presented in the appendix of WK97.}
\tablecomments{$\sigma$ is the Stefan-Boltzmann constant.}
\end{center}
\end{table}
%\end{deluxetable}
\afterpage{\clearpage}

Equation~(\ref{obl_eq:diffu eq1}) is solved as described in SMS08, on
a grid uniformly spaced in latitude ($1.25\degr$ resolution, found to
be sufficient from convergence tests).  We again choose ``hot start''
($T\geq 350\rm~K$) initial conditions, to minimize the likelihood that
models will undergo a dynamical transition to a fully ice-covered
(snowball) states from which they cannot recover because of ice-albedo
feedback.

To summarize, we make the following assumptions in the models
presented below:
\begin{enumerate}
 \item {\it Heating/Cooling} -- The heating and cooling functions are
    given by the diurnally averaged insolation from a sun--like
    (1~$M_\sun$, 1~$L_\sun$) star, with albedo and insolation
    functions described above and in Table~\ref{obl_tab:one}.
 \item {\it Latitudinal Heat Transport} -- We test the influence on
    climate of three different efficiencies of latitudinal heat
    transport, within the diffusion equation approximation: an
    Earth-like diffusion coefficient, and diffusion coefficients
    scaled down and up by a factor of 9 (which correspond to 8-hour
    and 72-hour rotation according to the $D\propto {\Omega_p}^{-2}$
    scaling).
 \item {\it Ocean Coverage} -- We vary both the fraction and the
    distribution of ocean coverage.  For ocean fraction, we present a
    series of models with Earth--like 30\%:70\% land:ocean ratio, and
    another series of models that represent a desert-world, with a
    90\%:10\% land:ocean ratio.  For ocean distribution, we present
    models in which there is a uniform distribution (in every latitude
    band) of land and ocean, and others in which the land-mass is a
    single continent centered on the North Pole, while the rest of the
    planet is covered with ocean.
 \item {\it Initial Conditions} -- As described in SMS08, the models
    all have a hot-start initial condition, with the planet's
    temperature uniform and at least 350~K, to minimize the chances of
    ending up in a global snowball state owing solely to the choice of
    initial conditions.  Time begins at the Northern winter solstice.
\end{enumerate}


\section{Model Validation}
\label{obl_sec:valid}

In SMS08, we verified that our ``fiducial'' model at 1~AU -- 70\%
ocean, $I_2,A_2$ cooling/heating -- predicts temperatures that match
the Earth's actual temperature distribution at all latitudes that are
not significantly affected by Antarctica (north of $60\degr$~S or so).
This indicates that the model accounts for the overall planetary
energy balance reasonably well.  Another obvious test is whether the
model correctly predicts the individual energy fluxes that together go
into the overall balance.

The diffusion equation model is essentially a statement of
conservation of energy.  By definition,
\begin{equation}
C \frac{\partial T}{\partial t} \equiv \frac{\partial \sigma}{\partial t},
\label{eq:def hc}
\end{equation}
where $\sigma$ is the energy surface density (internal energy per unit
surface area on the globe).  The diffusion equation, therefore, says
that the rate of change of internal energy at a given point equals the
sources of energy (insolation), minus the sinks (infrared radiation),
minus whatever energy flows away from the point under consideration.

Figure~\ref{obl_fig:set30heat_cool_23p5} presents a comparison between
the annually averaged fluxes of incoming and outgoing radiative energy
in the fiducial model with the corresponding fluxes on Earth, taken
from NASA's Earth Radiation Budget Experiment (ERBE) in the mid-1980s
\citep{barkstrom_et_al1990}.  While our model does not capture the
full shape of the Earth's cooling and heating functions -- in
particular, the annually averaged model heating function is a bit
below the Earth's at the poles -- still, both cooling and heating
fluxes are within 10\% of the Earth's over most of the planet's
surface.

\begin{figure}[p]
\plotone
{figures/chapter6_obl/fig1.eps}
\caption[Annually averaged cooling and heating fluxes for fiducial
model at 1~AU and Earth.]{Annually averaged cooling and heating fluxes
for fiducial model at 1~AU and Earth. The thick red line is infrared
cooling in our fiducial model (70\% ocean, $I_2,A_2$); the thick blue
line is absorbed solar flux in our fiducial model. The thin dashed
magenta line is the Earth's annually averaged long wavelength infrared
radiation, and the thin dashed cyan line is the annually averaged
absorbed solar flux on Earth.}
\label{obl_fig:set30heat_cool_23p5}
\end{figure}

\afterpage{\clearpage}

Figure~\ref{obl_fig:set30heat_cool_monthly23} offers an even more
compelling validation.  In this figure, each of the 12 panels shows
solar (i.e., heating), terrestrial (i.e., cooling), and net (solar
minus terrestrial) radiative fluxes as functions of latitude for one
month.  We were heartened to see that not only are our annually
averaged cooling and heating functions in reasonable gross agreement
with Earth's, as per Figure~\ref{obl_fig:set30heat_cool_23p5}, but
furthermore the temporal variability of radiative fluxes in our model
is similar to that of Earth.


\afterpage{\clearpage}

\begin{figure}[p]
\plotone
{figures/chapter6_obl/fig2.ps}
\caption[Monthly cooling, heating, and net fluxes for fiducial model
at $23.5\degr$ obliquity, at 1~AU, and for Earth.]{Monthly cooling,
heating, and net fluxes for fiducial model at $23.5\degr$ obliquity,
at 1~AU, and for Earth.  Each panel presents the average cooling,
heating, and net (heating minus cooling) radiative fluxes, as
functions of latitude, for one month of the year, starting at the
Northern winter solstice (upper left panel), and incrementing by one
month with each panel to the right.  These fluxes are presented for
both the model (thick solid lines) and the Earth (thin dashed lines).
Model heating is blue; model cooling is red; model net heating is
green.  Earth heating is cyan; Earth cooling is deep magenta; Earth net
heating is black.}
\label{obl_fig:set30heat_cool_monthly23}
\end{figure}

%\clearpage
\afterpage{\clearpage}

Another way to consider the time variability of heating and cooling
fluxes is to look at the global average of each with respect to time.
Figure~\ref{obl_fig:set30_heat_cool_vs_time} presents precisely this
comparison, for both our fiducial model and for actual Earth data.
The bottom panel of this figure shows the net heating flux as a
function of time of year, measured in fraction of a year from the
Northern winter solstice.  Earth's heating flux varies by about 5\%
with respect to the heating flux, while our model's varies somewhat
less.  The heating function for the Earth exceeds the cooling function
during Northern winter for two main reasons: First, the nonzero
eccentricity ($e\approx 0.0167$) of the Earth's orbit places its
perihelion -- which occurs during Northern winter -- approximately
3.4\% closer to the Sun than its aphelion.  This is responsible for
$\sim 7\%$ of the annual $\sim 10\%$ annual variation in net heating
flux.  Another contributing factor is that the Earth's oceans on
average absorb somewhat more insolation than the land, and the
Southern hemisphere -- which faces the Sun during Northern winter --
has greater ocean coverage than the Northern hemisphere.

\begin{figure}[p]
\plotone
{figures/chapter6_obl/fig3.eps}
\caption[Global average cooling, heating, and net radiative flux, as
functions of time in fiducial model at $23.5\degr$ obliquity.]{Global
average cooling, heating, and net radiative flux, as functions of time
in fiducial model at $23.5\degr$ obliquity.  Model fluxes are solid
lines; ERBE data are stars. {\it Top Panel:} Global average cooling
(red curves and magenta stars) and heating (blue curves and cyan
stars) fluxes as a function of time of year, measured in fraction of a
year from the Northern winter solstice.  {\it Bottom Panel:} Net
heating flux (heating minus cooling) for the model (green curve) and
ERBE data (black stars), plotted as percent of the cooling flux.}
\label{obl_fig:set30_heat_cool_vs_time}
\end{figure}

\afterpage{\clearpage}


So far we have considered the radiative fluxes, but what about
diffusive energy flux?  We may combine equation~(\ref{obl_eq:diffu
eq1}) with equation~(\ref{eq:def hc}) to produce:
\begin{equation}
\frac{\partial \sigma}{\partial t} - \frac{\partial}{\partial x} \left\{ D\cos^2\lambda \frac{\partial}{\partial x}T \right\} = \left({\rm sources - sinks}\right)_{\rm energy~per~area},
\label{eq:act. eq. sig}
\end{equation}
where we have substituted $\cos^2 \lambda$ for $(1-x^2)$.  The
$\lambda$ part of the divergence in spherical coordinates is
\begin{equation}
{\rm Div}\left[\vec{F}\right]_\lambda = \left(\frac{1}{R \cos \lambda}\right) \frac{\partial}{\partial \lambda} \left\{ \cos \lambda F_\lambda \right\},
\label{eq:sph div}
\end{equation}
where $F_\lambda$ is the flux of some quantity in the $+\lambda$
direction.  Since $x\equiv \sin \lambda$, $dx = \cos\lambda d\lambda$.
Therefore, we may rewrite equation~(\ref{eq:sph div}) as follows:
\begin{equation}
{\rm Div}\left[\vec{F}\right]_\lambda = \left(\frac{1}{R}\right) \frac{\partial}{\partial x} \left\{ \cos \lambda F_\lambda \right\} = \frac{\partial}{\partial x} \left\{ \frac{\cos \lambda F_\lambda}{R} \right\}
\label{eq:sph div in x}
\end{equation}

By the form of the diffusion equation (eq~\ref{eq:act. eq. sig}), the
term whose divergence is being taken is the rate of energy transport
(per unit length).  Referring back to equation~(\ref{eq:sph div in
x}), the term inside the curly braces equals $(\cos\lambda/R)
F_\lambda$, where $F_\lambda$ is the rate of energy transport per unit
length.  As a result,
\begin{equation}
F_\lambda = R D \cos\lambda \frac{\partial T}{\partial x}.
\label{eq:flux}
\end{equation}
The total diffusive flux $\mathcal{F}_\lambda$ (taking into account
pathlength), therefore, is $2\pi R \cos\lambda F_\lambda$:
\begin{equation}
\mathcal{F}_\lambda = 2\pi R \cos\lambda F_\lambda = 2 \pi R^2 D \cos^2\lambda \frac{\partial T}{\partial x}
\label{eq:total flux}
\end{equation}

Figure~\ref{obl_fig:set80merid_heat_flux} shows the meridional
diffusive heat flux in our fiducial model, at Earth-like $23.5\degr$
obliquity and at extreme $90\degr$ obliquity.  In the Earth-like
configuration, heat flows from the equator toward the poles.  In the
highly oblique configuration, heat flows in the other direction, from
the poles to the equator (in an annually averaged sense).
\citet{williams+pollard2003} present a full general circulation model
(GCM) of an Earth-like planet at Earth-like and higher obliquity.
Figure~2 of that paper shows the meridional heat flux within their
models for $23.5\degr$ obliquity and $85\degr$ obliquity, and the
results are strikingly similar to
ours.\footnote{\citet{trenberth+solomon1994} present the oceanic
poleward heat transport for the Earth, which is roughly 50\% the flux
in the GCM.  This is sensible, because the job of carrying heat to the
poles is shared roughly evenly between the atmosphere and the oceans.}
At $23.5\degr$ obliquity, our model's diffusive flux is very close to
that of the GCM.  At higher obliquity, the flux of our model still
appears to be within 30\% of the GCM's at all latitudes.\footnote{We
say ``appears'' because we are gauging the degree of correspondance
from visual inspection of their figure, since we do not have access to
the GCM data.}  This reasonable concordance indicates that the
treatment of heat transport within our model, despite being very
simple, is still likely to make plausible predictions of heat
transport in less-Earth-like conditions.  We emphasize that it is a
nontrivial point that this entirely different regime of transport
should remain well-captured by a diffusion approximation.

%%%%%%%%%%%%%% Merid Heat Flux
\begin{figure}[p]
\plotone
{figures/chapter6_obl/fig4.eps}
\caption[Annually averaged, longitudinally integrated, meridional heat
transport at $23.5\degr$ and $90\degr$ obliquity.]{Annually averaged,
longitudinally integrated, meridional heat transport at $23.5\degr$
and $90\degr$ obliquity.  The curves indicate the longitudinally
integrated, annually averaged, northward diffusive heat transport.  In
the Earth-like case (blue curve), heat flows from the equator to the
poles; in the highly oblique case (green curve), the annually averaged
heat flows in the opposite direction.}
\label{obl_fig:set80merid_heat_flux}
\end{figure}


\afterpage{\clearpage}



\section{Study of Habitability}
\label{obl_sec:results}

For model planets with $23.5\degr$ obliquity at 1~AU, both pairs of
infrared cooling functions and albedo functions that we present in
this paper are reasonably good matches for the Earth's current climate
as measured by latitudinally averaged temperature, with $(I_2,A2)$
being the somewhat closer fit.  This gives us some confidence that
these functions are reliable guides to how the climate might respond
under different forcing conditions.  In this investigation, we
consider how variations in intrinsic planetary characteristics combine
with the changes in insolation and year-length that would result from
moving changing a model planet's orbital radius from 1~AU, in an
effort to map the zone of regionally habitable climates.

We follow SMS08 and say that, at a given time, a part of a planet is
habitable if its surface temperature is between 273~K and 373~K,
corresponding to the freezing and boiling points of pure water at
1~Atm pressure.  This criterion may be criticized for reasons
discussed in SMS08 and references therein, but it provides a
reasonable starting point for making investigations concrete.  We will
frequently quantify habitability of pseudo-Earths with the temporal
habitability fraction, $f_{\rm time}[a,\lambda]$, defined in SMS08,
where $a$ is orbital semimajor axis, $\lambda$ is latitude, and
$f_{\rm time}$ is the fraction of the year that the point in parameter
space specified by $(a,\lambda)$ spends in the habitable temperature
range.


\subsection{Efficiency of Heat Transport}
\label{obl_ssec:efficiency}

Terrestrial planets with different rotation rates will redistribute
heat from the substellar point (or, in a 1D model, the substellar
latitude) differently.  According to the scaling described above,
wherein the effective diffusion coefficient varies with the inverse
square of the planetary rotation rate, slower spinning planets will
redistribute heat more efficiently, while faster spinning planets will
do so less efficiently.  But from where, and to where, is heat
redistributed?  How does this depend on obliquity and rotation rate?
And what influence does this have on climatic habitability?

\subsubsection{Direction of Heat Flow}
\label{obl_sssec:direction}

For an Earth-like $23.5\degr$ obliquity, the substellar latitude does
not vary very much over the course of the year: the tropics are fairly
close to the equator (the tropical region is less than one third of
the Earth's surface area).  As a result, it is a reasonable
approximation that heat is always being transported from the equator
to the poles.  In contrast, on a planet with significantly larger
obliquity, the direction of heat flow changes over the course of the
annual cycle.  At the equinoxes, the equator is the most strongly
insolated part of the planet (regardless of the obliquity), and so at
the equinoxes heat builds up at the equator, eventually to be
partially redistributed by atmospheric motions (including weather
systems).  But on a highly oblique planet, polar summers are extremely
intense, as measured by diurnally averaged insolation.  As a result,
heat builds up at the poles during their summers, and the flow of heat
reverses direction.

Figure~\ref{obl_fig:set30heat_cool_6090} demonstrates the effect of
such strong polar summers on the global radiation budget of a model
planet.  In Figure~\ref{obl_fig:set30heat_cool_23p5} we see the
annually averaged cooling and heating, as functions of latitude, for
an Earth-like $23.5\degr$ model.  As one would expect, over the annual
cycle, the equator receives significantly more solar radiation than do
the poles, and accordingly the annually averaged heating exceeds the
cooling at the equator.  This indicates that atmospheric motions
transport heat poleward from the equator.  In
Figure~\ref{obl_fig:set30heat_cool_6090}, we present the analogous
functions, in the case of high and extreme obliquity models. The left
panel shows the heating and cooling functions for a model at $60\degr$
obliquity; the right panel shows the same functions for a model at
$90\degr$ obliquity. In these models, the polar summers are so intense
that, averaged over the year, the most strongly insolated parts of the
planet are the North and South Poles!  In an annually averaged sense,
then, heat flows from the poles to the equator, although clearly the
direction of flow changes with the seasons, as described above.  The
import of these plots is that our notion that the poles are the cold
parts might have to be revised in the case of highly oblique worlds.

\begin{figure}[p]
\plottwo
{figures/chapter6_obl/fig5a.eps}
{figures/chapter6_obl/fig5b.eps}
\caption[Annually averaged cooling and heating fluxes for high and
extreme obliquity model planets model and for Earth.]{Annually
averaged cooling and heating fluxes for high and extreme obliquity
model planets model and for Earth.  Analogous to
Fig.~\ref{obl_fig:set30heat_cool_6090}, which shows the annually
averaged heating and cooling as functions of latitude for $23.5\degr$
obliquity, here we present the same functions for Earth-like models at
$60\degr$ obliquity (left panel) and at $90\degr$ obliquity (right
panel). Thick lines are model results; thin dashed lines are ERBE
data.  For both high obliquity cases, unlike on Earth, there is net
annually averaged heating at the poles (i.e., heating exceeds
cooling), and, especially for the extreme obliquity case, net annually
averaged cooling at the equator.}
\label{obl_fig:set30heat_cool_6090}
\end{figure}


\afterpage{\clearpage}

Figures~\ref{obl_fig:set30heat_cool_monthly60} and
\ref{obl_fig:set30heat_cool_monthly90} show in greater detail the
extreme way that the insolation varies over the annual cycle in highly
oblique models.  Notice that the cooling remains much more consistent
than the heating in these models.  This is because of the high
effective heat capacity of the atmosphere above ocean (recall that in
these models, 70\% of the surface area in every latitude band is
ocean).  In models with less ocean coverage, or oceans that are
nonuniformly distributed, the cooling too can vary dramatically over
the annual cycle.


\begin{figure}[p]
\plotone
{figures/chapter6_obl/fig6.ps}
\caption[Monthly cooling, heating, and net fluxes for fiducial model
at $60\degr$ obliquity, at 1~AU, and for Earth.]{Monthly cooling,
heating, and net fluxes for fiducial model at $60\degr$ obliquity, at
1~AU, and for Earth.  Each panel presents the average cooling,
heating, and net (heating minus cooling) radiative fluxes, as
functions of latitude, for one month of the year, starting at the
Northern winter solstice (upper left panel), and incrementing by one
month with each panel to the right.  These fluxes are presented for
both the model (thick solid lines) and the Earth (thin dashed lines).
Model heating is blue; model cooling is red; model net heating is
green.  Earth heating is cyan; Earth cooling is deep magenta; Earth
net heating is black.}
\label{obl_fig:set30heat_cool_monthly60}
\end{figure}

\afterpage{\clearpage}

\begin{figure}[p]
\plotone
{figures/chapter6_obl/fig7.ps}
\caption[Monthly cooling, heating, and net fluxes for fiducial model
at $90\degr$ obliquity, at 1~AU, and for Earth.]{Monthly cooling,
heating, and net fluxes for fiducial model at $90\degr$ obliquity, at
1~AU, and for Earth.  Each panel presents the average cooling,
heating, and net (heating minus cooling) radiative fluxes, as
functions of latitude, for one month of the year, starting at the
Northern winter solstice (upper left panel), and incrementing by one
month with each panel to the right.  These fluxes are presented for
both the model (thick solid lines) and the Earth (thin dashed lines).
Model heating is blue; model cooling is red; model net heating is
green.  Earth heating is cyan; Earth cooling is deep magenta; Earth
net heating is black.}
\label{obl_fig:set30heat_cool_monthly90}
\end{figure}

\afterpage{\clearpage}

\subsubsection{Implications for Habitability}
\label{obl_sssec:hab}

As SMS08 demonstrates, model planets with efficient heat transport
(slowly spinning planets) are more nearly isothermal than models with
Earth-like rotation, which themselves exhibit less latitudinal
variation of temperature than those with inefficient heat transport
(fast spinning planets).  As a result, models corresponding to
slow-spinning worlds tend to be either entirely habitable or entirely
non-habitable at any given time.  In contrast, Earth-like and faster
spinning models may be only partially habitable at a particular time.
They may, for instance, be frozen at the poles and temperate at the
equator, or vice versa in the case of a highly oblique world.

Figure~\ref{obl_fig:obliq_rotI2A2I3A3} demonstrates the kind of
complicated interplay that can go on between obliquity and efficiency
of heat transport, in determining a planet's habitability.  This
figure shows the temporal habitability fraction, as a function of
orbital semimajor axis and latitude, for each of 12 different
combinations of obliquity ($0\degr$, $30\degr$, $60\degr$, $90\degr$)
and diffusion coefficient ($D_{\rm fid}/9$, $D_{\rm fid}$, and
$9D_{\rm fid}$, corresponding respectively to 72-hour, 24-hour, 8-hour
rotation).  The top panel shows these plots for the ($I_2,A_2$) pair,
and the bottom panel depicts the ($I_3,A_3$) pair.  In both panels,
the left column of plots represents efficient transport; the middle
column represents Earth-like transport; and the right column
represents inefficient transport.  Each of the 24 plots in this figure
shows the results of model runs from $0.45$~AU to 1.25~AU in
increments of $0.025$~AU, with the color of each point indicating the
fraction of the year that the latitude at that point spends in the
habitable temperature range (273~K - 373~K) on a model planet at the
specified orbital semimajor axis.  In each plot, the white vertical
dashed lines indicate the radiative equilibrium habitable zone,
calculated (as discussed in SMS08) for a 0-dimensional model planet
with annually averaged, globally averaged insolation.

\begin{figure}[p]
\plotone
{figures/chapter6_obl/fig8a.eps}\\
\plotone
{figures/chapter6_obl/fig8b.eps}
\caption[Model temporal habitability fraction under different
obliquities, rotation rates, and cooling/heating functions.]{Model
temporal habitability fraction under different obliquities, rotation
rates, and cooling/heating functions. In both panels, obliquity varies
from $0\degr$ (top row) to $90\degr$ (bottom row) and diffusion
coefficient varies from $(1/9)D_0$ (left column) to $9D_0$ (right
column).  The $x$-axis of each cell is orbital radius, in AU, and the
$y$-axis is latitude.  The color of each point indicates the fraction
of the year that that part of parameter space spends in the habitable
temperature range (273~K - 373~K).  {\it Top Panel:} $I_3,A_3$
combination. {\it Bottom Panel:} $I_2,A_2$ combination.}
\label{obl_fig:obliq_rotI2A2I3A3}
\end{figure}
\afterpage{\clearpage}

There are a number of intriguing features of this figure.  The most
obvious is that, as expected, at every obliquity, less efficient
transport results in more strongly latitudinally differentiated
temporal habitability.  In addition, at each transport-efficiency
value, the {\tt $>$}~sign shape of the seasonally habitable ribbon at
low obliquity reverses to something more like a {\tt $<$}~sign at high
obliquity.  In other words, at low obliquity, the relatively cold
poles are habitable closer to the star and the relatively warm equator
is habitable farther from the star, and at high obliquity, this
reverses and the poles are relatively warm, while the equator is
relatively colder.

Furthermore, notice that, in both panels, the plots in most cells show
a very abrupt outer boundary to the seasonally habitable zone.  This
is because, as discussed in SMS08, the ice-albedo feedback renders
these models quite sensitive to changes in forcing.  Small reductions
in insolation can become amplified, because the ice-coverage
increases, which increases the global albedo and leads to further
reduction in insolation.  This feedback mechanism renders these models
susceptible to succombing to a global snowball state, from which they
cannot recover within our model framework.  The main exceptions to
this trend are the low obliquity, fast-spinning models in the upper
right corners of both panels (although even these models drop to 0\%
habitability at orbital radii that are small relative to the outer
boundary of the radiative equilibrium habitable zone, indicated by the
white dashed lines).  Interestingly, the fairly small change in
cooling/heating functions from $I_2,A_2$ to $I_3,A_3$ allows the cell
in the lower right corner of the bottom panel -- extreme obliquity,
inefficient transport -- to avoid snapping to a snowball in a single
step in orbital radius.  In that model, the intense summer insolation
at the poles, combined with the relative thermal isolation of
different latitudes, allows the poles to heat up above the freezing
point of water during their summers, even at orbital distances where
other models would be entirely frozen.  In sum, the susceptibility to
snowball conditions is sensitive to details of parameterizations.


\subsection{Land/Ocean Distribution}
\label{obl_ssec:ocean distribution}

As described in SMS08, the large covering fraction of oceans on the
Earth (roughly 70\%) stabilizes our climate over an annual cycle, by
virtue of the large effective heat capacity of atmosphere over ocean:
over land, the thermal relaxation timescale is several months; over
the 50~m mixing layer of the ocean, the thermal relaxation timescale
is more than a decade.  As a result, in a 1D model (such as ours) that
does not resolve continents in longitude, any latitude band with
significant ocean fraction will have strongly suppressed annual
temperature fluctuations relative to a latitude band with low ocean
fraction.  Because we do not know of any way to predict the
distribution of continents and oceans on an extrasolar planet, it is
important to consider the influence on climatic habitability of other
possible distributions.

\subsubsection{Nonuniform Ocean Coverage}
\label{obl_sssec:nonuniform}

We consider several planet models with distributions of land and ocean
that are not uniform across different latitudes: one with 30\% land
coverage, with the land-mass centered on the North Pole (extending
down to $\sim 24\degr$ North latitude), and the other (discussed in
\S~\ref{obl_sssec:desert}) with 90\% land, again centered on the North
Pole.\footnote{It is more natural to conceive of this model as having
an ocean centered at the South Pole.}  Because of the relatively low
thermal inertia of atmosphere over land, parts of a model planet that
are dominated by land can freeze or boil during the course of the year
and still return to temperate conditions at other times.  In fact, at
some orbital distances, and at high obliquity, the polar regions of
some models freeze {\it and} boil within an annual cycle.

Figure~\ref{obl_fig:set31_23p5_60_90} displays the tremendous swings
of temperature that can occur over latitude bands that lack ocean, and
also indicates that annually averaged calculations may miss a lot of
the story of what conditions are like on planets.  This figure shows
the annually averaged temperature, and the temperature evolution, on a
model planet with a North Polar continent that is 30\% of the total
surface area, at 1~AU.  This model uses $I_2,A_2$ cooling/heating, and
shows results with obliquity $23.5\degr$, $60\degr$, and $90\degr$.
At all three obliquities, the left column -- the annually averaged
temperature profile -- provides an impoverished view of the actual
climatic conditions.  Looking at just the left panels: the $23.5\degr$
obliquity model appears slightly asymmetrical in temperature
distribution, with the continental North Pole 8~K warmer than the
oceanic South Pole; the $60\degr$ obliquity model appears cooler at
the continent pole; and the $90\degr$ obliquity model again appears
warmer at the continent pole, but appears frozen over the whole globe.
In truth, all three models reach significantly higher temperatures at
the continent pole during its summer than at the other pole.  At both
$60\degr$ and $90\degr$ obliquity, North Pole summer temperatures
exceed 410~K, as the Sun shines nearly straight down on the pole for
months.  An obvious puzzle is implicit in this figure: Although we may
not have much intuition for what the polar summers should be like on
high obliquity planets, it is surprising to see a prediction of summer
polar continent temperatures in excess of 310~K in the $23.5\degr$
obliquity model, given that Antarctica -- Earth's continental pole --
is significantly {\it colder} than the non-continental pole, and for
the most part neither pole ever reaches temperatures above freezing.
Accounting for Antarctica in our model framework is not easy, and we
address this issue in greater detail in \S~\ref{obl_ssec:cold_SP?},
but our sense is that the resolution of this issue may have to do with
model initial conditions, or, in the case of the Earth, with the
longterm history of temperature evolution.

\begin{figure}[p]
\plottwo
{figures/chapter6_obl/fig9a.eps}
{figures/chapter6_obl/fig9b.eps}\\
\plottwo
{figures/chapter6_obl/fig9c.eps}
{figures/chapter6_obl/fig9d.eps}\\
\plottwo
{figures/chapter6_obl/fig9e.eps}
{figures/chapter6_obl/fig9f.eps}\\
\caption[Annually averaged and space-time plot of temperatures on
models with a North Polar continent that takes up 30\% of surface area
at 1~AU, cooling/heating = $I_2,A_2$, obliquity = $23.5\degr$,
$60\degr$, $90\degr$.]{Annually averaged and space-time plot of
temperatures on models with a North Polar continent that takes up 30\%
of surface area (the other 70\% is ocean) at 1~AU, cooling/heating =
$I_2,A_2$, obliquity = $23.5\degr$, $60\degr$, $90\degr$. {\it Left:}
The magenta curve shows the annually averaged temperature profile for
models with North Polar continents that extend down to $\sim 24\degr$
North latitude.  The blue and dashed green curves are for reference --
blue: the fiducial model (identical to this one, except the 70\% ocean
is uniformly distributed in every latitude band); dashed green: the
Earth's actual temperature profile, measured by NCEP/NCAR in 2004
\citep{kistler_et_al1999, kalnay_et_al1996}.  {\it Right:}
Latitude/time plot of temperature evolution, from model-years 8
through 50, for the polar continent models. {\it Top Row:
$23.5\degr$.}  {\it Middle Row: $60\degr$.}  {\it Bottom Row: $90\degr$.}}
\label{obl_fig:set31_23p5_60_90}
\end{figure}

\afterpage{\clearpage}

Figure~\ref{obl_fig:set26_obliq} presents plots of the temporal
habitability fraction for the world just considered, at obliquities
$0\degr$, $30\degr$, $60\degr$, $90\degr$.  We see in
Fig.~\ref{obl_fig:set31_23p5_60_90} that the presence of land at the
North Pole causes tremendous swings in temperature there; we see in
Fig.~\ref{obl_fig:set26_obliq} that at nonzero obliquity this is the
case at other values of semimajor axis, too.  These large seasonal
variations lead to exotic shapes in plots of temporal habitability.
Compared with uniformly ocean-dominated worlds, much more of the
parameter space at each obliquity except $0\degr$ is habitable neither
0\% nor 100\% of the year, but somewhere in between.

\begin{figure}[p]
\plotone
{figures/chapter6_obl/fig10.eps}
\caption[Model temporal habitability fraction under different
obliquities, North Polar continent covering 30\% of surface.]{Model
temporal habitability fraction under different obliquities, North
Polar continent covering 30\% of surface.  The diffusion coefficient
is $D=D_0$.  Similar to Figure~\ref{obl_fig:set25_obliq}.}
\label{obl_fig:set26_obliq}
\end{figure}

\afterpage{\clearpage}

\subsubsection{Desert Worlds}
\label{obl_sssec:desert}

We now consider two model planets with just 10\% ocean fraction.  We
examine the cases of both uniformly distributed ocean (10\% in every
latitude band) and ocean concentrated at the South Pole (extending
northward to $\sim 53\degr$ South latitude).
Figure~\ref{obl_fig:set25_obliq} presents the temporal habitability
for the uniform desert world, and Figure~\ref{obl_fig:set28_obliq}
presents the analogous plot for the South Polar ocean world.  We again
see regions of some model planets that swing from freezing to boiling
temperatures over the course of the year.  This is responsible for the
butterfly shape of the temporal habitability plots in the $60\degr$
and $90\degr$ cells of Fig.~\ref{obl_fig:set25_obliq}: at $a\sim
0.9\rm~AU$, the poles are habitable for less of the year than both the
more equatorial regions at that orbital distance and than the poles at
closer and more distant orbits.

\begin{figure}[p]
\plotone
{figures/chapter6_obl/fig11.eps}
\caption[Model temporal habitability fraction under different
obliquities, 10\% ocean uniformly distributed.]{Model temporal
habitability fraction under different obliquities, 10\% ocean
uniformly distributed.  The diffusion coefficient is $D=D_0$.  Similar
to Figure~\ref{obl_fig:obliq_rotI2A2I3A3}.  The color of each point
indicates the fraction of the year that that part of parameter space
spends in the habitable temperature range (273~K - 373~K).}
\label{obl_fig:set25_obliq}
\end{figure}

\afterpage{\clearpage}

\begin{figure}[p]
\plotone
{figures/chapter6_obl/fig12.eps}
\caption[Model temporal habitability fraction under different
obliquities, North Polar continent covering 90\% of surface (South
Polar ocean).]{Model temporal habitability fraction under different
obliquities, North Polar continent covering 90\% of surface (South
Polar ocean).  The diffusion coefficient is $D=D_0$.  Similar to
Figure~\ref{obl_fig:set25_obliq}.}
\label{obl_fig:set28_obliq}
\end{figure}

\afterpage{\clearpage}

These models, and those presented in \S~\ref{obl_sssec:nonuniform},
suggest that at extreme obliquity the inner edge of the zone of
regionally and seasonally habitable climates is extended dramatically
inward, while the outer boundary is extended only mildly outward.  A
caveat that should accompany this observation is that assuming a
cooling function that is constant with orbital radius probably leads
to a flawed treatment of both the high and low insolation limits of
these models.  At the inner edge of the habitable zone, large
increases in atmospheric water content can cause a reduction in the
cooling function that causes a runaway greenhouse effect; the eventual
catastrophic water loss can result in Venus-like outcomes, as
described by \citet{kasting_et_al1993} and references therein
(although this type of outcome might be mitigated by a reduction of
the heating function due to increased cloud-albedo, as mentioned in
SMS08).  At the outer edge, reduced efficiency of the
carbonate-silicate weathering cycle is likely to lead to a significant
increase the partial pressure of CO$_2$ \citep{kasting_et_al1993},
which could extend the habitable zone from $\sim 1\rm~AU$ in our
models to $\sim 1.4\rm~AU$ or more in some cases.  We discuss in
greater detail the effect of a cooling function that varies with
orbital distance in \S~\ref{obl_ssec:IWK97}.

Still, for most plausible cooling functions, the low thermal inertia
of atmosphere over land might lead to severe polar climates in highly
oblique models.  What are we to make of partial ``habitability'' by
our criterion in the case of a region of a planet that actually boils
and freezes every year?  There are some microbes on Earth that can
reproduce at freezing temperatures, and others that can reproduce at
boiling temperatures, although none of which we are aware that can do
both.  If a part of a planet regularly swings through these wild
extremes of climate, is it fair to call it habitable?  This is an open
question, but it is worthwhile to keep in mind that microbes on Earth
appear to be as hardy as they need to be: nearly everywhere that
biologists have searched, they have found some microbes thriving.
Perhaps most significant from the perspective of habitability, the
reduced thermal inertia of these models appears to render them less
susceptible to global snowball events.


\subsection{Modeling the Far Reaches of the Habitable Zone}
\label{obl_ssec:IWK97}

\citet{walker_et_al1981} propose that a planet's temperature is
regulated on long timescales by a feedback mechanism involving
weathering of silicate rocks through carbonic acid from CO$_2$
disolved in water.  They argue that since the rate of weathering (and
hence of removal of CO$_2$ from the atmosphere) increases with
temperature, this process is an important negative feedback on climate
that acts to keep temperatures near the freezing point of water.
\citet{kasting_et_al1993} point out that this negative feedback can
significantly offset the extreme sensitivity of climate to changes in
orbital distance away from 1~AU seen in models such as those of
\citet{hart1979} and the models presented in SMS08 and thus far in
this paper.  These models are so sensitive because they contain a
significant positive feedback of the Earth-climate -- the ice-albedo
feedback whereby at lower temperatures the absorbed insolation is
dramatically reduced because of the high albedo of ice -- but they
ignore the counterbalancing negative feedback of the
carbonate-silicate weathering cycle.

In order to probe the combined influence on climate of rotation rate
and obliquity in the context of the expected CO$_2$-rich atmosphere
that a pseudo Earth would have at 1.4~AU, we switch to the cooling
function $I_{\rm WK97}$ used in WK97, with CO$_2$ partial pressure
($p$CO$_2$) set to 1~bar and 2~bars; we maintain the simple albedo
function $A_2$.  We find that at both levels of CO$_2$, model planets
maintain globally temperate conditions at all obliquities for both
$D=9D_0$ and $D=D_0$ -- corresponding to slow and Earth-like rotation.
Perhaps counterintuitively, at reduced transport efficiency,
corresponding to fast rotation, these models are more susceptible to
global glaciation.  One might expect that the thermal isolation of
different latitudes in these models, relative to slower spinning
models, might allow parts of the models that receive less insolation
to cool off without having much effect on other latitudes.  Indeed, in
the limit as $D\rightarrow 0$ this must happen.  But with a small but
nonvanishing diffusion coefficient, it turns out that the thermal
near-isolation allows cold regions to become very cold, but this
information is still communicated to other regions, dragging down
their temperatures.  Sometimes, this process can drag down global
temperatures enough to plunge the models into global snowball state.

Figure~\ref{obl_fig:set45_23p5_60_90} shows the global average
temperature and the climate evolution for fast-spinning model planets
at 1.4~AU with 1~bar atmospheric CO$_2$, with $I_{\rm WK97},A_2$
cooling/heating.  At $23.5\degr$ obliquity, the cold temperatures at
the poles drag the model into a snowball state; at $90\degr$
obliquity, the cold equator drags the model to the same fate; and at
$60\degr$ obliquity, no part of the planet receives consistently low
enough insolation to end up in this trap.  Note that the snowball
effect seen in the $23.5\degr$ and $90\degr$ obliquity models might be
particularly calamitous because of the possibility of atmospheric
collapse on a much shorter timescale than the vulcanism can replenish
CO$_2$.  At 1~bar, the freezing point of CO$_2$ is $\sim 195 \rm~K$.
Both the $23.5\degr$ and $90\degr$ obliquity models reach temperatures
below this threshold over large enough regions of their surfaces that
significant amounts of the atmospheric CO$_2$ might condense out as
dry ice, thereby reducing the greenhouse effect.  The risk of
atmospheric collapse would be somewhat lessened because of the latent
heat of condensation, which would tend to prevent too much CO$_2$ from
freezing out during any winter.  Realistically treating this
possibility would require incorporating a latent heat term in the
energy balance equation, as \citet{nakamura+tajika2002} do in their
Mars EBM.




%%%%%%%%%%%%%% Set45
\begin{figure}[p]
\plottwo
{figures/chapter6_obl/fig13a.eps}
{figures/chapter6_obl/fig13b.eps}\\
\plottwo
{figures/chapter6_obl/fig13c.eps}
{figures/chapter6_obl/fig13d.eps}\\
\plottwo
{figures/chapter6_obl/fig13e.eps}
{figures/chapter6_obl/fig13f.eps}
\caption[Annually averaged and space-time plot of temperatures on fast
spinning world at 1.4~AU, WK97 cooling function, $p\rm CO_2 = 1~bars$,
obliquity = $23.5\degr$, $60\degr$, $90\degr$.]{Annually averaged and
space-time plot of temperatures on fast spinning world at 1.4~AU, WK97
cooling function, $p\rm CO_2 = 1~bars$, obliquity = $23.5\degr$,
$60\degr$, $90\degr$.
%
Same as Figure~\ref{obl_fig:set31_23p5_60_90} except cooling function
is $I_{\rm WK97}$ with $p\rm CO_2 = 1~bars$, and $D=(1/9)D_0$.}
\label{obl_fig:set45_23p5_60_90}
\end{figure}

\afterpage{\clearpage}

Figure~\ref{obl_fig:set42_23p5_60_90} shows similar plots, but for a
thicker atmosphere, with $p$CO$_2=2\rm~bars$.  In this case, all three
obliquities avoid a snowball fate.  Still, poles in the $23.5\degr$
model are below the freezing point of CO$_2$ at 2~bars ($\sim 203
\rm~K$) for more than a third of the year, indicating that it is
conceivable that this model might also risk losing its CO$_2$
greenhouse on a shorter timescale than it can be replenished.

%%%%%%%%%%%%%% Set42
\begin{figure}[p]
\plottwo
{figures/chapter6_obl/fig14a.eps}
{figures/chapter6_obl/fig14b.eps}\\
\plottwo
{figures/chapter6_obl/fig14c.eps}
{figures/chapter6_obl/fig14d.eps}\\
\plottwo
{figures/chapter6_obl/fig14e.eps}
{figures/chapter6_obl/fig14f.eps}
\caption[Annually averaged and space-time plot of temperatures on fast
spinning world at 1.4~AU, WK97 cooling function, $p\rm CO_2 = 2~bars$,
obliquity = $23.5\degr$, $60\degr$, $90\degr$.]{Annually averaged and
    space-time plot of temperatures on fast spinning world at 1.4~AU,
    WK97 cooling function, $p\rm CO_2 = 2~bars$, obliquity =
    $23.5\degr$, $60\degr$, $90\degr$.  Same as
    Figure~\ref{obl_fig:set42_23p5_60_90} except $p\rm CO_2 =
    2~bars$.}
\label{obl_fig:set42_23p5_60_90}
\end{figure}

\afterpage{\clearpage}


\section{Discussion and Conclusions}
\label{obl_sec:disc+conc}

\subsection{Whence Antarctica's Extreme Cold?}
\label{obl_ssec:cold_SP?}

As discussed in \S~\ref{obl_ssec:IWK97}, many of the models presented
here suffer from several limitations, including overly simplified
treatments of both the inner and outer edges of the habitable zone.
An additional puzzle, suggested in \S~\ref{obl_sssec:nonuniform}, is
to figure out why our Earth-like obliquity models with polar
continents have such different climate from the Earth's Solar Polar
continent of Antarctica.  There are various reasons why Antarctica is
so cold.  One obvious and important factor is that it is covered with
over a mile of ice.  This guarantees that on short timescales anyway,
regardless of the temperature the albedo is always high.  In
actuality, the enforced high albedo of course helps to prevent the
temperature from exceeding the freezing point of water.  In an attempt
to simulate the Earth's sharply colder temperatures at the South Pole,
we therefore set up several models in which we impose ice at the South
Pole, regardless of temperature.  This prescription affects the
albedo, forcing it to be constanly high, and the heat capacity,
increasing it above the heat capacity for atmosphere above pure land
by a factor of 2.

Figure~\ref{obl_fig:set32_34_23p5} shows the results of two attempts
to duplicate the temperature at Antarctica within our models.  In both
models, there is a South Polar ice continent that extendes to
$70\degr$ South latitude ($\sim 3\%$ the total surface area), and the
rest of the planet is $\approx 72.2\%$ covered with ocean, which
combine to make an overall ocean fraction of 70\%.  The top panels of
this figure use the standard $A_2$ albedo function everywhere.
Clearly, the high albedo (constantly 0.77) at the ice-continent has
only a very mild effect on that continent's temperature.  The bottom
panels use a modified albedo function: $A_2[T]$ everywhere except the
simulated Antarctica, and ``shiny ice''($A=0.88$) on the South Polar
continent.  This prescription forces the temperature at the ice
continent somewhat colder than the other continent, but
through the diffusion term the cooler South Polar temperature is
communicated to the rest of the planet, too, eventually supressing the
North Polar temperature by more than 10~K.  Using even shinier ice at
the South Pole can, when using the $I_2$ function, drive the planet to
a snowball state.

%\clearpage

%%%%%%%%%%%%%% Set32/34
\begin{figure}[p]
\plottwo
{figures/chapter6_obl/fig15a.eps}
{figures/chapter6_obl/fig15b.eps}\\
\plottwo
{figures/chapter6_obl/fig15c.eps}
{figures/chapter6_obl/fig15d.eps}
\caption[Annually averaged and space-time plot of temperatures on
models with Antarctica analogs, at 1~AU, cooling/heating = $I_2,A_2$,
obliquity = $23.5\degr$.]{Annually averaged and space-time plot of
temperatures on models with Antarctica analogs, at 1~AU,
cooling/heating = $I_2,A_2$, obliquity = $23.5\degr$.  These models
have 70\% ocean coverage overall, with a South Polar ice continent
extending up to $70\degr$ South latitude, and $\approx 72.2\%$ ocean
fraction over the rest of the planet. {\it Top Row:} Normal ice
($A=0.77$) at South Pole.  {\it Bottom Row:} ``Shiny ice'' ($A=0.88$)
at South Pole.
%
Similar to Figure~\ref{obl_fig:set31_23p5_60_90}.}
\label{obl_fig:set32_34_23p5}
\end{figure}

\afterpage{\clearpage}

\subsection{Applicability of Global Radiative Balance Calculations}
\label{obl_sssec:global_balance}

Historically, calculations of habitable zones have often assumed
global radiative balance conditions.  Although these calculations by
definition cannot account for the regional character of habitability,
one might hope that they would still provide a decent proxy for the
global average of actual conditions.  Indeed, as seen in
Fig.~\ref{obl_fig:set30_heat_cool_vs_time}, the Earth itself is within
$\sim 5\%$ of radiative equilibrium throughout the year.  Accordingly,
global radiative balance models have provided a very useful starting
point for considerations of how habitability depends on orbital
radius.

Figure~\ref{obl_fig:set31_heat_cool_vs_time} presents the globally
averaged cooling, heating, and net radiative fluxes for a model with a
North Polar continent that covers 30\% of the planet's surface, at
$60\degr$ and $90\degr$ obliquity.  The $60\degr$ obliquity model gets
up to nearly 40\% out of global radiative balance at some times during
the year, and the $90\degr$ obliquity model reaches nearly 60\% out of
global radiative balance!  It has always been recognized that planets
on highly eccentric orbits experience forcings that are significantly
different from annually and globally averaged conditions, but our here
results show that, even on circular orbits, planets can experience
conditions that are far from radiative equilibrium.  This further
underscores the importance of regional, time-dependent models for
addressing the habitability of extrasolar terrestrial planets.

%\clearpage

\begin{figure}[p]
\plotone
{figures/chapter6_obl/fig16a.eps} \\
\plotone
{figures/chapter6_obl/fig16b.eps}
\caption[Global average cooling, heating, and net radiative flux, as
functions of time in model with North Polar continent at $60\degr$ and
$90\degr$ obliquity.]{Global average cooling, heating, and net
radiative flux, as functions of time in model with North Polar
continent at $60\degr$ and $90\degr$ obliquity.
%
Similar to Figure~\ref{obl_fig:set30_heat_cool_vs_time}.  {\it Top two
panels:} $60\degr$ obliquity; {\it Bottom two panels:} $90\degr$
obliquity.}
\label{obl_fig:set31_heat_cool_vs_time}
\end{figure}

\clearpage

\subsection{Summary}
\label{obl_sssec:summary}

We have presented a series of 1D EBMs to address the variety of
possible climatic conditions that might result on extrasolar
terrestrial planets.  We considered forcings determined by a variety
of unobservable planetary attributes, including obliquity, rotation
rate, the distribution of land/ocean coverage, and the detailed nature
of the radiative cooling and heating functions.  Previous
investigations have found that high obliquity does not necessarily
render planets more susceptible to global glatiation.  Our results
provide preliminary indications of an even more promising result:
models at high obliquity often appear to be less prone to snowball
states, which is a result that appears to hold for a wide variety of
other properties.  Counterintuitively, faster rotating Earth-like
planets may fall victim to snowball glaciation events at closer
orbital radii than slower rotating planets; this result appears to
hold for various cooling functions, including both Earth-like infrared
cooling and the expected radiative cooling for the kind of massive
CO$_2$ atmosphere that \citet{kasting_et_al1993} predict to obtain in
the outer parts of habitable zones.  We find that planets with small
ocean fractions or polar continents can experience very severe
seasonal climatic variations, but that these planets also might
maintain seasonally and regionally habitable conditions over a larger
range of orbital radii than more Earth-like planets.  We furthermore
find that high obliquity models with nonuniform distributions of land
and ocean experience forcing conditions that are very far from global
radiative balance relative to what the Earth experiences.  This last
finding provides a substantial justification of the present
time-dependent, regional model.


