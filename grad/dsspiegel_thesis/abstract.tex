\thispagestyle{empty}
\begin{center}

{\Large \bf ABSTRACT}

\vskip.5in
{\Large \bf On Constraining Nontrivial Properties of Exoplanets\\ and Other Topics in Astrophysics}\\

\vskip.5in
{\large David Solomon Spiegel} \\
%\vskip0.3in
%{\large Sponsors:  Frits B.S. Paerels} \\
\vskip.5in
\end{center}

\noindent {\bf Part I}\\
%
At the present epoch, a significant fraction of the baryonic matter in
the universe is in a hot, X--ray emitting intragroup medium of small
galaxy groups; but at higher redshift the nature of the intergalactic
medium of such groups is unconstrained.  I examine the X--ray
luminosity of 21 galaxy groups in the \textsl{CNOC2} Field Galaxy
Redshift Survey, at redshifts $0.1 < z < 0.6$, and find that not one
was visible in over 100~ksec of observation with XMM-{\it Newton},
therefore providing tentative evidence that groups at intermediate
redshift are underluminous relative to their local cousins.
%.  Of the 21 groups in
%the field of view of the \textsl{EPIC--PN} camera on XMM--{\it
%Newton}, not one was visible in over 100~ksec of observation, even
%though three of the them have velocity dispersions high enough that
%they would easily be visible if their luminosities scaled with their
%velocity dispersions in the same way as nearby groups' luminosities
%scale.  I therefore find tentative evidence that groups at
%intermediate redshift are underluminous relative to their local
%cousins.

%Previous work
%examining the gravitational lensing signal of the \textsl{CNOC2}
%groups has shown that they are likely to be genuine, gravitationally
%bound objects.  

%The \textsl{CNOC2} Field Galaxy Redshift Survey
%identified several dozen associations of galaxies, claimed to be
%gravitationally bound groups, at redshifts $0.1 < z < 0.6$.  Previous
%work has found a statistical gravitational lensing signal from these groups.

%I examine the X--ray luminosity of galaxy groups in the
%\textsl{CNOC2} survey, at redshifts $0.1 < z < 0.6$.  Previous work
%examining the gravitational lensing signal of the \textsl{CNOC2}
%groups has shown that they are likely to be genuine, gravitationally
%bound objects.  Of the 21 groups in the field of view of the
%\textsl{EPIC--PN} camera on XMM--{\it Newton}, not one was visible
%in over 100~ksec of observation, even though three of the them have
%velocity dispersions high enough that they would easily be visible if
%their luminosities scaled with their velocity dispersions in the same
%way as nearby groups' luminosities scale.  I consider the possibility
%that this is due to the reported velocity dispersions being
%erroneously high, and conclude that this is unlikely.  I therefore
%find tentative evidence that groups at intermediate redshift are
%underluminous relative to their local cousins.

\noindent {\bf Part II}\\
%
As the emerging field of astrobiology pursues the quest to learn if
there is life on planets around other nearby stars, there are various
properties of planets that it will be important to constrain yet
difficult to measure.  I address how to constrain three nontrivial
properties of extrasolar planets: \pn{i} the composition, and \pn{ii}
the rotation rate and atmospheric dynamics of hot Jupiters; and
\pn{iii} the habitability of terrestrial planets.\\
%
\pn{i}
When a giant planet in the source plane of a gravitational microlensing
event crosses the fold-caustic of a foreground binary lens, the
planet-star brightness ratio is greatly enhanced, which in a favorable
situation could allow for a crude optical spectrum of the planet.\\
%
\pn{ii}
As a giant, tidally locked planet transits across the face of its
star, the rotation of the planet's atmosphere will impose a Doppler
shift on absorption features arising from the planet's atmosphere
relative to where the features would be on a non-rotating planet. With
future telescopes that have larger collecting areas, we will be able
to place constraints on the rotation rate of nearby transiting hot
Jupiters.\\
%
\pn{iii}
Finally, I revisit the habitability of terrestrial planets around
sun-like stars: Using a 1-dimensional energy balance model, I
investigate how the temperature distribution is likely to depend on
many factors, including the covering fraction of ocean, the rotation
rate, the obliquity, and more.\\
