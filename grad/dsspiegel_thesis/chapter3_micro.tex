\chapter[Can We Probe the Atmospheric Composition of an Extrasolar Planet from its Reflection Spectrum in a High--Magnification Microlensing Event?]{Can We Probe the Atmospheric Composition of an Extrasolar Planet from its Reflection Spectrum in a High--Magnification Microlensing Event?}
\label{ch:micro}
\markright{}



\section{Introduction}
\label{micro_sec:intro}
With the first discovery, 16 years ago, of a planet orbiting a star
other than our Sun \citep{wolszczan+frail1992}, astronomy finally
entered an age in which we could hope to answer scientific questions
about distant planets, with the ultimate aim of detecting and
characterizing an extrasolar Earth-like planet.  Among the most
tantalizing questions is: What is the chemical composition of an
extrasolar planet?  In this paper, we suggest a way to test for the
presence of certain compounds in the atmospheres of extrasolar planets
at much greater distances than has previously been discussed.

While the vast majority of the $>280$ currently known extrasolar
planets have been discovered in radial velocity surveys (e.g., see
Marcy \& Butler, 1998 or Woolf \& Angel, 1998 for reviews
\footnote{For up--to--date information on the current status of these
searches, see http://exoplanet.eu and http://exoplanets.org.}), two
methods have recently been proposed to search for extrasolar planets
via their gravitational microlensing signatures.  These methods are
complementary to the radial velocity surveys, in that they can detect
planets at larger distances, well beyond our solar neighborhood, and
one of these methods has the advantage of potentially providing
information about the spectrum and therefore the composition of the
planet (transit surveys have the same two advantages;
e.g. \citet{charbonneau_et_al2002}).

\citet{mao+paczynski1991} and \citet{gould+loeb1992} suggest that as a
background star passes behind a lens--star with a companion planet,
the planet could be detected as lens, since it will cause a secondary,
sharp spike in the source star's light--curve.  Indeed, two years ago,
a lens-plane planet was finally discovered \citep{bond_et_al2004}.
Recently, \citet*[hereafter GG00]{graff+gaudi2000} and
\citet{lewis+ibata2000} have suggested that extrasolar planets might
instead be detected in the source-plane, as they cross the caustics of
a foreground lens system and are highly magnified relative to their
parent star (\citet{heyrovsky+loeb1997} also discuss from a
theoretical perspective the possibility of using microlensing
light--curves to probe structures in the source plane, in their case
the structure of a background star behind a point--like lens, and
other authors since then have carried out such studies,
e.g. \citet{albrow_et_al1999}, and \citet{castro_et_al2001}).

While detecting the planet as a lens, as Mao \& Paczynski suggest, has
the potential to reveal a statistically important sample of extrasolar
planets, the drawback is that we receive no information about the
planets except for perhaps their masses and projected separations from
their host stars.  Reflected light from a planet, however, contains
information about physical parameters of the planet (presence and
sizes of rings, satellites, spots and bands, for example).  Detecting
a planet as a lensed source therefore holds the promise of allowing
these parameters to be measured, as suggested by \citet[hereafter
GCH03]{gaudi_chang_han2003} (\citet{lewis+ibata2000} suggest further
that polarization fluctuations during microlensing events could be
indicative of properties of planetary atmospheres).  In the present
work, we investigate the viability of detecting an extrasolar planet
as a microlensed source, and the extent to which a measurement of the
magnified reflection spectrum can be used to glean information about
the planet's atmospheric composition.

An unperturbed, isolated point--like lens (such as a single planet or
a star) produces a point--like caustic.  A binary lens, however, can
produce a closed caustic curve, consisting of a set of piecewise
concave curves that meet in cusps.  In the present context, a binary
lens, then, has several advantages over a point--lens: first, the
relatively large spatial extent (compared to a point) of the binary
lens caustic implies a much larger region in the sky in which for a
high--magnification event to occur; second, since the caustic of a
binary lens is a closed curve, caustic crossings come in pairs, and
the second crossing can be anticipated; third, both star and planet
can cross the caustic of a binary lens, while it is unlikely that both
would cross the point--caustic of a point lens.  When a background
star with a companion planet crosses the caustic of a binary lens, a
unique observational signature will be produced in the light--curve.
If such a signature is detected on ingress (or, if the lensed
light--curve shows, at least, that the star has entered the
inter--caustic region of a binary lens), GG00 suggest that many
observatories could train their telescopes on this system so as to
obtain dense sampling of the light--curve at egress (exiting the
caustic region).  If the planet's reflected light is sufficiently
magnified, multi--color light--curves, or even detailed time-dependent
spectra might, in principle, be obtained.  Such spectral binning of
the signal would shed light on the wavelength-dependence of the
planet's albedo, which could in turn yield information about the
chemical composition of the planet's atmosphere.

GCH03 suggest that morphological features such as moons or rings
around extrasolar planets may be detectable, and they find a
signal-to-noise ratio of $\sim 15$ for $I$-band detection of a planet
in a typical planet-star-lens configuration with a 10m telescope.  If
the light (in a given wavelength range) is split up into $N$ bands,
the signal-to-noise ratio should go down roughly as $1/\sqrt{N}$.
Signal-to-noise is also directly proportional to the diameter of the
telescope's aperture.  This suggests that with a 10m--class telescope,
light could be split up into a few broad spectral bins before
signal-to-noise becomes unacceptably low, and motivates us to examine
whether useful information about the atmospheric composition of the
planet could be obtained with this method.

The rest of this paper is organized as follows. In \S~\ref{micro_sec:model},
we present our model of a planetary caustic--crossing, including a
detailed discussion of both the model of the planet and the
computation of the caustic--crossing light--curve. In
\S~\ref{micro_sec:detectability}, we discuss the detectability of extrasolar
planets through microlensing. In \S~\ref{micro_sec:prspectra}, we describe
the albedos and reflection spectra of gas--giant planets in our own
solar system. In \S~\ref{micro_sec:specres}, we analyze the possibility of
determining the wavelength--dependence of the albedo of a microlensed
extrasolar planet.  In \S~\ref{micro_sec:discussion} we present a detailed
discussion of the factors that affect the S/N of the detection of
planets with various features in their reflection spectra.  Finally,
in \S~\ref{micro_sec:conc}, we discuss the limitations of current
technology, and conclude with projections of what may be possible with
future instruments.

\section{Modeling Planetary Caustic--Crossing Events}
\label{micro_sec:model}

\subsection{The Planet--Star System}
\label{micro_ssec:pssystem}

\begin{figure}[p]
\plotone{figures/chapter3_micro/fig1.eps}
\caption[Schematic illustration of the planet--star system in the
``planet--leading'' configuration.]{Schematic illustration of the
planet--star system in the ``planet--leading'' configuration.  The
planet--star system is moving to the right, while the caustic stays
still.  In the reverse (``planet-trailing'') configuration, the planet
would be to the left of the star as they move to the right.}
\label{micro_fig:intercaustic} 
\end{figure}
\afterpage{\clearpage}

We consider a star with a companion planet as it crosses the
fold--caustic of a binary lens.  Figure \ref{micro_fig:intercaustic} shows
an illustration of the configuration we model.  The observed
surface--brightness of the planet at a given wavelength depends on
properties of the star, the planet, and their relative geometry --
specifically, the stellar flux, the albedo and phase of the planet,
and the reflection or scattering properties of its atmosphere.  For
the stellar spectrum, we adopt that of a G0V star, and for the
wavelength--dependent albedo of the planet's atmosphere, we use the
gas giants in our own solar system as a guide (both to be described in
more detail in \S~\ref{micro_sec:prspectra} below, where spectral features
are considered).  The planetary phase is given by the angle $\phi$
between the line--of--sight and the ray from the star to the planet
(e.g. $\phi=0\degr$ corresponds to the ``full-moon'' phase), as
described in, e.g., GG00, GCH03, and \citet{ashton+lewis2001}).  GCH03
adopted and compared two simple reflectance models (uniform and
Lambert scattering) that prescribe the angular dependence of
reflectivity; and \citet{ashton+lewis2001} considered the effects of
planetary phase.  Neither of these studies, however, considered
simultaneously the effects of the planet's phase and its reflectance
model on the lensed light--curve.  In our studies, we compared three
different reflectance models: uniform, Lambert, and Lommel-Seeliger
reflection (see, e.g., \citet{efford1991} for more detailed
discussions of these models, and see the Appendix for details on the
computation of planet-models).

In Figure~\ref{micro_fig:planets}, we illustrate the surface brightness maps
of three planets, one for each reflectance model described, each at
fixed phase $\phi=45\degr$.  The maps were created numerically on a
square grid of $401\times401$ pixels that we find to be sufficiently
fine to converge on the light--curves we obtain below.  We concur with
GCH03 that with current technology it would be impossible to infer the
true reflectance model of an extrasolar planet during a microlensing
event, and so we use only one model, Lommel-Seeliger reflectance -- 
which we expect to be the most realistic one, in calculating the
light--curves that we present below.

\begin{figure}[p]
\plotthree{figures/chapter3_micro/fig2a.epsf}{figures/chapter3_micro/fig2b.epsf}{figures/chapter3_micro/fig2c.epsf}
\caption[Illustration of the surface brightness of 3 model
planets.]{Illustration of the surface brightness of planets
illuminated by a star at a phase angle of $\phi=45$ degrees.  The
three panels assume different reflectance models. {\em Left panel:}
uniform illumination; {\em Middle panel:} Lambert reflectance; {\em
Right panel:} Lommel--Seeliger reflectance.  The color maps differ
slightly between the different models; all three scatter approximately
equal amounts of light toward the viewer.}
\label{micro_fig:planets} 
\end{figure}
\afterpage{\clearpage}


\subsection{Modeling the Caustic--Crossing Event}
\label{micro_ssec:ccevent}
For a good description of the details of gravitational microlensing,
including see, for example, \citet{mao+paczynski1991}; for details
regarding the generation and shape of fold-caustics, see GG00.  We
assume that the planet--star system described in
\S~\ref{micro_ssec:pssystem} above is in the source plane of a binary lens.
The lensing stars are massive enough and close enough to one another
that they generate a fold--caustic in the source-plane, a closed curve
of formally infinite magnification.  The caustic is considered to be a
straight line (we follow GG00 and GCH03 and assume a low probability
of crossing the caustic near a cusp) that sweeps across the planet and
star.  We assume that the plane of the planet's orbit is edge--on and
is normal to the caustic at the point where the star-planet system
crosses; we will argue in \S~\ref{micro_sec:specres} below that this
simplification is not critical to our results.  A source is magnified
by the binary lens proportionally to the inverse square-root of the
source's distance from the caustic when it is in the inter--caustic
region (ICR) and not otherwise.  We use the discretized magnification
equation given by \citet{lewis+belle1998} and
\citet{ashton+lewis2001}:
%
\begin{equation}
\label{micro_eq:ampd}
\mathcal{A}_{\rm{pix}}(x_k) = 2 \frac{\kappa}{\Delta x} \left(
        \sqrt{x_k + \Delta x} - \sqrt{x_k} \right) + \mathcal{A}_0,
\end{equation}
%
where $\mathcal{A}_{\rm{pix}}$ is the magnification of the pixel,
$\kappa$ is a constant close to unity that represents the ``strength''
of the caustic, $\Delta x$ is the width of a single pixel, and $x_k$
is the distance of the $k$th pixel from the caustic.  In
equations~(\ref{micro_eq:ampd}), distances are measured in units of the
Einstein radius of the lens system: $\theta_E = \sqrt{2
R_{\rm{Sch}}/D}$, where $R_{\rm{Sch}} = 2 G M/c^2$ is the
Schwarzschild radius of the lens, and $D \equiv D_{\rm{os}}
D_{\rm{ol}}/D_{\rm{ls}}$ ($D_{\rm{os}}$, $D_{\rm{ol}}$, and
$D_{\rm{ls}}$ are the distances between the observer and source, the
observer and lens, and the lens and source, respectively).  In our
model, we use equation~(\ref{micro_eq:ampd}) to compute the brightness of
each pixel across the face of the planet.  We then sum the
contributions from all of the $401\times401$ pixels to determine the
total brightness at a given position $x$ (corresponding to a given
time during the lensing event).


\subsection{Preview of Results}
\label{micro_ssec:preview}
Using the model described above, we study a number of different
scenarios. In all cases, we assume that the source star is a clone of
HD209458, a G0V 8 kpc away with a companion planet that has the
properties of HD209458b, i.e. a ``Hot Jupiter,'' with radius $R_p =
1.35$ times the radius of Jupiter, and with orbital radius $a = 0.046$
AU (for details of the discovery of HD209458b, see
\citet{henry_et_al2000}; \citet{charbonneau_et_al2000}).  In order to
reproduce the published results of GCH03, we assume generous viewing
conditions with albedo $A = 1$ (all incident light is reflected).  To
model a more realistic situation, we examine several other albedo
models, including Jupiter's albedo and a ``gray atmosphere'' -- a
constant, wavelength--independent albedo.  The lens stars are assumed
to be typical bulge stars, 6~kpc away, each with a mass of $0.3{\rm
M_\odot}$.

We first consider a simple estimate of the flux from such a system.
The un--magnified flux from a solar-type star 8 kpc away is quite low;
$F_* = L_\sun / 4 \pi (8 \rm{kpc})^2 \approx 5 \times
10^{-13}~\rm{erg}~\rm{cm}^{-2}~\rm{s}^{-1}$. Since a typical photon
($\sim 500$ nm) from such a star carries about $4 \times 10^{-12}$
ergs, this flux corresponds to a photon number flux of approximately
$0.1 \rm{photons}$ $\rm{cm}^{-2}$ $\rm{s}^{-1}$.  Even a large,
close--in planet, such as the one under consideration, subtends a
small solid angle from the star's perspective; so even with an albedo
of $A=1$, the flux of photons from the planet is reduced by a factor
of $\gtrsim 10^4$ from the stellar flux. As a result, the flux from
the planet is $\approx 6 \times 10^{-6}$ photons $\rm{cm}^{-2}$
$\rm{s}^{-1}$.  This is the well--known reason why gravitational
microlensing is essential for detecting the reflected light of a
planet around such a distant star.

Crossing a fold-caustic can lead to impressive magnification.  In the
situation under consideration, the Einstein radius of the lens is
approximately 4000 times the radius of the planet, which means that,
according to equation (\ref{micro_eq:ampd}), a magnification factor of
$\mathcal{A} \sim \sqrt{4000} \approx 60$ can be
achieved.\footnote{Our calculations agree with those of
\citet{kayser+witt1989}, and indicate that the maximum effective
magnification of a uniform disk is $\sim 1.4/\sqrt{\rho}$, where
$\rho$ is the disk radius in units of the Einstein radius of the lens.
The maximum effective magnification is slightly greater for Lambert
and for Lommel-Seeliger scattering.  Thus, an effective magnification
factor of $\sim 1.5 \times \sqrt{4000} \approx 100$ can be achieved.}
As a result, as shown in the lightcurves below, the planet can perturb
the total flux by as much as $\sim 1\%$.

Blending of background starlight in crowded fields makes detecting
microlensing events more difficult.  For details, see GG00 and GCH03.  We
follow GCH03 and ignore blending, for with good seeing its effect is
negligible.

Note that the order in which the star and planet cross the caustic
matters a great deal for the detectability of the planet.  There are
two basic ways in which the planet-star system can be configured as it
crosses the caustic region -- planet leading star or planet trailing
star.  Since a planet will almost surely not be detected on ingress,
as the system enters the ICR, the favorable configuration for
detecting a planet is the planet trailing the star on egress, so that
the star is not magnified as the planet crosses the caustic.  If the
planet is leading the star on egress, the configuration is much less
favorable for detecting the planet.

Finally, consider a future microlensing survey that uses a telescope
large enough to discern the ingress signature of a planet crossing the
caustic.  Then, in a fraction of star-planet systems that cross
fold-caustics, the system could be in the favorable configuration for
both caustic--crossings (i.e., with the star outside the ICR when the
planet crosses the caustic).  This is because the ICR--crossing--time
($\sim 3-4$ days) is comparable to the semi-orbital period of a
close-in extrasolar planet. For example, consider a planet with
orbital period $\sim 6$ days, twice the ICR--crossing time of $\sim 3$
days.  In this case, there should be a $\sim 50\%$ chance that the
planet will be in the planet-leading configuration on ingress and in
the planet-trailing configuration on egress after having traversed
half an orbit (and an equal chance of being unfavorably oriented both
in ingress and egress); and so a planet \emph{could} be detected on
both ingress and egress.  Clearly, the actual likelihood of catching
the same planet on both ingress and egress crossings depends on the
poorly known distribution of orbital radii for both the planets and
for the binary lenses, but it is unlikely that the probability is
negligibly small.  While the coincidence between the orbital and
intra-caustic-region-crossing timescales is interesting, we note that,
in practice, a planet is unlikely to be detected on ingress -- unless
a deep future survey is devoted to blind monitoring of stars for
lensing at $\sim$ hour time--resolution.



\section{Detectability}
\label{micro_sec:detectability}
Detecting the presence of a planet is, of course, challenging, since
even when the planet is on the caustic, its flux is a small fraction
($\lsim 1\%$) of even the un--magnified flux from the star.  As an
example, in the inset in Figure~\ref{micro_fig:tailhead}, we present a model
$R$--band light--curve for a 10m telescope, showing first the star,
and then the planet exiting the ICR in the favorable configuration for
detecting the planet.  The planet is modeled with Jupiter's albedo
(described in \S~\ref{micro_sec:prspectra} below), corresponding to
$A\approx 0.45$. The solid dots show simulated data points.  The broad
peak between 0--2 hrs results from the star crossing the caustic.  The
three large dots at 3.0--3.3 hrs correspond to the planet crossing the
caustic.  On this scale, the magnified planetary flux is invisible
against the un--magnified star-flux.  Nevertheless, we next show that,
as we suggest in \S~\ref{micro_ssec:preview} above, with the current
generation of 10m telescopes, it is possible to detect a planet when
the star is outside the ICR (but not when the star is inside the ICR).

In Figure~\ref{micro_fig:tailhead}, the top panel shows the tail of the
light--curve (after the star has exited the ICR) for the planet's
caustic--crossing egress at $\phi=10\degr$ (i.e. a zoom--in version of
the planet signal from the inset). The bottom panel in this figure
shows a random realization of the flux from the planet for the
planet's caustic--crossing egress at $\phi=-45\degr$ (i.e. in the
unfavorable orientation).  In both panels, we show error--bars
corresponding to the $\sqrt{N}$ shot noise from the total
photon--flux.  (We ignore instrumental noise because shot noise will
dominate for bright bulge stars.)  We sum the signal and the
$\sqrt{N}$ photon noise over five 8--minute integrations around the
planetary caustic crossing. In the favorable orientation (the top
panel), we find that the planet is detectable with a total $S/N \sim
15.3$ while in the unfavorable orientation, the planet is essentially
undetectable ($S/N \sim 3.6$).

The relationship between the signal-to-noise of detection and the
phase angle is summarized in Figure~\ref{micro_fig:snvsphase} below.  The
planet--flux/star--flux ratio is maximized when the planet is in the
``full moon'' phase ($\phi \sim 0\degr$).  When the star is outside
the ICR, therefore, the planet's detectability is maximized for low
phase angles.  Phase $\phi\approx 10\degr$ is optimal (the star
intersects a fraction of the planet's surface for $\phi\lsim 8\degr$,
leading to a rapid decrease in the S/N ratio for still smaller phase
angles).  When the star is inside the ICR, however, there is a more
delicate balance.  Since magnification in the ICR decreases with
distance from the caustic, the planet's detectability is improved when
the the projected impact parameter $b$ is large, which happens for
$\phi \sim \pm 90\degr$.  These two competing factors (planet--flux
and star--flux) balance to maximize the planet/star flux ratio at
about $|\phi|\sim45\degr$.

The reader can estimate from Figure~\ref{micro_fig:snvsphase} what fraction
of the orbit will yield acceptable signal--to--noise.  The $S/N$ of
detection exceeds 5 for approximately $90\degr$ of phase, or $1/4$ of
the orbit.  Since we are inquiring what will be possible in good
viewing conditions, we hereafter will consider only the favorable
orientation ($\phi=10\degr$).

Note that, for most phases, the $S/N$ of detection is decreased if the
plane of the planet's orbit is inclined less than $90\degr$, but is
unaffected if the plane of the orbit is not normal to the caustic.

\begin{figure}[p]
\plotonesmall{figures/chapter3_micro/fig3a.ps}
\plotonesmall{figures/chapter3_micro/fig3b.ps}
\caption[Planetary lensing egress light--curves.]{Planetary lensing
light--curves on egress in the $R$-band, assuming Lommel--Seeliger
reflection off a planet similar to HD20y9458b and with Jupiter's
albedo, expressed as percentage change in the total flux caused by the
presence of the planet.  Solid (blue) curves show theoretical
light--curves; dashed (red) curves show 1-$\sigma$ errors; solid
(blue) dots show random mock data (theoretical light--curve plus
noise) with 1-$\sigma$ error bars; large (cyan) dots denote times when
the planet's surface intersects the caustic. {\em Top panel:}
Favorable (planet--trailing) orientation for detection of the planet
on egress.  The inset shows the entire caustic-crossing lightcurve for
both the star and the planet, with the ordinate showing total photons
collected in 8 minute observations with a 10~m. telescope; the big
plot is a zoom-in of the tail end of the inset.  Dashed lines indicate
the relationship between the inset and the parent image.  An optimal
phase angle of $\phi=10\degr$ was assumed. {\em Bottom Panel:}
Unfavorable (planet--leading) orientation on egress, with the optimal
value of $\phi=-45^\circ$.}
\label{micro_fig:tailhead} 
\end{figure}
\afterpage{\clearpage}

\begin{figure}[p]
\plotone{figures/chapter3_micro/fig4.ps}
\caption[S/N vs. phase angle.]{This figure shows the dependence of
$S/N$ on phase angle for $R$-band detection of a planet in a clone of
the HD209458 system at 8~kpc, lensed by binary $0.3~M_\sun$ stars at
6~kpc, in 8-minute integrations with a 10~m telescope.  The plane of
the planet's orbit is assumed to be inclined $90\degr$ to the
line-of-sight (edge--on) and normal to the fold-caustic at the point
of crossing.  Typical proper-motion of lens and source (e.g. GCH03) is
assumed.  Since the plane of orbit is at inclination $90\degr$, the
planet disappears behind the star for phase angles $|\phi|\lsim
8\degr$ and detectability drops to zero.}
\label{micro_fig:snvsphase}
\end{figure}
\afterpage{\clearpage}



\section{Planetary Reflection Spectra}
\label{micro_sec:prspectra}
HD209458 is one of a relatively small number of stars confirmed to
have a transiting companion (HD209458b).  By carefully comparing the
spectrum of this star during a planetary transit against its spectrum
outside transit, \citet{charbonneau_et_al2002} measured how the
opacity of the transiting planet's atmosphere varies with wavelength,
and inferred the presence of sodium in the atmosphere.  Performing a
similar analysis, \citet{vidal-madjar_et_al2003} claim to find an
extended hydrogen Ly$\alpha$-emitting envelope surrounding the planet.

We here investigate the prospects of analogously observing, instead of
a transmission spectrum during a transit event, a reflection spectrum
during a caustic--crossing event.
%
Although near--future ground--based coronographs (such as
Lyot\footnote{http://lyot.org}) and more distant future space
projects, such as the Terrestrial Planet
Finder\footnote{http://planetquest.jpl.nasa.gov/TPF/tpf$\_$index.html},
will be able to probe the spectra of planets around nearby stars using
coronographs or nulling interferometers (e.g.,
\citet{kuchner+Traub2002}), we are not aware of any other ways at this
time to study the reflection spectra of extrasolar planets.  We
emphasize that the technique we present in this paper is not in
competition with current coronographic work, but is rather in several
senses complementary (for more information on coronographic
techniques, see, e.g., \citet{oppenheimer_et_al2001}).  First, current
and near--future coronographic studies are not sensitive to close--in
planets, because these planets lack sufficient angular separation from
their host--stars, but these are precisely the planets that are most
readily seen in the source--plane of a microlensing event.  Second,
the population of planets available to microlensing is in the Galactic
bulge, or at a distance $\sim8{\rm~kpc}$, and is therefore
complementary to the nearby population of planets that will eventually
be available to the other methods just mentioned.

As a first step toward modeling the reflection spectrum of an
extrasolar planet around a solar-type star, we adopt the reflection
spectra of the Jovian planets in our solar system, because these are
the only gas-giant planets whose wavelength-dependent albedos have
ever been measured.  Atmospheric conditions, and hence reflection
spectra, of hot Jupiters (extrasolar giant planets with short orbital
periods) are likely to be much different from those of Jupiter,
Saturn, Uranus, and Neptune (for detailed discussions of hot Jupiter
atmosphere models, see, e.g., \citet[hereafter
SBH03]{sudarsky_et_al2003}; \citet{burrows_et_al2004}; and
\citet{seager_et_al2000}).  However, given the uncertainty and
differences among published atmospheric models of extrasolar giant
planets, we prefer to base our calculations on the unambiguously
measured albedos of the solar--system gas giants.  We will then
discuss (at the end of \S~\ref{micro_sec:discussion} below) the expected
differences for the hot--Jupiter atmospheres, and identify features in
the theoretical spectra that could be detected at a similar
significance.

To obtain our desired reflection spectra, we need the spectrum of a
G2V star, and the albedos of the gas giants in our solar system (with
albedo defined as the ratio of reflected flux to incident flux).  We
obtained an incomplete G2V spectrum from Greg Bothun's
webpage\footnote{\textsl{http://zebu.uoregon.edu/spectrar.html}}, that
had data missing at wavelengths of strong atmospheric absorption.
Regions of missing data up to 1050 nm were filled in with a best-fit
$T=6000~{\rm~K}$ blackbody spectrum, and the spectrum was normalized
to a peak value of unity for clarity of presentation (see
Fig.~\ref{micro_fig:reflect}).  Planetary albedos are taken from
\citet{karkoschka1994}, interpolated on a cubic spline (every 5nm) to
the G2V reference wavelengths, and are shown for the four gas giants
in Figure~\ref{micro_fig:albedos}.  Reflection spectra (in arbitrary units),
then, are just the product of the albedo and the solar spectrum (shown
by the bottom curve in Fig.~\ref{micro_fig:reflect}).

\begin{figure}[p]
\plotone{figures/chapter3_micro/fig5.ps}
\caption[G2V spectrum and reflection spectrum off Jupiter.]{Spectrum
of a G2V star (top, dashed-dotted magenta curve) and reflection
spectrum of a G2V off a planet with Jupiter's albedo (bottom, blue
curve).  Inset: zoom--in of Jupiter's reflection spectrum over the
wavelength range 870--1050nm.}
\label{micro_fig:reflect} 
\end{figure}
\afterpage{\clearpage}

\begin{figure}[p]
\plotone{figures/chapter3_micro/fig6.ps}
\caption[Wavelength-dependent albedos of giant planets.]{Albedos of
Jupiter (thick blue curve), Saturn (thin black curve), Uranus (dashed
red), and Neptune (dashed-dotted green), adopted from
\citet{karkoschka1994}.}
\label{micro_fig:albedos} 
\end{figure}
\afterpage{\clearpage}

While a reflection spectrum of an extrasolar planet with a high
signal-to-noise ratio, covering a full range of wavelengths from the
visible into the near infrared (NIR), would be ideal (see discussion
in \S~\ref{micro_sec:specres} below), certain bands of the visible and NIR
spectrum provide more information about chemical composition than
others.  By comparing the reflected flux from wavelength ranges where
Jupiter's albedo is low with reflected flux from comparable wavelength
ranges where Jupiter's albedo is much higher, we can infer the
presence of those compounds responsible for the low albedo.  It is
clear from the bottom (blue) curve in Figure~\ref{micro_fig:reflect} that in
a narrow band around 900nm (880nm-905nm) and in a slightly wider band
around 1000nm (980nm-1030nm), Jupiter's albedo is quite low ($\sim
0.05$ and $\sim 0.1$, respectively); while in--between (920nm-950nm),
its albedo is much higher ($\sim 0.45$).  These troughs are caused by
absorption by methane in the Jovian atmosphere \citep{karkoschka1994}.
This stark contrast in albedo between adjacent wavelength bands
suggests a way to search for, e.g., methane (or other elements or
compounds that are expected, in theoretical models for hot Jupiters,
to cause features with a similar equivalent width; see discussion
below) in the atmosphere of an extrasolar planet.

We note that extrasolar giant planets can orbit very close to their
host star (0.05 AU or less), but thermal emission from a planet would
nevertheless contribute negligibly to the reflected flux at
wavelengths $\sim1\mu$m even for a hot planet ($\sim 1500$K).


\section{Modeling Spectra During Caustic--Crossing}
\label{micro_sec:specres}
If the (unlensed) flux from the star has spectrum $F_*(\lambda)$ and
the planet has wavelength-dependent albedo $A(\lambda)$, then the flux
from the planet may be written
%
\begin{equation}
\label{micro_eq:Fp}
F_p(\lambda,t) = F_*(\lambda) A(\lambda) f(t),
\end{equation}
%
where the multiplicative function $f(t)$ depends on various
geometrical factors (the solid angle that the planet subtends from the
perspective of the star, whether the planet has moons or rings, how
far the planet is from the caustic, etc), and also on the reflectance
model.  The total observed flux, therefore, can be written (in the
favorable orientation, with the star un--magnified, as discussed
above) as
%
\begin{eqnarray}
\nonumber F_T(\lambda,t) & = & F_*(\lambda) + F_p(\lambda,t) \\
\label{micro_eq:Ft}
 & = & F_*(\lambda) ( 1 + A(\lambda)f(t) ).
\end{eqnarray}
%
The observables are $F_T$ and $F_*$.  The physically interesting
characteristics of the planetary system, however, are $A(\lambda)$ and
$f(t)$, and these may be solved for as
%
\begin{equation}
\label{micro_eq:Af}
A(\lambda)f(t) = \frac{F_T(\lambda,t)}{F_*(\lambda)} - 1 \equiv G(\lambda,t),
\end{equation}
%
where we define the function $G$ as the observable quantity
constructed on the right--hand side of equation~(\ref{micro_eq:Af}).  With
perfect data, the time-difference between the star's and the planet's
caustic crossings breaks the apparent degeneracy between $A$ and $f$
in the general solution,
\begin{eqnarray}
\label{micro_eq:A}
A(\lambda) & = & k_1 \mbox{exp}\left[\int\frac{\partial G/\partial \lambda}{G}d\lambda\right]\\
f(t) & = & k_2 \mbox{exp}\left[\int\frac{\partial G/\partial t}{G}dt \right],
\end{eqnarray}
(where $k_1$ and $k_2$ are constants of integration such that $Af = G$).

In practice, with data as noisy as can be expected with the current
generation of telescopes, it is impossible to separate $A$ from $f$,
and $A$ may be determined only given a model for $f$.  Still, it is
possible in principle to posit a model for $f$ (as outlined in
\S~\ref{micro_sec:model} above) and then to solve for $A(\lambda)$.  In this
case, since the signal--to--noise ratio for the detection of the
planet we find is only $\sim 15.3$, it is still only possible to split
the light into a few broad spectral bands, rather than into a resolved
spectrum.

In order to test the idea that we could look for the spectral
signature of a particular compound in the reflected light from a
distant extrasolar planet, we model a planet with Jupiter's
reflection-spectrum and scrutinize the model data for evidence of
methane.  In order to maximize signal--to--noise, we assume an egress
caustic--crossing with planetary phase $\phi=10\degr$.

To search for signatures of methane, we construct a mock ``methane
band filter'' (hereafter ``MBF''), that allows complete transmission
from 880nm-905nm \emph{and} from 980nm-1030nm (the bands where
Jupiter's albedo is low because of methane, as discussed in \S
\ref{micro_sec:prspectra}) and zero transmission elsewhere (MBF is therefore
a ``double top-hat'' filter).  Note that we do not necessarily mean a
physical filter here; we effectively assume that the flux in a
low-resolution spectrum can be binned and computed in these wavelength
ranges. A more realistic analysis would have to take into account the
additional instrumental noise in any physical implementation of such a
filter (such as read--out noise in the case of a spectrograph).  We
then compare the MBF light--curve of a model planet with Jupiter's
albedo to the MBF light--curve of a model planet with the methane
feature removed -- i.e., a model planet where the albedo is replaced
by a constant equal to Jupiter's mean albedo $\bar{A}=0.45$.  Figure
\ref{micro_fig:methalb} shows this comparison: the top panel shows the MBF
light--curve for a planet with Jupiter's albedo (here, the planet is
detected at $S/N=1.8$, which counts as a non-detection); the bottom
panel shows the MBF light--curve for a planet with constant albedo
$A=0.45$ (here, the planet is detected at $S/N=8.4$).

In practice, the observational strategy would involve employing a
``high albedo filter'' (hereafter ``HAF'') that uses a region of the
spectrum that is relatively unaffected by methane and that is
comparable in width to the MBF filter (e.g. the adjacent 920nm-950nm
region, and/or other regions where Jupiter's albedo is high).  The
flux measured through the HAF filter would then be used to predict the
expected MBF flux according to the no--methane null--hypothesis.  In
practice, then, a \emph{non}--detection of the planet in the MBF band
together with a simultaneous \emph{detection} in the HAF band, would
be evidence for the presence of methane in the atmosphere of the
planet.
%
\begin{figure}[p]
\plotonesmall{figures/chapter3_micro/fig7a.ps}
\plotonesmall{figures/chapter3_micro/fig7b.ps}
\caption[Egress light-curves in MBF band.]{Egress light--curves in the
MBF band, which covers two deep methane absorption features and
includes light from 880nm-905nm and from 980nm-1030nm.  The meaning of
the symbols are as in Figure~\ref{micro_fig:tailhead}.  {\em Top
panel:} Jupiter's low albedo is adopted, which leads to a
non--detection of the planet.  {\em Bottom panel:} A constant albedo
of $A=0.45$ is used showing what the light--curve would look like if
there were no methane present ($S/N = 8.4$).  This plot is quite
similar to the light--curve that would be obtained through a filter in
a band where there is low methane absorption and Jupiter's albedo is
much higher ($\sim0.4-0.5$).}
\label{micro_fig:methalb} 
\end{figure}
\afterpage{\clearpage}
%
The $S/N$ is proportional to the square root of the number of photons
collected, and to the diameter of the telescope.  With a future 30 or
100m telescope, therefore, it would be possible to achieve a $S/N$ of
$\sim 25$ or $\sim 80$, respectively, in detecting the presence of
methane.



\section{Discussion}
\label{micro_sec:discussion}
Note that, strictly speaking, our $S/N$ calculations are for a
space-based observatory, because we do not include sky brightness.
Detailed data on sky brightness are available in
\citet{leinert_et_al1998}.  In R-band, the contribution to total flux
from the sky is small for good seeing (for seeing $\sim0.75\arcsec$,
the star is more than an order of magnitude brighter than the sky
within the aperture subtended by the star).  At $900$nm, the star is
still several times brighter than the sky for good seeing conditions,
but by $1\mu$m the sky is comparably bright to the star, which would
increase the noise in an observation by a factor of $\sim\sqrt{2}$ and
would therefore decrease the $S/N$ by the same factor.  For a 10m
telescope, this still indicates $S/N\sim6$ for good seeing conditions
in the situation modeled above.  Note that if the plane of the
planet's orbit is inclined less than $90\degr$, then for nearly all
phases the $S/N$ for detection of the planet is reduced, for any
spectral filter.  The ratio of the flux from the planet through the
MBF to that through the HAF, however, is independent of both
inclination and phase.

Although some models of close--in extrasolar giant planets predict
significantly less methane than is present in Jupiter's visible
cloud-layer, this prediction is not universal.
\citet{seager_et_al2000}, for example, present a model of 51 Peg b
that has spectroscopically significant methane--levels.  They point
out, however, that these methane features might be present only in the
coolest, least absorptive models.

Even if future models should converge upon the conclusion that the
gaseous methane content of hot Jupiters is very low, there are
spectral features due to other chemicals that are predicted to be
present in the atmospheres of close--in extrasolar giant planets, that
are predicted to be comparably strong to Jupiter's methane features.
A search for these other predicted features would be analogous to the
methods described above.  SBH03 identify five classes of extrasolar
giant planets, ranging from class I (Jupiter--like) through classes IV
($a\sim0.1$AU) and V ($a\sim0.05$).  A caveat introduced by SBH03 is
that, for class IV and V planets, the planet's spectrum redward of
$\sim500$nm includes increasing levels of thermal emission, and so it
makes more sense to discuss the ``emergent spectrum'' rather than the
reflection spectrum.  Their model emergent spectra for class IV
planets include several strong absorption features. In the visible,
sodium ($\sim600$nm) and potassium ($\sim800$nm) are predicted to
induce absorption features with a comparable equivalent width to the
methane features we consider above; in the NIR ($\sim1.4\mu$m), water,
which is thought to condense too deep in Jupiter to affect the
cloud-top albedo, is predicted to cause an even deeper (factor of
$\sim100$) trough in the emergent spectrum.  This water feature is at
a wavelength where the Earth's sky is fairly bright (an order of
magnitude or more brighter than the star), which would make it
difficult to discern from ground-based observations but which should
pose no difficulty for a space-based telescope.  Table
\ref{ta:eqwidths} summarizes the strengths of the three absorption
features predicted in SBH03 class IV planets described above and
compares them to the methane features previously considered.  If a
planet is detected with high $S/N$ in a HAF but is (un)detected in a
filter centered on a spectral feature with $S/N$ much less than the
number quoted in the last column of Table \ref{ta:eqwidths}, this
would be evidence for the presence of the chemical responsible for the
feature.


The right-hand side of Figure \ref{micro_fig:snvsphase} is summarized and
generalized in Equation~\ref{micro_eq:SNcalc} below, which gives a rough
estimation of the expected signal--to--noise for detection of a planet
whose orbit is at inclination $90\degr$, in which the source is a
Main-Sequence star.  The lensing configuration is taken to be the one
described above.  $D$ is the diameter of the telescope's aperture,
$EW_f$ is the equivalent width of the spectral filter used, and $EW_l$
is the equivalent width of a spectral line or feature.
\begin{equation}
\label{micro_eq:SNcalc}
\frac{S}{N} \sim \left( 16 - \frac{\phi}{9\degr}\right) \times
      \Theta(\bar{A}, D, EW_f, EW_l) \times \Psi(M_*, R_p, a)
\end{equation}
where $\Theta$ and $\Psi$ are the functions given below:
\begin{eqnarray}
\label{micro_eq:Theta}
\Theta(\bar{A}, D, EW_f, EW_l) = \frac{\bar{A}}{0.45} \times
                                 \frac{D}{10{\rm~m}}
                                 \sqrt{\frac{EW_f}{150{\rm nm}}}
                                 \left(1-\frac{EW_l}{EW_f}\right) \\
\label{micro_eq:Psi}
\Psi(M_*, R_p, a)  = 10^{-3} \left( \frac{2M_*}{M_\odot} -0.8 \right)
                            \left( \frac{R_p}{R_J} \right)^{3/2} 
                            \left( \frac{a}{1~{\rm AU}}\right)^{-2}.
\end{eqnarray}
Observe that both $\Theta$ and $\Psi$ are unity in the case of an
$R$-band observation through a 10 m telescope of the planetary system
considered above.

Equation \ref{micro_eq:SNcalc} slightly over-predicts $S/N$ for filters on
the red side of $R$-band and slightly under-predicts $S/N$ for filters
on the blue side; furthermore, as above, it does not include
sky-brightness, which is particularly important redward of $1\mu$m.


%\begin{deluxetable}{lcrrr}
\begin{table}[p]
\begin{center}
%\tablewidth{240pt}
%\tablecolumns{4}
%\small
%\tablecaption{Absorption Features\tablenotemark{a}}
%\tablehead{\colhead{Spectral} & \colhead{SBH03} & \colhead{Line}   & \colhead{Eq.}   & \colhead{$S/N$} \\
%           \colhead{Feature}  & \colhead{Class} & \colhead{Center} & \colhead{Width} & \colhead{Ratio}
\caption[Line-center and EW of 5 spectral features.]{~~This table
shows the line-center and equivalent width for each of 5 spectral
features.  Three of these features, the Sodium (Na), Potassium (K),
and Water features, are expected in planets classified by SBH03 as
class IV ($a\sim0.1$AU).  The other two features are the methane
features considered in detail in this paper, with data taken from
Jupiter's reflection spectrum, available in \citet{karkoschka1994}.
The last column shows the predicted $S/N$ ratio given the absence of
the chemical, and for Na and K it is computed from Equation
\ref{micro_eq:SNcalc} (using $\bar{A}=0.35$ and $EW_l=0$), while for
the Methane features it is computed from our simulations.  This ratio
is not applicable to the case of water, because the sky is too bright
for ground-based observations at this wavelength.}
\vspace{0.2in}
\begin{tabular}{lcrrr}
  \tableline
  \tableline
  Spectral       & SBH03        & Line           & Eq.         & $S/N$\\
  Feature        & Class        & Center         & Width       & Ratio\\[0.1in]
  \tableline
Sodium           &  IV          & $\sim600$nm    & $\sim80$nm  & 14  \\
Potassium        &  IV          & $\sim780$nm    & $\sim20$nm  &  7  \\ 
Methane          &  I           & $\sim990$nm    & $\sim20$nm  &  5  \\
Methane          &  I           & $\sim1.00\mu$m & $\sim40$nm  &  7  \\
Water            &  IV          & $\sim1.40\mu$m & $\sim200$nm & N/A \\
\label{ta:eqwidths}
\end{tabular}
\vspace{-0.4cm}
\end{center}
\end{table}
%\end{deluxetable}


\section{Conclusions}
\label{micro_sec:conc}
In the Galactic bulge there is a large number of stars and,
presumably, a comparably large number of planets.  With current and
future microlensing surveys in the direction of the bulge, we expect
that some solar systems will cross the fold-caustics of binary lenses.
Unfortunately, although in such events there will be two
caustic--crossings, it appears that current technology will only allow
for detection of a planet orbiting the source-star during the egress
caustic--crossing -- and furthermore only when the star-planet system
is in the favorable configuration.  Still, with its expected mean
albedo, the planet should reflect enough light that, in the case we
consider, it should be detectable for roughly $1/4$ of its orbit with
a 10m telescope.

Our results suggests that the strategy outlined by GG00 and Lewis \&
Ibata should be viable: each time a bulge star is seen to cross a
fold-caustic into the ICR, the egress event should be closely
monitored in order to detect a planet in the trailing (favorable)
configuration, should such a favorable orientation occur.  If $10\%$
of bulge stars have hot Jupiter companions, then, since a quarter of
planet-star systems will be have appropriate in the planet-trailing
configuration on egress, $\sim2-3\%$ of bulge stars that cross
fold-caustics will be seen, under close monitoring (in $\sim$ 8--min
integrations) during the egress crossing, to have planetary
companions.

If such planets are detected, it will be possible, in principle, to
determine various properties of the planet, including physical
(reflectance model, phase, angular orientation relative to the
caustic, presence of moons or rings; see GCH03) and chemical
characteristics (the presence of specific constituent compounds of the
atmosphere, as suggested by our results). Since the expected
perturbations to an observed light--curve from the physical
characteristics are either small (moons, rings, angular orientation)
or degenerate with other effects on the total brightness of the
planet, such as the planet's albedo or the solid angle it subtends
from the perspective of its star (reflectance model, phase), it will
be difficult in practice to determine these physical characteristics.
For example, if we were to observe the egress caustic-crossing light--curve of
a planet in the bulge that has rings around it, is at illumination phase
$\phi=45^\circ$, and obeys Lommel-Seeliger reflection, we would most likely
not be able to infer the presence of the rings, the phase, or the
nonuniform reflectance because the data could be fit
equally well (within error bars) by a best-fit $\phi=0^\circ$ model with no 
rings (at $\phi=0^\circ$, a Lommel-Seeliger planet is uniformly illuminated).
The expected perturbations from some atmospheric compounds, however,
are much greater (a factor of $\sim5$ or more) and do not suffer from
analogous geometrical degeneracies.  

In our example, using 8--minute observations on a 10m telescope, we
found that the presence of methane could be inferred from a
non--detection of the planet in the strong broad methane absorption
band at $\approx 0.9\mu$m, accompanied by a $S/N\sim 10$ detection in
adjacent bands.  Observations such as the ones described in this paper
will provide a crucial constraint on models of roaster atmospheres.
Then, in turn, as more accurate atmosphere models become available,
this $S/N$ could improved by fitting the data to
spectral templates with free parameters corresponding to variable
compositions.  Future generations of large telescopes might therefore be 
able to probe the composition of the atmospheres of distant extrasolar
planets.



%\appendix
\section*{APPENDIX}
\label{micro_sec:appendix}

In this Appendix, we present the specifics of our model of the planet,
including details regarding the three reflectance models we consider.
%
\begin{figure}[p]
\plottwo{figures/chapter3_micro/fig8a.eps}{figures/chapter3_micro/fig8b.eps}
\caption[Viewing Geometry.]{Viewing Geometry. {\em Left panel:} Top
view (``God's eye view'') of the viewing geometry, showing the phase
angle ($\phi$), the orbital radius ($a$), and the projected impact
parameter ($b = a \mathrm{sin}\phi$).  {\em Right panel:} Observer's
view of the planet.}
\label{micro_fig:viewgeom} 
\end{figure}
\afterpage{\clearpage}

The viewing geometry of a planet--star binary can be described in a
three--dimensional coordinate system, centered on the planet, as
illustrated in Figure~\ref{micro_fig:viewgeom}.  The $z$--axis is defined to
point toward the observer; the $x$--axis is in the direction from the
planet to the star, as projected on the sky from the perspective of
the observer; and the $y$--axis is defined by the $x$ and $z$-axes and
the usual right--hand rule.  The phase is as defined in Section
\ref{micro_ssec:pssystem} given by the angle.

The reflectance models considered are the following:
\noindent
{\it Uniform reflection:} the planet has uniform surface--brightness
as seen from the observer, regardless of phase.
%
{\it Lambert reflection:} the surface--brightness of a patch of
projected surface is proportional to the cosine of the angle between
the incident radiation and the surface--normal vector: $B \propto
\cos(\theta_{\rm{ill.}})$.  Let $\hat{x}$, $\hat{y}$, and $\hat{z}$ be the
dimensionless coordinates on the planetary surface ($\hat{x} = x/R_p$;
$\hat{y} = y/R_p$, and $\hat{z} = z/R_p$, where $R_p$ is the planet's radius).
The unit vector to the star is $\mathbf{s} = (\sin\phi,0,\cos\phi)$, and the
unit surface-normal vector is $\mathbf{N} = (\hat{x},\hat{y},\hat{z}) =
(\hat{x},\hat{y},\sqrt{1-(\hat{x}^2 + \hat{y}^2)})$, so the desired cosine
is given by $\mathbf{s} \cdot \mathbf{N} = \hat{x} \sin\phi +
\sqrt{1- \hat{x}^2 - \hat{y}^2}\cos\phi$.  A failing of the Lambert
reflectance model is that in the ``full-moon'' phase, the specific intensity
from the edge of the projected disk drops to zero, in conflict with the
appearance of the Moon and other planets in our Solar system.
%
{\it Lommel--Seeliger reflection} is a phenomenological model,
designed to reproduce the reflectance of the Moon, that also mimics
well the appearances of a number of other bodies in the Solar System.
Neither the Lambert nor the Lommel--Seeliger model -- and certainly
not the uniform model -- can capture in detail the appearance of a
patch of planetary or Lunar surface at high resolution; but the
Lommel-Seeliger model in particular is successful at reproducing at
low resolution the whole planetary disk.  The surface--brightness of a
patch of projected surface in the Lommel-Seeliger model is
proportional to the cosine of the illumination angle, and inversely
proportional to the sum of the cosines of the illumination angle and
the viewing angle: $B \propto
\cos(\theta_{\rm{ill.}})/[\cos(\theta_{\rm{ill.}}) +
\cos(\theta_{\rm{view}})]$.  The cosine of the viewing angle is the
dot product of $\mathbf{N}$ with the unit vector to the observer,
$\mathbf{z} = (0,0,1)$, or $\mathbf{N} \cdot \mathbf{z} = \sqrt{1 -
(\hat{x}^2 + \hat{y}^2)}$.

Within our coordinate system, the un--magnified flux from a patch of
surface at projected coordinates $(\hat{x},\hat{y})$ can be represented as
follows
\begin{equation}
dF = K P \frac{B(\hat{x}, \hat{y})}{4 \pi r^2} d\hat{x} \spa d\hat{y}.
\end{equation}
Here $P$ is the total incident stellar power that the planet reflects,
or $L_* A (\pi R_p^2)/4 \pi a^2$, where $L_*$ is the luminosity of the
star; $A$ and $a$ are the planet's albedo and orbital
semi-major axis, respectively; $B(\hat{x}, \hat{y})$ gives the spatial
dependence of the apparent brightness of the planet, and depends on
the reflectance model; $r$ is the distance of the observer from the
system; and $K$ is an overall scale-factor so that the total reflected
light equals the total intercepted light times the albedo $A$.  What
remain to be given, then, are $K$ and $B$ for each reflectance model.

The resulting constants and formulae are:
\begin{eqnarray}
\nonumber K_U    & = & 2/ \pi \\
\nonumber K_L    & = & 4/ \pi \\
K_{LS} & = & 1.556,
\end{eqnarray}
and
\begin{eqnarray}
\nonumber B_U(\hat{x},\hat{y}) & = & 1 \\
\nonumber B_L(\hat{x},\hat{y}) & = & \hat{x}\sin\phi + \sqrt{1-(\hat{x}^2+\hat{y}^2)}\cos\phi \\
B_{LS}(\hat{x},\hat{y}) & = & \frac{\hat{x}\sin\phi + \sqrt{1-(\hat{x}^2+\hat{y}^2)}\cos\phi}{\hat{x}\sin\phi + \sqrt{1-(\hat{x}^2+\hat{y}^2)}(\cos\phi + 1)},
\end{eqnarray}
%
where the $U$, $L$, and $LS$ subscripts refer to uniform, Lambert, and
Lommel-Seeliger reflectance, respectively.

