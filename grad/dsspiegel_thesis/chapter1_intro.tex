\chapter{Introduction}
\label{ch:introduction}
\markright{}


This thesis comprises observational work searching for the
intergalactic medium and theoretical work on extrasolar planets.

%\section{Part I: ~\large Seeking Hiding Baryons: the Intragroup Medium}
\section[Part I: ~Seeking Hidden Baryons]{Part I: ~Seeking Hidden Baryons}
\label{intro_sec:groups}
\subsection[Background and Motivation]{Background and Motivation}
\label{intro_ssec:baryons background}
A fundamental goal of cosmology is to account for the matter in the
universe (since Einstein, the matter/energy-density).  This endeavor
is difficult because nature has ways of hiding matter and energy that
are clever enough that cosmologists have remained in the dark for many
years.  A major complication was discovered by Fritz Zwicky in the
1930s, when he found that the luminous content of galaxy clusters
seems to be insufficient to bind galaxies together, given their
velocity dispersions \citep{zwicky1937,zwicky1938}.  Many different
lines of astronomical inquiry during the last 70 years have begun to
provide a framework for understanding Zwicky's observations: The
available evidence now strongly points to the daunting proposition
that not only is $73\%$ of the energy-density of the universe dark
energy, but of the remaining $27\%$ that consists of
matter
only a small fraction is the sort that we experience in everyday life
-- baryonic matter.

The energy budget of the universe, then, is apparently dominated by
forms of energy with which we are entirely unfamiliar.  Moreover,
despite some remarkable success, our current understanding of the
evolution and morphology of even the baryons in the universe is
strikingly incomplete.  Although estimates based on various kinds of
observations of the high-redshift universe are converging to form a
picture of the universe in which $\sim 4.5\%$ of the present-day
energy-density consists of baryonic matter \citep[hereafter
  FP04]{fukugita+peebles2004}, only $\sim$10\% of the predicted
baryons are detected in the local universe in the form of stars and
stellar remnants, cool neutral gas, or hot gas in galaxy clusters,
leaving nearly 90\% unaccounted (FP04; see
Figure~\ref{intro_fig:piechart} below).  This disparity leads to a
profound conclusion that is also suggested by cosmological simulations
\citep{dave_et_al2001}: At the present epoch, a large fraction of
baryons are in very low-density environments -- the warm-hot
intergalactic medium (WHIM) and the intragroup medium of small groups
of galaxies.

\begin{figure}[p]
\plotone{figures/chapter1_intro/piechart.eps}
\caption[Local universe baryon budget.]{A summary of the detailed
accounting of the baryon budget of the local universe presented in
\citet{fukugita+peebles2004} The two largest pie-wedges (Warm-Hot
Intergalactic Medim -- or WHIM -- and Group Plasma/Galactic Halos) are
almost entirely undetected and presumably contain nearly 90\% of the
baryons at the present epoch.  The sum of these wedges is known to
greater precision than either one is individually.  This is because
their sum is the difference between unity and the sum of all observed
sources of baryonic matter, whereas the breakdown into 35\% and 53\%
is motivated in part by simulations the simulations of
\citet{dave_et_al2001}.}
\label{intro_fig:piechart}
\end{figure}
\afterpage{\clearpage}

\citet{mulchaey2000} describes \textsl{ROSAT} X-ray observations that
indicate that such simulations are correct in predicting that part of
the missing baryonic matter is in the intragroup medium of small
galaxy groups: In the local universe, a majority of small galaxy
groups exhibit diffuse X-ray emission that reveals significant
quantities of baryonic matter that are not visible at optical
wavelengths.  How the universe evolved to this state, and when the
baryons began collecting in the potential wells of small groups,
however, is observationally undetermined.

\subsection[The Baryonic Content of Groups at Moderate Redshift?]{The Baryonic Content of Groups at Moderate Redshift?}
\label{intro_ssec:groups}
Part~\ref{part:IGM} describes an attempt to learn about the state of
the intergalactic medium of small groups when the universe was only
$\sim 2/3$ its current age.  It consists of a single
chapter~(Chapter~\ref{ch:groups}\footnote{Chapter~\ref{ch:groups} is
nearly identical to the text of \citet{spiegel_et_al2007a}.}) in which
I present results and analysis of a sensitive X-ray observation of
galaxy groups at intermediate redshift.  The \textsl{CNOC2} Field
Galaxy Redshift Survey \citep{carlberg2001eg,carlberg2001gg}, a deep
spectroscopic survey of several fields on the sky, identified a number
of group-candidates -- collections of three or more galaxies that meet
a specified set of requirements that are intended to minimize the
number of chance-alignments.  The statistical weak-lensing signal from
these putative groups \citep{hoekstra_et_al2003CNOC2} provides
compelling complementary evidence that they are indeed gravitationally
bound objects, not physically unconnected galaxies that just happen to
lie near one another in redshift space.  Of the 21 of these
\textsl{CNOC2} groups in the field of view of the \textsl{EPIC-PN}
camera on XMM-{\it Newton}, not one was visible in over 100~ksec of
observation, even though a simple estimate implies that three of the
groups have velocity dispersions high enough that they would easily be
visible if their luminosities scaled with their velocity dispersions
in the same way as nearby groups' luminosities scale.

Note, however, that the statistical power of the survey is
insufficient to draw definitive conclusions.  Although
\citet{fang_et_al2007} similarly find evidence that groups at
intermediate redshift are less X-ray luminous than dynamically
comparable ones in the local universe, \citet{andreon_et_al2008} and
\citet{fassnacht_et_al2007} make some arguments suggesting otherwise.
Therefore, the question of how the cosmic baryon inventory has evolved from
redshift $\sim 0.5$ to the present remains open.



\section[Part II: ~Extrasolar Planets]{Part II: ~Extrasolar Planets}
\label{intro_sec:exoplanets}
\subsection[Background and Motivation]{Background and Motivation}
\label{intro_ssec:planets background}
On February 17, 1600, 9 years before Galileo built his first
telescope, Giordano Bruno was burned at the stake for having proposed
that there are many worlds, including ones more distant than our Sun.
Although observing technology remained insufficient to detect worlds
around other stars until the last half of the 20$^{\rm th}$ century,
astronomers never lost the desire to engage in observations and
speculation regarding very distant worlds.  Starting in the middle of
the 19$^{\rm th}$ century, when \citet{jacob1855} ascribed what he
said was an anomalous motion in the binary system 70 Ophiuchi to the
presence of an unseen planet, and continuing for more than 100 years,
more than a dozen observers claimed to have detected astrometric
wobbling of nearby stars that was caused by planetary companions.
However, such early announcements of planets found beyond our Solar
System are now generally considered to have been mistaken
\citep{batten_et_al1984,heintz1988}.  Although the claims based on
astrometry were erroneous, Otto Struve prophetically proposed that
high precision radial velocity (RV) analysis of nearby stars might
reveal massive planets in close orbits \citep{struve1952}.  When
\citet{campbell_et_al1988} found spectroscopic evidence of what is now
considered to be the first extrasolar planet ever discovered, Bruno's
fateful belief was finally confirmed.

In the last 20 years, progress in extrasolar planet detection has
accelerated rapidly.  \citet{wolszczan+frail1992} were the first to
obtain essentially indisputed evidence of a planet around another star
(although the star was a pulsar).  Several years later,
\citet{mayor+queloz1995} and \citet{marcy+butler1996} found the first
widely accepted evidence of planets around solar--type
stars.\footnote{The planet of \citet{campbell_et_al1988} was not
confirmed until further analysis \citep{hatzes_et_al2003}.}

\citet{huang1959} had considered and rejected the possibility of
performing systematic photometric searches for extrasolar planets
transiting along the lines-of-sight from Earth to their parent stars.
Even though contemporary observers could achieve the $\sim$1\%
photometric accuracy that would be needed to detect a transiting
Jupiter around a nearby Sun-like star, the probability that any
particular planetary system would have the appropriate geometrical
configuration was considered to be very low.  For randomly oriented
orbital planes, the probability that a planet -- at some point during
its orbit -- will transit across the face of its star from Earth's
vantage is approximately $R_*/a$, where $R_*$ is the star's radius and
$a$ is the orbital semi-major axis.  If other solar systems were built
like our own, with Jupiter-like planets at Jupiter-like distances from
the star, the probability of seeing a transit for a given solar system
would be $\sim$0.1\%, and each transit would be seen for only about a
day per decade.

On the other hand, since many of the first-discovered planets were of
comparable mass to Jupiter and were in very close orbits
($\lsim$0.1~AU), it was clear that many extrasolar planetary systems
have dramatically different structures from ours.  By the late 1990s,
therefore, it seemed sensible to perform accurate photometric
observations of stars known to have planetary companions -- and of
other stars as well.  Within just a few years, \citet{henry_et_al2000}
found that one of the planets that had been discovered through RV
analysis -- HD209458b -- also transits.

The discovery of the first transiting planet sparked tremendous
excitement, and led to several dozen surveys dedicated to searching
for transits.  In a conference proceeding titled, ``Status and
Prospects of Planetary Transit Searches: Hot Jupiters
Galore'',\footnote{Close-orbiting gas-giant planets are often called
``hot Jupiters''.}
%
Keith Horne predicted that 23 transit surveys operating in the year
2002 would cumulatively detect 191 planets each month
(\citealt{horne2003}; though this was generally thought to be overly
optimistic even at the time).  Figure~\ref{intro_fig:num_transits}
summarizes his survey-by-survey estimates.  A naive extrapolation
(that, in fairness, Horne cautioned against) indicates that, if he had
been right, there would now be over ten-thousand transiting exoplanets
known.  Instead, there are fewer than 300 exoplanets known, fewer than
50 of which transit.  Detecting transits from ground-based telescopes,
it turns out, is much more difficult than was initially appreciated.

\begin{figure}[p]
\plotone{figures/chapter1_intro/horne_num_transits.eps}
\caption[Predicted number of transits to be seen in 2002 transit
surveys]{This figure is reproduced from Figure~2 of \citet{horne2003}.
That paper predicted that 23 transit surveys operating in the year
2002, using telescopes with diameters ranging from 3.6~cm to 4~m,
would cumulatively detect 191 planets each month.  Each open circle in
this figure indicates the diameter and predicted number of planets per
month for one survey telescope.}
\label{intro_fig:num_transits}
\end{figure}
\afterpage{\clearpage}

With a pair of space-based transit-detecting observatories set to
monitor $\sim$200,000 stars over the next five-to-seven years,
astronomers are again expecting that large numbers of new planets will
soon be discovered.  In its one year of operation, COROT has already
found one of the most exotic planets ever discovered (COROT-Exo-1b),
orbiting with a period of 1.5 days, with a radius reportedly as great
as 1.78 times that of Jupiter.\footnote{COROT--Exo--1b is reported at
http://www.esa.int/esaCP/SEMCKNU681F\_index\_0.html.}  The {\it
Kepler} mission is scheduled to launch in February of 2009, and has
been predicted to find anywhere from several hundred to over 10,000
planets, many of which might be smaller, terrestrial ones
\citep{borucki_et_al2007, borucki_et_al2003, basri_et_al2005}.

For most of the planets that have been discovered, all that is known
are their orbital elements; since the orientation of the orbital plane
is typically not known, the mass cannot be ascertained beyond a
$\sin[i]$ degeneracy.  For the transiting planets, this degeneracy is
broken, and both their mass and radius can be easily determined, which
places some constraints on their interior structures.  But we would
like to know much more about them.  We would like to learn about the
composition and dynamics of their atmospheres, the detailed structure
of their deep interiors, their chemical and dynamical evolution.
Ultimately, we would like to know if there are any living organisms on
them.

Many planetary properties that we will wish to constrain will be
difficult to measure.  In Part~\ref{part:planets}, I address how to
constrain three nontrivial properties of extrasolar planets: \pn{i}
the composition of hot Jupiters (\S~\ref{intro_ssec:micro};
Chapter~\ref{ch:micro}); \pn{ii} the rotation rate and atmospheric
dynamics of hot Jupiters (\S~\ref{intro_ssec:rot};
Chapter~\ref{ch:rot}); and \pn{iii} the habitability of terrestrial
planets (\S~\ref{intro_ssec:hab} and \S~\ref{intro_ssec:obl};
Chapters~\ref{ch:hab} and \ref{ch:obl}).

\subsection[The Composition of a Microlensed Planet?]{The Composition of a Microlensed Planet?}
\label{intro_ssec:micro}
%\noindent \pn{i}
In Chapter~\ref{ch:micro},\footnote{Chapter~\ref{ch:micro} is nearly
identical to the text of \citet{spiegel_et_al2005}.} I revisit the
possibility of detecting an extrasolar planet around a background star
as it crosses the fold caustic of a foreground binary lens.  During
such an event, the planet's flux can be magnified by a factor of $\sim
100$ or more.  The detectability of the planet depends strongly on the
orientation of its orbit relative to the caustic.  If the source star
is inside the inter--caustic region, detecting the caustic--crossing
planet is difficult against the magnified flux of its parent star.  In
the more favorable configuration, when the star is outside the
inter--caustic region when the planet crosses the caustic, a close--in
Jupiter--like planet around a Sun--like star at a distance of 8 kpc is
detectable in 8-minute integrations with a 10~m telescope at maximal
$S/N\sim15$ for phase angle $\phi\sim10\degr$ (assuming an edge-on
orbit).  Most exoplanets currently known are less than several hundred
parsecs from Earth, and the community has considered probing the
atmospheric composition of only these relatively nearby exoplanets
(e.g., the work of \citet{brown2001}, \citet{charbonneau_et_al2002},
\citet{grillmair_et_al2007}, etc.).  A micro{\it lensed} planet, in
contrast, is likely to be several kiloparsecs away or more.  In this
chapter, I consider whether, with favorable viewing conditions, it
might be possible to probe the composition of such very distant
planets.  In the example described above, I find that the presence of
methane, at its measured abundance in Jupiter, and/or water, sodium
and potassium, at the abundances expected in theoretical atmosphere
models of close--in Jupiters
\citep{sudarsky_et_al2000,sudarsky_et_al2003}, could be inferred from
a non--detection of the planet in strong broad absorption bands at
$0.6-1.4\mu$m caused by these compounds, accompanied by a $S/N\sim 10$
detection in adjacent bands.  I conclude that future generations of
large telescopes might be able to probe the composition of the
atmospheres of very distant extrasolar planets.

\subsection[A Transiting Planet's Rotation Rate?]{A Transiting Planet's Rotation Rate?}
\label{intro_ssec:rot}
In Chapter~\ref{ch:rot},\footnote{Chapter~\ref{ch:rot} is nearly
identical to the text of \citet{spiegel_et_al2007b}.} I investigate
the effect of planetary rotation on the transit spectrum of an
extrasolar giant planet, in order to address in detail the
observability of some spectral signals originally discussed by
\citet{brown2001}.  During ingress and egress, absorption features
arising from the planet's atmosphere are Doppler shifted by of order
the planet's rotational velocity ($\sim 1-3 {\rm ~km~s^{-1}}$)
relative to where they would be if the planet were not rotating.  I
focus in particular on the case of HD209458b, which ought to be at
least as good a target as any other known transiting planet.  For
HD209458b, this shift should give rise to a small net centroid shift
of $\sim 60 {\rm ~cm~s^{-1}}$ on the stellar absorption lines.  Using
a detailed model of the transmission spectrum due to a rotating star
transited by a rotating planet with an isothermal atmosphere, I
simulate the effect of the planet's rotation on the shape of the
spectral lines, and in particular on the magnitude of their width and
centroid shift.  I then use this simulation to determine the expected
signal--to--noise ratio for distinguishing a rotating from a
non--rotating planet, and assess how this S/N scales with various
parameters of HD209458b.  I find that with a 6~m telescope, an
equatorial rotational velocity of $\sim 2 {\rm ~km~s^{-1}}$ could be
detected with a S/N $\sim 5$ by accumulating the signal over many
transits over the course of several years.  With a 30~m telescope, the
time required to make such a detection reduces to less than 2 months.

\subsection[Climatic Habitability]{Climatic Habitability}
\label{intro_ssec:hab}
In Chapter~\ref{ch:hab},\footnote{Chapter~\ref{ch:hab} is nearly
  identical to the text of \citet{spiegel_et_al2007bb}.} I begin to
lay the groundwork for interpreting the climatic habitability of the
many terrestrial exoplanets that we hope will be found in the next
several years and beyond.  According to the standard liquid-water
definition, the Earth itself is only partially habitable.  I
reconsider planetary habitability in the framework of energy-balance
models (EBMs) -- the simplest seasonal models in physical climatology
-- to assess the spatial and temporal habitability of Earth-like
planets. In order to quantify the degree of climatic habitability of
these models, I define several metrics of fractional habitability.
Previous evaluations of habitable zones may have omitted important
climatic conditions by focusing on close Solar System analogies. For
example, I find that model pseudo--Earths with different rotation
rates or different land--ocean fractions generally have fractional
habitabilities that differ significantly from that of the Earth
itself.  Furthermore, the stability of a planet's climate against
albedo-feedback snowball events strongly impacts its
habitability. Therefore, issues of climate dynamics may be central in
assessing the habitability of discovered terrestrial exoplanets,
especially if astronomical forcing conditions generally differ from
the moderate Solar System cases.

\subsection[Habitability and Planetary Obliquity]{Habitability and Planetary Obliquity}
\label{intro_ssec:obl}
In Chapter~\ref{ch:obl}, I expand upon the investigations begun in
Chapter~\ref{ch:hab}, adding several new ingredients to the EBM.
Without the stabilizing influence of the Moon, the Earth's obliquity
might vary significantly.  Extrasolar terrestrial planets with the
potential to host life may therefore also be subject to large
obliquity variations.  With this in mind, I revisit the habitability
of oblique planets.  I further enrich the EBM by allowing variations
of land/ocean distribution, and I incorporate an infrared cooling
function that explicitly accounts for the large increase in $\rm CO_2$
concentration expected at orbital distances greater than 1~AU from a
$1~L_\sun$ star.  The several new tunable knobs in the EBM together
produce a vast new parameter space, one that must be explored in order
to understand what climatic trends are likely on less-Earth-like
planets.  I again pay particular attention to dynamical transitions to
ice-covered snowball states that result from ice-albedo feedback.  Of
course, if one is to trust that the climate model might produce
reliable results for conditions that differ from those on Earth, one
would hope that it at least predicts Earth-like climates for
Earth-like conditions.\footnote{This caveat applies equally to
Chapter~\ref{ch:hab}.} And indeed, despite the great simplicity of the
EBM, it captures rather well the seasonal cycle of global energetic
fluxes at Earth's surface.  It also performs satisfactorily against a
full-physics climate model of a highly oblique Earth, in an unusual
regime of circulation characterized by heat transport from the poles
to the equator.  Having verified that the model produces reasonable
results for the conditions for which I can cross-check it, I then use
it to find several important results.  First, I find that oblique
terrestrial planets can violate global radiative balance throughout
much of their seasonal cycles if they have asymmetrical distributions
of continents and ocean.  This is likely to limit the usefulness of
simple radiative equilibrium arguments.  Second, while the climates of
high obliquity planets can be severe, with large seasonal variations,
I find (as did \citet{williams+kasting1997}) that they are not
necessarily more prone to snowball transitions than low obliquity
planets.  In fact, the models indicate that high obliquity planets may
be {\it less} prone to global snowball events, which raises the
interesting possibility that moon-less worlds might in general have a
higher probability of maintaining a temperate climate.  Finally, I
find that, depending upon the efficiency of latitudinal heat
transport, terrestrial planets with massive $\rm CO_2$ atmospheres,
typically found in the outer regions of habitable zones, also can be
subject to dynamical snowball transitions.

\subsection[Conclusions: Looking Forward]{Conclusions: Looking Forward}
\label{intro_ssec:conc}
In Chapter~\ref{ch:conc}, I extrapolate beyond the studies described
in the first \ref{ch:obl} chapters, in anticipation of several
challenging investigations that may prove to be important and
interesting in coming years.  Specifically:
\begin{itemize}
\item Various authors have been suggested that a planet may be much
more likely to be hospitable to life if it possesses a magnetic field;
but can an extrasolar planet's magnetic field be measured?  I consider
the possibility that, with a space-based array of radio antennas, it
will be possible to detect cyclotron maser radiation from a nearby
terrestrial planet.
\item Through comparing observed infrared light curves and spectra of
hot Jupiters to detailed models of internal structure and composition,
various investigators are starting to learn about the formation,
evolutionary history, and current structure of these objects.  In the
coming decades, similar work will improve our understanding of
terrestrial planets.  I discuss how an enhancement to the sort of
simple energy balance models considered here may not only allow us to
predict which terrestrial planets may be habitable, but also may be of
use in designing, and interpreting the results of, future missions to
obtain spectra of terrestrial planets, such as {\it Terrestrial Planet
Finder} missions.
\end{itemize}


\cleardoublepage
