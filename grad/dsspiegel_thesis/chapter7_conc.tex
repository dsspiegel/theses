\chapter[Conclusion: Looking Forward]{Conclusion: Looking Forward}
\label{ch:conc}
\markright{}


This thesis addresses several properties of extrasolar planets that
are beyond present observational capacities.  It is interesting to
think about what attributes will be considered cutting edge or
nontrivial in the future, especially since, although nearly 300
planets are now known around other stars, just two dozen years ago
none were.  At that time, the mere {\it existence} of exoplanets was
an extremely nontrivial property.  The next several decades will see
many more, increasingly sophisticated, discoveries.  In concluding, I
will speculate about a few of these.


\section{Magnetic Planets and Moons?}
\label{conc_sec:mag}

As more and more terrestrial planets are discovered, improving our
understanding of what planetary attributes are conducive to
habitability will be an ever more important goal.  Climatic
habitability will surely continue to draw attention, and deservedly
so, because of the clear influence on climate (temperature) of orbital
distance, one of the main observable properties of exoplanets.  But
what about other aspects of habitability?  The high energy particles
in the solar wind might prevent the formation of complex biological
molecules if it were not for the Earth's magnetic field, which acts as
a shield.  Indeed, \citet{dehant_et_al2007} and others have argued
that a planet might require a magnetic field in order to develop and
maintain life as we know it.  But how to measure the magnetic field of
an extrasolar planet?

The gas giant planets of our Solar System all exhibit strong, low
frequency (kilometric to decametric) radio emission \citep{zarka1992},
which has prompted researchers to search for radio emission from
extrasolar giant planets.  Although \citet{lazio+farrell2007} reported
that observations of the $\tau$ Bootis system with the Very Large
Array yielded no detection, \citet{farrell_et_al2004} and
\citet{griebmeier_et_al2005} have suggested that finding evidence of a
magnetic field around an exo-Jupiter might be within the reach of the
Low Frequency Array ({\it LOFAR}).  Such a detection would be exciting
but would bear no direct relevance to astrobiology, at least not to
the habitability of terrestrial planets.

More germane from an astrobiological perspective: it might not be
beyond the capability of future astronomers to detect radio signals
from even an Earth-like planet.  A simple analysis indicates that,
with the right instrument, radio wavelength radiation may be among the
easiest types of radiation to detect from an extrasolar
planet.\footnote{The right instrument -- a space-based radio array, as
  described below -- is stunningly simple from a physics perspective,
  though building it might pose significant technical and engineering
  challenges.}  It is in general very difficult to detect light from
an extrasolar planet: a solar--type star is brighter than a
hot--Jupiter by a factor of at least $\sim 10^4$ at optical
wavelengths.  The ratio of starlight to planet--light is somewhat
smaller (a few times $10^3$) at infrared wavelengths, but still only a
small fraction of the light that we receive is from the planet.
Although emergent planetary radiation has been detected in several
transiting systems, the technical challenge of detecting emitted or
reflected radiation from even a hot--Jupiter remains great.  Optical
and infrared radiation from an Earth-clone orbiting a Sun-like star at
1~AU would be $\sim 10^4 - 10^5$ times dimmer than that from a hot
Jupiter.  In parts of the radio spectrum, however, the star--planet
contrast is less daunting.  More than three decades ago,
\citet{gurnett1975} discovered that the Earth is a very strong
radio--emitter at long wavelengths.  There are several mechanisms
responsible for the Earth's kilometric radiation, the strongest of
which is the cyclotron maser interaction of the solar wind with the
magnetosphere at the poles, the same mechanism that is responsible for
optical polar aurorae.  Strong radio bursts are observed to be
associated with bright auroral arcs, and so these flares are called
Auroral Kilometric Radiation (AKR).  During particularly strong
flaring, the Earth shines at wavelengths from 1 to 30 km with
integrated power of several billion watts ($10^{16}
{\rm~erg~s^{-1}}$), which makes Earth brighter than the combined
output of the Sun and Jupiter by a factor of 10 or more.  The bulk of
this radiation originates at a height above the Earth's surface of
between $0.75 R_\earth$ and $5 R_\earth$, with some emission detected
out to $40 R_\earth$ \citep{mutel_et_al2004}.

Because AKR is well below the plasma frequency of the ionosphere
($\sim 10 {\rm~MHz}$), it does not penetrate to the ground, so an
instrument to detect extraterrestrial AKR would need to be in orbit,
above the ionosphere.  An Earth--like planet around a Sun--like star
could be expected to have similar emission properties to Earth's, if
it had a similar magnetic field.  Searching for such emission poses
several difficulties that may turn out to be prohibitive, but the
promise of such a favorable contrast ratio, together with the
important complementary information that a detection of AKR from an
exo-Earth would give us, indicate that a study of the feasibility of
such an observation warrants further
investigation. \citet{janhunen_et_al2003} have proposed a space-based
interferometric array for exactly this purpose.  The good news is that
the detectors would not need to be anything terribly sophisticated.
Simply an array of kilometer-long wire antennas in space might allow
synthesis-imaging of an exo-Earth.  But even with an ideal radio
array, could we hope to see Earth at 10~pc?  The answer is unclear.
\citet{novaco+brown1978} provide a rudimentary characterization of
nonthermal Galactic emission at very long wavelengths, but not much is
actually known about the strength and spatial structure of this
radiation.  An immediate concern is that scattering by the
interstellar medium might severely degrade the quality of an image,
similar to how the Earth's atmosphere reduces the quality of seeing at
optical wavelengths.  In fact, \citet{cohen+cronyn1974} and others
characterize the wavelength dependence of interstellar scattering, and
find that the apparent diameter of a source scales with square of the
wavelength, at long wavelengths.  With $\lambda\sim 1$~km, the formula
of \citet{cohen+cronyn1974} predicts that a point source would be
blurred to a $\sim 1\degr$ disc, hopelessly spread out.  Still,
existing characterizations of interstellar scattering tend to be based
on assuming a very-distant source.  Within our local solar
neighborhood, the scattering might be significantly less intense, and
it might still be reasonable to hope to see extraterrestrial AKR from
nearby Earth-like planets.

In the near future, it will probably be impossible to detect an
Earth-like magnetic field around a terrestrial planet, but it is
conceivable that detecting the magnetic field of an Earth-like moon of
a giant planet will be possible with instruments that will come online
in the near future.  \citet{williams_et_al1997} and \citet{scharf2006}
have proposed that moons of giant planets might be hospitable
locations for life, both in our own Solar System and in others.  An
icy moon far away from the star, such as Europa, might be able to
protect sub-ice life from high energy stellar radiation without the
help of a magnetic field; on the other hand, life beneath kilometers
of ice in another solar system will probably be practically impossible
to detect from Earth.  But a moon in the star's habitable zone would
presumably require a magnetic field to shelter its atmosphere and its
inhabitants from its stellar wind.  If a moon orbiting a magnetized
giant planet possesses its own magnetic field, the interaction of the
moon's and the planet's magnetic fields must influence the planetary
decametric radiation in a similar way to how Io's magnetic field
influences Jupiter's radio emission: it enhances power at the
high-frequency ($\sim20$~MHz) end of the spectrum \citep{zarka1992}.
Observations with {\it LOFAR}, and in the future with the Square
Kilometer Array, may allow astronomers to detect the radio emission
from a magnetized moon orbiting a magnetized exo-Jupiter, if the
radiometric signature of the moon would be a discernible variant of
the emission from a planet that lacks a strongly magnetized moon.



\section{Predicting Spectra with EBMs}
\label{conc_sec:earth_spec}

The study of hot Jupiter planets is rapidly becoming a data-driven
field.  As {\it Spitzer} obtains more phase curves and secondary
eclipse measurements, studies such as those of
\citet{burrows_et_al2007b} are becoming crucial to the interpretation
of the emergent radiation of these planets.  We are still several
years away from having similar photometric or spectroscopic data for
Earth-like extrasolar planets, but it is not too early to begin to
think about setting up a framework for similarly deriving information
about planetary structure from those future data.

The type of simple energy balance models discussed in
Chapters~\ref{ch:hab} and \ref{ch:obl} might prove to be an invaluable
resource in interpreting future observational studies of terrestrial
planets.  The 1D (latitude only) nature of the EBM discussed earlier
in this thesis is reasonable only to the extent that a planet may be
said to have temperature that varies primarily with latitude, which
for the most part is a justifiable assumption only for planets that
spin much faster than they orbit.  This assumption is therefore likely
to fail for planets in the habitable zones of lower mass stars, which
may be locked into synchronous or near-synchronous rotation; as a
result, the 1D EBM I have already developed will not be appropriate
for modeling them.  It might be the case that the majority of
habitable planets are orbiting stars significantly less massive than
the Sun; so it is imperative to develop realistic climate models of
these planets.  \citet{joshi_et_al1997} and \citet{joshi2003} have run
impressively detailed 3D global circulation models (GCMs) of tidally
locked planets around M-dwarfs.  As a critical complement to the work
of \citet{joshi2003}, I hope to modify my current 1D model to include
a second spatial dimension.  Although neglecting explicit calculations
of vertical structure and of motions in the vertical direction will be
a limitation of this approach, the advantages are twofold: I will both
be able to \pn{i} run the model at higher spatial resolution, and
\pn{ii} still run it efficiently enough to test the influence of both
observable and unobservable planetary properties on regional
habitability.  \citet{charbonneau+deming2007} have proposed monitoring
all 10,000 M-dwarfs within 35~pc of Earth for transits.  As this
monitoring campaign proceeds and more such planets are discovered, the
2D EBMs that I will develop will be an essential tool for determining
the likelihood that regions of these planets will be habitable.

Furthermore, there is an entire regime of habitability that has so far
been largely ignored, between solar mass stars and the lowest mass
stars.  Around a K-dwarf, it is probable that planets will have
habitable temperatures at distances at which they are not quite
synchronous rotators, but instead might be rotating slowly, similar to
Mercury.  A database of 2D climate models, structured similarly to the
investigations carried out so far in this thesis, could provide an
important resource when considering possible climate structures on the
terrestrial planets that will be discovered orbiting K-dwarf stars.

But how will a 1D or 2D EBM help with interpretative studies?  In
order to do so, it will be important to produce models that not only
predict scarcely observable features such as habitability, but also
predict features that are more easily observable (e.g., albedo,
infrared luminosity, and strong spectral features).  Tying the output
of energy balance models to data that might become available with new
observatories in the next decade will allow these models to be helpful
tools not only for choosing targets of biosignatures-measuring
telescopes of the distant future, but also for doing useful science in
the more short-term future.

\cleardoublepage
