\chapter[A Possible Dearth of Hot Gas in Galaxy Groups at Intermediate Redshift]{A Possible Dearth of Hot Gas in Galaxy Groups at Intermediate Redshift}
\label{ch:groups}
\markright{}



\section{Introduction}
\label{groups_sec:intro}
According to the paradigm of hierarchical structure formation, as
matter falls from low--density environments to high--density
environments (i.e., clusters of galaxies) it passes through stages of
intermediate density, namely, groups.  Rich clusters of galaxies,
containing as many as a thousand member--galaxies or more, are
visually quite prominent and have therefore attracted research
interest for over 70 years.  Over the last several decades, however,
it has become increasingly apparent that small groups of galaxies,
containing fewer than 50 members and often containing as few as 3--10,
constitute by far the most common environment in which galaxies are
found in the universe today, and are therefore the dominant stage of
structure--evolution at the present epoch.  In order to understand the
structure and evolution of matter in the universe, then, we must
understand the properties of groups of galaxies.

In rich clusters, a large fraction of the baryonic mass ($\gtrsim
80\%$) exists as diffuse, hot, intracluster gas -- see
\citet{ettori_et_al2003} and the review by \citet{rosati_et_al2002}
and references therein -- that has been detected with space--based
X--ray telescopes for over 30 years \citep{giacconi_et_al1974,
rowanrobinson+fabian1975, schwartz1978}.  If small galaxy groups have
a similar hot--gas component, then a large fraction of the baryonic
mass of the universe could be hiding in these groups \citep[and
references therein]{fukugita+peebles2004}.  Alternatively, if groups
do not contain a diffuse, extended intragroup medium (IGM), this would
be troubling news for the idea that structure forms in the
hierarchical fashion that is widely assumed.

In the last 15 years, the improved resolution and sensitivity of new
X--ray telescopes have allowed us to begin to study the X--ray
properties of groups of galaxies.  \textsl{ROSAT} observations
indicate that in the local universe, half to three quarters of groups
have a hot, X--ray emitting IGM \citep{mulchaey2000}.
\citet*[hereafter PT04]{plionis+tovmassian2004} examine the
relationship between the X--ray luminosity ($L_X$) and the velocity
dispersion ($\sigma_v$) of nearby groups in the \citet[hereafter
M03]{mulchaey_et_al2003} catalog (although there remain significant
uncertainties associated with the analysis).  At higher redshift,
however, a systematic study of this kind has not yet been performed,
although some recent evidence suggests that the relationship at $z
\gsim 0.5$ is different from the local one: a deep \textsl{Chandra}
observation failed to reveal X--ray emission from groups discovered by
the \textsl{DEEP2} Galaxy Redshift Survey, which
\citet{fang_et_al2007} find strongly suggests that the groups in their
survey (at $z \sim 1$) are less X--ray luminous than the nearby
relationship would predict.

A recent deep imaging and spectroscopic study by the Canadian Network
for Observational Cosmology Field Galaxy Redshift Survey
(\textsl{CNOC2}) made it possible to identify a large number of galaxy
groups at intermediate redshift ($0.1 < z < 0.6$)
\citep{carlberg2001eg, carlberg2001gg}.  Gravitational lensing
analysis of the \textsl{CNOC2} fields yields statistical lensing
masses of these groups \citep{hoekstra_et_al2001}.  This data set,
therefore, provides an ideal laboratory for studying the X--ray
properties of groups at these redshifts.

Fortunately, the data archive for \textsl{XMM--Newton} contains
approximately 110 ksec worth of observations mostly overlapping one of
the \textsl{CNOC2} fields (21 of the groups are in the field of view
of the \textsl{EPIC--PN} camera; see Table~\ref{groups_ta:21Groups}).
If the relation between $L_X$ and $\sigma_v$ of groups at these
redshifts is the same as the relation that holds for nearby groups,
then we will argue that a few of the most massive groups ought to have
been visible in the aggregate data from these observations.  In this
analysis, we looked for whether the X--ray photons received are at all
correlated to the groups: \pn{i} Are there more (or fewer) photons
where there are groups than where there are not?  \pn{ii} Is the
spectral energy distribution of the photons where there are groups
different from that of the background?

In \S~\ref{groups_sec:obs}, we describe the data we used and how we
reduced it.  In \S~\ref{groups_sec:spatial} and
\S~\ref{groups_sec:spec}, we describe our analysis of the data --
spatial analysis and spectral analysis, respectively.  We found no
spatial or spectral evidence for the groups; and in
\S~\ref{groups_sec:disc}, we discuss what we had expected the data to
look like, and how surprising it is that we failed to detect the
groups.  In \S~\ref{groups_sec:conc}, we attempt to draw conclusions,
and we speculate as to the reasons why we did not detect the groups.
Finally, in Appendix~I, we give a brief analysis of several different
methods of fitting lines to data; and in Appendix~II, we describe how
we estimate the probability that groups at intermediate redshift share
the same X--ray luminosity function as those in the local universe.


\section{Observations and Data Reduction}
\label{groups_sec:obs}
The \textsl{CNOC2} survey comprised four fields on the sky.  We
examined the X--ray properties of the optically identified galaxy
groups in the 1447+09 field \citep{carlberg2001eg, carlberg2001gg}.
For our analysis, we used the \textsl{EPIC--PN} data from three
publicly available observations of that field in the
\textsl{XMM--Newton} data archive, of duration 33 ksec, 33 ksec, and
43 ksec.  We did not use the corresponding \textsl{EPIC--MOS} data in
the current analysis because of the lower sensitivity of the
\textsl{MOS} to low energy photons, and the attendant modest increase
in sensitivity in background--limited images: including \textsl{MOS}
would have increased the number of counts at the locations of the
groups in our analysis by a factor of only $\sim 50\%$.  The relevant
summary data on the 21 groups in the field of view of
\textsl{EPIC--PN}, from \citet{carlberg2001eg}, is presented in
Table~\ref{groups_ta:21Groups}.

For basic data reduction, including generating good--time--intervals
(GTI) files to deal with periods of high solar activity, removing
known hot--pixels, and generating images and exposure maps for energy
bands of interest, we used \textsl{XMM--Newton} Science Analysis
Software (SAS) release 6.1.0.  The light--curves used to generate GTI
files consisted of integrated flux in the range 0.3--15.0~keV across
the whole detector.  Time intervals of particularly high background
({\tt RATE>9}) were flagged as periods of flaring and were removed
from the data.  For the purpose of the analysis described below, we
considered four energy bands: 0.3--0.8~keV, 0.8--1.5~keV,
1.5--4.5~keV, and 0.2--1.0~keV, which we will hereafter refer to as
bands A, B, C, and D, respectively.  To make images in each band, we
selected events in the GTI with {\tt PATTERN<=4} and {\tt FLAG==0}.
Images and corresponding exposure maps were made at $2''$ per pixel,
roughly twice the resolution of the intrinsic pixel size of
\textsl{EPIC--PN} ($4.1''$).

Since we were looking for faint emission (from extended sources) that
was expected to outshine the background by only a slim margin, it was
essential to remove point sources from the images.  In addition, it
was crucial to characterize the background carefully before further
analysis so as to maximize the accuracy of the measurements of both
the background flux level and the group luminosities.  Two SAS tasks
-- {\tt eboxdetect} and {\tt emldetect} -- were used to find and
remove point sources in all observations, in all energy bands;
locations of point sources were then masked in the images and exposure
maps.  In order to characterize the background, we used two methods
that are described in detail in the next section.  In short, one was
to smooth the point--source excluded image, and the other was to
sample the average count--rate over the groups--excluded image.

%\begin{table}[p]{rcccccc}
\begin{table}[p]
\begin{center}
%\tablewidth{240pt}
%\tablecolumns{7}
%\small
%\tablecaption{The 21 \textsl{CNOC2} Groups in the Field of View}
\caption[The 21 \textsl{CNOC2} Groups in the XMM Field of View.]{~~The
21 \textsl{CNOC2} Groups in the Field of View.}
\vspace{0.2in}
\begin{tabular}{rcccccc}
  \tableline
  \tableline
  Group ID & R.A. (J2000) & Decl. (J2000) & $z$\tablenotemark{a} & $N_z$\tablenotemark{b} & $\sigma_v$ ($\rm km~s^{-1}$)\tablenotemark{c} & Est. $M$ ($h^{-1} M_\sun$)\tablenotemark{d}\\[0.1in]
  \tableline
 1 & 14 49 42.453  & 09 02 44.76  & 0.165  & 3  & 164 $\pm$ 126 & $1.3 \times 10^{13}$ \\
 2 & 14 48 55.799  & 09 08 48.48  & 0.229  & 3  & 162 $\pm$ 139 & $7.2 \times 10^{12}$ \\
 3 & 14 48 57.854  & 08 57 05.58  & 0.262  & 4  & 229 $\pm$  77 & $1.2 \times 10^{13}$ \\
 4 & 14 49 28.406  & 08 51 54.07  & 0.270  & 3  & 104 $\pm$  92 & $3.0 \times 10^{12}$ \\
 5 & 14 49 13.637  & 08 50 10.34  & 0.271  & 4  & 112 $\pm$  66 & $3.0 \times 10^{12}$ \\
 6 & 14 49 44.220  & 08 57 37.84  & 0.273  & 3  & 165 $\pm$ 120 & $8.1 \times 10^{12}$ \\
 7 & 14 48 49.049  & 08 57 50.15  & 0.306  & 3  &  93 $\pm$  65 & $1.3 \times 10^{12}$ \\
 8 & 14 50 20.387  & 09 06 00.62  & 0.306  & 4  & 199 $\pm$ 161 & $1.6 \times 10^{13}$ \\
 9 & 14 50 13.703  & 08 57 37.28  & 0.325  & 4  & 175 $\pm$ 151 & $9.9 \times 10^{12}$ \\
10 & 14 49 03.618  & 09 07 02.86  & 0.359  & 4  &  82 $\pm$  66 & $1.6 \times 10^{12}$ \\
11 & 14 49 40.860  & 09 02 15.68  & 0.372  & 3  & 126 $\pm$  94 & $2.8 \times 10^{12}$ \\
12 & 14 50 22.850  & 09 01 14.74  & 0.373  & 4  &  44 $\pm$  41 & $2.1 \times 10^{11}$ \\
13 & 14 49 49.201  & 08 51 23.08  & 0.374  & 4  & 291 $\pm$ 193 & $3.4 \times 10^{13}$ \\
14 & 14 50 00.825  & 08 49 06.65  & 0.394  & 4  & 308 $\pm$ 257 & $3.6 \times 10^{13}$ \\
15 & 14 49 14.155  & 09 11 15.38  & 0.394  & 3  & 394 $\pm$ 406 & $4.8 \times 10^{13}$ \\
16 & 14 49 55.433  & 08 56 11.39  & 0.394  & 3  & 507 $\pm$ 469 & $7.8 \times 10^{13}$ \\
17 & 14 49 32.746  & 09 03 41.28  & 0.468  & 4  & 488 $\pm$ 417 & $7.5 \times 10^{13}$ \\
18 & 14 49 29.974  & 09 09 08.15  & 0.469  & 4  & 217 $\pm$ 224 & $8.9 \times 10^{12}$ \\
19 & 14 49 31.428  & 09 05 05.22  & 0.472  & 3  & 123 $\pm$  99 & $4.9 \times 10^{12}$ \\
20 & 14 49 23.516  & 08 58 47.83  & 0.511  & 3  & 565 $\pm$ 668 & $1.2 \times 10^{14}$ \\
21 & 14 49 23.763  & 08 55 17.84  & 0.543  & 3  & 151 $\pm$ 132 & $8.7 \times 10^{12}$ \\
\label{groups_ta:21Groups}
\end{tabular}
\vspace{-0.4cm}
\tablenotetext{a}{ID Numbers are assigned to the groups that were used
for the analysis in this paper.  Groups are listed in order of
increasing redshift ($z$).}
\tablenotetext{b}{$N_z$ is the number of galaxies with identified
redshifts in each group.}
\tablenotetext{c}{$\sigma_v$ is line-of-sight velocity dispersion of
group galaxies.}
\tablenotetext{d}{Estimated mass, and all other numbers in this table
(except ID Number), are taken from \citet{carlberg2001eg}.}
\end{center}
\end{table}


\section{Spatial Analysis}
\label{groups_sec:spatial}
We investigated whether the spatial locations of received photons were
at all correlated with the groups.  This inquiry required a metric of
surface brightness that would allow us to compare quantitatively
different patches of sky.  The idea behind the desired metric is to
find the average number of counts per unit sky area per unit time
within the patch.  A reasonable first guess of the metric would
therefore be the sum of the pixel values in the patch, divided by the
number of pixels, divided by the total exposure time.  In an
experiment in which the pixels have nonuniform effective exposure
time, however, they have unequal sensitivity.  Maximizing the
sensitivity of such an experiment requires weighting more heavily
those pixels that are more sensitive in the observation.  A reasonable
second guess of the metric of the surface brightness of a patch of $N$
pixels, then, is
\[
\R \equiv k \times \frac{\sum_{i=1}^N(p_i/e_i)w_i}{\sum_{i=1}^Nw_i},
\]
where $k$ is the number of square arcseconds per pixel (in our images,
$k = 4 {\rm~arcsec^2~pixel^{-1}}$), $p_i$ is the number of counts in
the $i$th pixel of the patch, $e_i$ is the effective exposure time
seen by that pixel, and $w_i$ is an arbitrary weighting factor.  In
order to maximize the signal--to--noise ratio of $\R$, the appropriate
weighting factor is $w_i = e_i$, so in the end the metric reduces to
\begin{equation}
\label{groups_eq:metric}
\R \equiv k \times \frac{\sum_{i=1}^Np_i}{\sum_{i=1}^Ne_i}.
\end{equation}

In order to determine the average surface brightness of the groups, we
had to know which patches of sky to define as the groups.  For each
group, we defined a circular aperture within which to search for
X--ray emission.  The best way to assign radii to groups is open to
debate, so we tried various ways, including two methods that depend on
the groups' properties -- $r_{500}$ and $r_{200}$, calculated with
formulae from the literature -- and several that do not.  The
dependence of $r_{500}$ \citep{osmond+ponman2004} and $r_{200}$
\citep{mahdavi+geller2004} on redshift $z$ and line--of--sight
velocity dispersion $\sigma_v$ are given by
\begin{eqnarray}
\label{groups_eq:r500}
r_{500} & = & 130 {\rm~kpc} \left( \frac{\sigma_v}{100 {\rm~km~s^{-1}}} \right) \left( \frac{H(z)}{73 {\rm~km~s^{-1}~Mpc^{-1}}} \right)^{-1} \\
\label{groups_eq:r200}
r_{200} & = & 230 {\rm~kpc} \left( \frac{\sigma_v}{100 {\rm~km~s^{-1}}} \right) \left( \frac{H(z)}{73 {\rm~km~s^{-1}~Mpc^{-1}}} \right)^{-1},
\end{eqnarray}
where $H(z)$ is the value of the Hubble constant at redshift $z$.  All
calculations in this paper assume $H_0 = 73 {\rm~km~s^{-1}~Mpc^{-1}}$,
$\Omega_M = 0.3$, and $\Omega_\Lambda = 0.7$.  We also tried assigning
a fixed physical size to each group ($125 {\rm~kpc}$, $250 {\rm~kpc}$,
and $500 {\rm~kpc}$).  In each case, physical size was converted to
angular size using $\alpha = r/D_A(z)$, where $r$ is the group's
physical size and $D_A(z)$ is its angular diameter distance.  Happily,
our main scientific results were independent of how we calculated the
size of the aperture around each group -- and this is an important
point, on which we will elaborate below.  In this paper, we will
present results for $250 {\rm~kpc}$ apertures around each group.

The $\R$--value of a group (according to
equation~(\ref{groups_eq:metric}), within the set of pixels defined by
the group's angular radius $\alpha$) cannot by itself be converted to
the group's luminosity, because it includes counts from the background
that are unrelated to the groups.  The background flux level must
therefore be calculated and subtracted.  The true measure of the
average surface brightness of each group is the difference between the
group's $\R$--value and the $\R$--value of the background:
\begin{equation}
\label{groups_eq:brightness}
SB_g \equiv \R_g - \R_{BG}.
\end{equation}
We calculated the average surface brightness of the background in two
different ways, and we also calculated the local background
corresponding to each group, which gave us multiple measures of the
surface brightness $SB_g$ of each group.

We calculated the average background for the whole image in two ways.
First, for each energy band, we determined the $\R$--value of the
image for all pixels not contained in any group's aperture.  Second,
we created a background image for each energy band by smoothing that
band's image with a Gaussian smoothing radius of 60 pixels and, as
before, determined the $\R$--value of the background image for all
pixels not contained in any group's aperture.  These two methods of
determining the average background of the whole image yielded nearly
identical results.  According to \citet{scharf2002}, at redshift of
$\sim 0.5$, low mass objects such as groups should be seen most
readily in a low--energy band such as D or A.  Since D--band contains
more counts than A--band, much of our analysis deals with D--band
data.  The different measurements of the whole--detector background
were similar; in D--band, for example, the background rate was
measured to be $3.26$ or
$3.08\times 10^{-6}{\rm~cts~arcsec^{-2}~s^{-1}}$, depending on whether
the image or the smoothed (background) image was used.

We also measured the local background corresponding to each group by
taking an annulus of inner radius $1.05\alpha$ and outer radius
$1.20\alpha$ around each group, and finding the $\R$--value of the
background image in the annulus surrounding the group.  These local
background measurements ranged from $2.44$ to
$4.21\times 10^{-6}{\rm~cts~arcsec^{-2}~s^{-1}}$, and had a median of
$2.77\times 10^{-6}{\rm~cts~arcsec^{-2}~s^{-1}}$ and a mean of
$2.96\times 10^{-6}{\rm~cts~arcsec^{-2}~s^{-1}}$.

Finally, to understand the spatial variability of the background, in
each energy band we placed an aperture of radius $0.9\arcmin$ (27
pixels) in 20,000 random locations on the point--source--excluded
image, and measured the $\R$--value within each aperture.  This
procedure gives an empirical measure of the probability distribution
function that describes the surface--brightness in apertures of
approximately the sizes of the groups.  Results for D--band are
displayed in the histogram in Figure~\ref{groups_fig:histfig}.  For
D--band, the modal $\R$--value of the samples of the background is
$2.8\times10^{-6}{\rm~cts~arcsec^{-2}~s^{-1}}$; the median
$\R$--value, $3.0\times10^{-6}{\rm~cts~arcsec^{-2}~s^{-1}}$, is
similar; and the mean, $3.5\times10^{-6}{\rm~cts~arcsec^{-2}~s^{-1}}$,
is somewhat higher owing to a few anomalously high measurements that
resulted from apertures with small numbers of pixels near the edge of
the detector or near masked regions.

Surprisingly, the $\R$--values of the groups were frequently less than
those of the background.  This means both that $\R$--values of
individual groups were often less than the corresponding $\R$--values
of the local background, and that the stacked $\R$--value for all 21
groups, $2.78\times 10^{-6}{\rm~cts~arcsec^{-2}~s^{-1}}$, was $10\%$
{\it less} than the lesser of the two measures of the average
background.  In fact, the average value of the surface brightness of
groups (calculated with equation~(\ref{groups_eq:brightness}) and
using the local background $\R$--value for $\R_{BG}$; average was
weighted by the product of the number of pixels and the effective
exposure time) turned out to be negative: $-8.6\times
10^{-8}{\rm~cts~arcsec^{-2}~s^{-1}}$.

\begin{figure}[p]
\plotone{figures/chapter2_groups/f1.eps}
\caption[Histogram of samples of D--band background brightness.]{
Histogram of the result of 20,000 random samples of the spatial
variability of the D--band background.  The number of samples in each
$\R$--bin is plotted.  The bin--size is $2\times 10^{-7}$ counts per
square arcsecond per second.  The mode, median, and mean values in
this histogram are consistent with other measures of the background.
}
\label{groups_fig:histfig}
\end{figure}
\afterpage{\clearpage}

Ultimately, because of the spatial variability of the background,
which is evident in Figure~\ref{groups_fig:histfig}, the most useful
description of the background is probably the local one.  The top
panel of Figure~\ref{groups_fig:diffplot} shows the number of counts
by which each of the 21 groups in our sample is brighter than its
estimated local background, plotted against redshift.  The bottom
panel of that figure shows the surface brightness above the local
background $SB_g$, again plotted against redshift.  The 1--$\sigma$
error bars are computed on the ordinates as the summation in
quadrature of the $N^{1/2}$ Poisson noise from the group ($\sigma_g$)
and from the background ($\sigma_{BG}$).  No group is seen more than
$5\sigma$ above the background, which is our minimum criterion for
detection.  The dashed line in each panel represents the average value
of the quantity on the ordinate of that panel, weighted by the product
of the number of pixels and the effective exposure time.

\begin{figure}[p]
\plotone{figures/chapter2_groups/f2.eps}
\caption[Groups intensities above background in D--band (0.2--1.0
keV).]{ Groups intensities above background in D--band (0.2--1.0 keV).
{\it Top Panel}: Counts in groups above estimated local background.
{\it Bottom Pannel}: Surface brightness of groups above background, in
units of $\R$--value -- i.e., counts per square arcsecond per second.
None of the groups is seen more than $5\sigma$ above the background,
which is our minimum criterion for detection.  The dotted line in each
panel is the zero line.  The dashed line in each panel represents the
weighted average value of the quantity on the ordinate of that panel.}
\label{groups_fig:diffplot}
\end{figure}
\afterpage{\clearpage}

In those cases where both the X--ray and the optical centers of low
luminosity groups are known, the correlation between the two shows
considerable variance \citep{mulchaey2000}.  Because of the
possibility that the X--ray centers might not coincide with the
optical centers of the groups in our survey, we paid particular
attention to the analysis with an aperture of radius 500 kpc around
each group, because an aperture this size should be large enough to
capture a large fraction of the X--rays even if there is an offset of
$\sim 0.5\arcmin$ between the X--ray center and the reported optical
center.  Inasmuch as no group had an $\R$--value $5\sigma$ or more
above the local background, the results were substantially the same
(no plot shown).  Moreover, a greater fraction of groups had negative
$SB$--values than when the $250 {\rm ~kpc}$ radius apertures were used
($15/21$ vs. $11/21$).

We finally note that the groups might potentially contaminate the
local background estimator.  Using reasonable models for the intensity
and the spatial distribution of emission from groups, described in
detail in \S~\ref{groups_subsec:limits} below, we estimate that $\ge 30\%$
of flux comes out within our aperture, and $\le 10\%$ of flux comes
out in the annulus from 1.05 to 1.20 times the aperture radius.  This
portion of group flux that falls within the annulus that we use to
define the local background ought to increase the local background
measurements, and therefore to depress the surface brightness
measurements.  The depression, however, is not severe: since the group
surface brightness profile is expected to peak toward the center, the
surface brightness is still expected to be positive.  Moreover, the
average $\R$--value of the local background samples was
$2.9\times10^{-6}{\rm~cts~arcsec^{-2}~s^{-1}}$, which is only slightly
more than the modal $\R$--value of the randomly placed apertures, and
slightly less than their median value.  It therefore does not appear
that the local background estimator was significantly increased by
flux from the groups.

As a final test of whether the estimate of the background was
contaminated by group--flux, we used an annulus farther from the
aperture: the annulus from 1.5 to 2.0 times the aperture radius.  This
change had no important effect on our results.


\section{Spectral Analysis}
\label{groups_sec:spec}
We also explored whether examining the shape of the X--ray spectrum
would enhance the contrast to the background in the search for groups.
In the simplest possible test, we investigated whether the spectral
energy distribution (SED) of photons received from the locations of
groups differed from that of photons from elsewhere.  The background
consists of the particle background, and the combined diffuse glows of
our galaxy, the Galactic halo, the local group, and extragalactic
point--sources and diffuse sources \citep{mccammon+sanders1990,
mccammon_et_al2002}.

Spectral results from our observations are presented in
Figure~\ref{groups_fig:spec}.  In our data, the spectrum of the groups
is indistinguishable from that of the background.  This can be seen in
several ways.  The top panel of Figure~\ref{groups_fig:spec} show,
with a bin--size of $10 {\rm~eV}$, the stacked spectrum of the groups
in blue, the spectrum of the background in green, and the
``rest--framed'' spectrum of the groups in red (that is, the energy of
each photon from the groups is multiplied by $(1+z_{\rm group})$ prior
to binning).  The groups and the background both show broad emission
features centered at $\sim 0.55 {\rm~keV}$ and $\sim 1.5 {\rm~keV}$
(top panel of Figure~\ref{groups_fig:spec}), that are most likely due
to O VII at redshift 0 and neutral Si on the detector, respectively.
The difference spectrum, however, is 0 to within the 1--$\sigma$ error
bars (bottom panel), indicating that the SED from regions of groups is
identical to that of other regions.  Furthermore, the rest--framed
groups spectrum shows no signs of any spectral features.

\begin{figure}[p]
\plotone{figures/chapter2_groups/f3.eps}
\caption[Spectra showing $\R$--values per 10 eV bin.]{Spectra showing
$\R$--values per 10 eV bin.  The rest--framed groups show no signs of
any spectral features {\it Top Panel}: Groups, background,
rest--framed groups spectra, 0.2--2 keV.  Spectrum for locations of
groups is in blue, for background is in green, and for rest--framed
groups is in red.  {\it Bottom Panel}: Difference spectrum, 0.2--2
keV.  1--$\sigma$ error bars are in green.}
\label{groups_fig:spec}
\end{figure}
\afterpage{\clearpage}

\section{Discussion}
\label{groups_sec:disc}
In recent years, research efforts have increasingly been devoted to
characterizing the extended X--ray emission from collapsed halos of a
variety of masses, from clusters to groups to galaxy--mass halos.  For
two examples (among many) of studies of gas in higher--mass systems --
clusters, \citet*{akahori+masai2005} investigate the $L_X$--$T_X$
scaling relation describing intracluster gas, and
\citet*{arnaud_et_al2005} investigate the $M$--$T$ scaling relation of
intracluster gas.

Our study is concerned with the properties of gas in lower--mass
systems.  \citet{xue+wu2000} attempt to bridge the gap between
clusters and groups in the local universe by performing an analysis of
the $L_X$--$T$, $L_X$--$\sigma_v$, and $\sigma_v$--$T$ relations of 66
groups and 274 clusters taken from the literature.  The
\textsl{XMM}--Large Scale Structure survey identified groups and
clusters out to redshift 0.6 and beyond, and \citet{willis_et_al2005}
examine the $L_X$--$T_X$ relation of the objects in their survey.
Interestingly, in contrast to the analysis in this paper, they detect
X--ray emission from low--mass objects out to $z=0.558$.
The groups that they found were X--ray selected, so the selection effect
surely contributed to their finding groups in their survey while
we did not in ours (although we note that this does not explain the physical
difference that is responsible for the discrepancy in X--ray luminosities).


\subsection{Fits to $L_X$--$\sigma_v$ Data}
\label{groups_subsec:fits}
In order to estimate how our data ought to have appeared, we need to
know what X--ray luminosity should be expected from a group with a
given mass or velocity dispersion.  The groups of the M03 catalog are
a convenient sample of nearby groups with known velocity dispersion
and X--ray luminosity.  Since any fit to the $L_X$--$\sigma_v$
relation in the M03 groups will show scatter, a fit is not a line or
curve but a 2-dimensional region in $L_X$ versus $\sigma_v$ space.

In fact, properly, one would want to construct a 2D probability
distribution function on $L_X$--$\sigma_v$ space based on the
distribution of the M03 sample.  This function would give the
probability density for a group drawn from the same population as the
reference sample to have a particular velocity dispersion and X--ray
luminosity. One could then properly examine the likelihood that the
\textsl{CNOC2} groups are drawn from the same population as the M03
groups.

In practice, it would be difficult to construct the 2-D distribution
function we require, because the M03 sample is not dense enough in
$L_X$--$\sigma_v$ space.  We therefore reverted to a conventional
parametric analysis, based on linear fits to $(\log \sigma_v, \log
L_X)$ data.  We will present several linear fits to the M03 data, and
we describe in detail our choice for which linear fit is most
appropriate.

PT04 examine the relationship between the X--ray luminosity ($L_X$)
and the velocity dispersion ($\sigma_v$) of galaxy groups in the M03
catalog.  They perform regression analysis on $\log L_X$ and $\log
\sigma_v$ and find that, as with any regression in which there is
significant scatter, which line is considered the best--fit depends on
which variable is considered to be independent and which dependent.
We repeated their analysis of the M03 groups and obtained best--fit
lines that are very similar (though not identical) to those that they
obtained.  In this paper, we present our own fits to the data.  If
velocity dispersion is taken to be the independent variable, the
best--fit regression line is
\begin{equation}
\label{groups_eq:M03D}
\log \left(L_X / 1 {\rm~erg~s^{-1}}\right) = 36.8 + 2.08 \log \left( \sigma_v / 1 {\rm~km~s^{-1}} \right).
\end{equation}
If, on the other hand, luminosity is taken to be the independent variable,
the best--fit regression line (what PT04 refer to as the ``inverse regression
line'') is
\begin{equation}
\label{groups_eq:M03I}
\log \left(L_X / 1 {\rm~erg~s^{-1}}\right) = 30.9 + 4.50 \log \left( \sigma_v / 1 {\rm~km~s^{-1}} \right).
\end{equation}
Data from M03 and best--fit lines are displayed in
Figure~\ref{groups_fig:maxlums}.

Which regression represents the ``true'' relation between the X--ray
luminosity and the velocity dispersion of groups of galaxies?  In
Appendix~I, we present a more detailed analysis of the relative merits
of various procedures for fitting lines to data.  The result of this
analysis is that neither direct nor inverse regression is ideal,
because surely the measurements of both $\sigma_v$ and $L_X$ were
subject to errors, and furthermore there is undoubtedly intrinsic
scatter in the $L_X$--$\sigma_v$ relation.  The better relation out of
these two is the one in which the variable taken to be independent has
lower relative uncertainty.  In our case, this is probably the inverse
regression, because a velocity dispersion determined from only a few
galaxies is likely to be quite uncertain.  Furthermore, the steeper
(inverse) relation comes closer to fitting the high--$\sigma_v$,
high--$L_X$ end of the distribution of groups, which is the end that
we are most interested in when looking for groups at $z\sim 0.5$.
Still, although the inverse regression is probably preferable to the
direct regression, since there is scatter in both variables, this
situation calls for least squares orthogonal distance fitting
(described in Appendix~I).  If we assume that the typical relative
errors on $\log(\sigma_v)$ and $\log(L_X)$ are the same (i.e., the
error is shared equally between the two variables), then the
distance--fit line,
\begin{equation}
\log \left(L_X / 1 {\rm~erg~s^{-1}}\right) = 31.5 + 4.25 \log \left( \sigma_v / 1 {\rm~km~s^{-1}} \right),
\label{groups_eq:distfit}
\end{equation}
is nearly as steep as the inverse regression line, as shown in
Figure~\ref{groups_fig:maxlums}.


\subsection{Limits on Group Luminosities}
\label{groups_subsec:limits}
Since none of the groups was visible at $5\sigma$ above the
background, we may place limits on their fluxes and therefore, since
we know their redshifts, on their luminosities.  The total number of
background counts $TC_{BG}$ at the location of a group is estimated
from the smoothed background image.  Assuming Poisson statistics, the
standard deviation on this number is $\sigma_{TC_{BG}} =
(TC_{BG})^{1/2}$.  The maximum number of group counts $TC_G$, then, is
limited by
\begin{equation}
TC_G < 5\sigma_{TC_{BG}} = 5 \sqrt{TC_{BG}},
\label{groups_eq:TCG}
\end{equation}
We estimate each group's temperature with the standard relation given in,
e.g., \citet{mulchaey2000}:
\begin{equation}
T = \frac{m_p}{\mu} \sigma_v^2 \times (k \beta),
\label{groups_eq:temp}
\end{equation}
where $m_p$ is the proton mass, $\mu$ is the mean molecular weight,
$k$ is Boltzmann's constant, and $\beta$, whose empirical value tends
to be around $2/3$, is the eponymous parameter of the
isothermal--$\beta$ model for density and may be considered to be
defined by this equation.  (For low-density objects, $\beta$ has been
measured to be closer to $1$, but for analytical simplicity we used
$\beta=2/3$ for most calculations.)  The $\beta$ density model for a
spherical isothermal plasma is
\begin{equation}
\rho(r) = \rho_0 \left(1 + \left(\frac{r}{r_c}\right)^2\right)^{-3\beta/2},
\label{groups_eq:beta}
\end{equation}
where $\rho_0$ is the central density, $r$ is the radial distance from
the sphere's center, and $r_c$ is the ``core radius,'' a distance
within which the density is nearly constant (approximately $\rho_0$).
Using the \textsl{Astrophysical Plasma Emission Code} (\textsl{APEC})
model in \textsl{XSPEC}, we solve for what X--ray luminosity, at a
given temperature, would lead to the number of group counts given by
the maximum possible value of $TC_G$ in (\ref{groups_eq:TCG}).

In Figure~\ref{groups_fig:maxlums}, the X's show the maximum
luminosity that each group could have had during our observations (if
the group were more luminous, it would have shown up at $>5\sigma$
above the background).  If a group is predicted to be less luminous
than the luminosity represented by its X, then it is no surprise that
we failed to see the group; conversely, if a group is predicted to be
more luminous than the luminosity represented by its X, then the group
is less luminous than predicted.  The dashed line shows the
$L_X$--$\sigma_v$ direct regression line, the dashed--dotted line
shows the inverse regression line, and the solid line shows the
distance--fit line.  All X's (all groups) lie above the direct
regression line, so \emph{if} that line represents the true
relationship between the variables, there is no surprise in seeing
none of the groups.  Three groups, however, lie below both the inverse
regression line and the distance--fit line; so if those lines
represent the true relationship between the variables, it might be
surprising that we failed to see three of the groups.  In the
remainder of this section, we will try to quantify how surprising it
is.

\begin{figure}[p]
\plotone{figures/chapter2_groups/f4.eps}
\caption[Linear Fits to M03 Group--Data And Empirical Upper Limits On
CNOC2 Group--Luminosities.]{Linear Fits to M03 Group--Data And
Empirical Upper Limits On CNOC2 Group--Luminosities: bolometric X--ray
luminosity vs. line--of--sight velocity dispersion.  The circles show
the M03 groups. The dashed line is the direct regression line, the
dashed--dotted line is the inverse regression line, and the solid line
(L.S.O.F.) is the least squares orthogonal fit line.  These three
lines are fits to the M03 data.  The X's (E.U.L.) display the
``empirical upper limits'' to groups' luminosities, i.e., the maximum
luminosities that the \textsl{CNOC2} groups could have had during our
observations without showing up at more than $5\sigma$ above the
background.  The uncertainty in velocity dispersion associated with
the X's is typically nearly $\sim100\%$, as shown in
Table~\ref{groups_ta:21Groups}.  Three of the X's lie beneath both the
dashed--dotted and the solid lines, indicating that these three groups
were predicted to be luminous enough to be detected, and yet were not.
}
\label{groups_fig:maxlums}
\end{figure}
\afterpage{\clearpage}

\subsection{Quantifying Surprise}
\label{groups_subsec:surprise}
The \textsl{CNOC2} catalog of small groups \citep{carlberg2001gg}
contains groups with as few as three galaxies in them.  Any
measurement of the velocity dispersion of such a group will
necessarily have large uncertainties.  In fact, for some groups, the
estimated uncertainty in velocity dispersion is actually greater than
the velocity dispersion itself.  Because the symbol $\sigma$ is
frequently used to denote both uncertainty and velocity dispersion, it
is inelegant to represent the uncertainty in velocity dispersion, but,
eschewing elegance, we shall use $\sigma_{\sigma_v}$ to denote
uncertainty on $\sigma_v$.

It is possible that our null result -- our failure to detect any of
the groups in our survey -- occurred because the velocity dispersions
of the three groups predicted to be observed were overestimated.  If
so, then perhaps their predicted luminosities based on, e.g.,
(\ref{groups_eq:distfit}), should have been low enough that the groups
should not have been seen after all.  Of course, if there were no bias
to the measurements of $\sigma_v$, it is just as likely that the
velocity dispersions of these three groups were {\it underestimated},
which would make their non--detection even more surprising.
Furthermore, in order for this sort of statistical error to explain
our null result, not only would the velocity dispersions of the three
groups that we think we ought to have detected need to have been
sufficiently overestimated, but no {\it other} groups could have have
had their velocity dispersions underestimated by too much.  In short,
the statistical errors would need to be in particular directions on
particular groups.  If there is no non--statistical reason for
measurements to be in error, then what is the likelihood that the
types of error needed to explain our null result actually occurred?

To answer this question, we simulated 10,000 mock--\textsl{CNOC2}
catalogs, each with its own set of mock--velocity dispersions
($\{\widehat{\sigma_v}(i)\}$) associated with the mock--groups.  If
the set of measured velocity dispersions from the actual
\textsl{CNOC2} catalog is $\{\sigma_v(i)\}$, then, for each
mock--catalog, the simulated velocity dispersion of the $i$th group
$\widehat{\sigma_v}(i)$ was drawn from from a gamma distribution with
mean $\sigma_v(i)$ and standard deviation $\sigma_{\sigma_v}(i)$.  The
gamma distribution has the attractive property that the mean and the
standard deviation may be independently specified.  Because the tail
of the gamma distribution extends to infinity, however, a small
fraction of mock--groups were initially given unrealistically high
velocity dispersions. Incidentally, this is preferable to the normal
distribution, which has another tail that extends below zero to
negative infinity -- unphysical values for velocity dispersion.  To
deal with the gamma distribution's problem of predicting a few
unrealistically high values, we, for each group $i$, eliminated from
the set of mock--catalogs all mock--groups with velocity dispersions
more than $4\sigma_{\sigma_v}(i)$ above the mean ($\sigma_v(i)$), and
replaced these values with new gamma--distributed variables.  We
repeated this procedure iteratively until there were no more
mock--groups with velocity dispersion more than $4\sigma_{\sigma_v}$
above the mean.  Because this procedure involves replacing the
high--end outliers with values closer to the mean, it had the effect
of slightly depressing the mean.  All values were therefore increased
by the difference between the new mean and $\sigma_v(i)$, thereby
raising the mean back to $\sigma_v(i)$, as it should be.

We also tried several other random distributions for assigning
mock--velocity dispersions to our mock groups.  We tried a gamma
distribution with no high--end cut--off; and we tried several
modifications of the normal distribution, all of which included a
low--end cut--off at $0$.  We found that those results were not
importantly different from results from the gamma distribution with
the high--end cut--off, which is the analytically simplest
distribution that produces realistic results.  Results are presented
only for the mock--catalogs generated with the high--end cut--off
gamma distribution.

Each mock--group was assigned the same aperture size as the
corresponding real group.  Its mock--intensity was spatially
distributed according to a spherical isothermal--$\beta$ model.  Since
emissivity is proportional to the square of density $\rho$, the
surface brightness or intensity $I$ is proportional to the integral
along the line--of--sight of $\rho^2$.  At a projected viewing
``impact parameter'' $b$ from the sphere's center,
\[
I(b) \propto \int_{r=b}^{r=r_h} \rho(r)^2 {\rm d} s,
\]
where $r_h$, the ``halo radius,'' is the outer boundary of the sphere
of plasma, and $s$ is the variable along the line--of--sight.
Substituting for $\rho$ with equation~(\ref{groups_eq:beta}) and
performing the integration results in the following expression for the
surface brightness:
\begin{equation}
\mathcal{B}(x,A,\beta) \propto (1+x^2)^{-3\beta} (A^2 - x^2)^{1/2} \spa { _2F_1}\left(\frac{1}{2},3\beta,\frac{3}{2},\frac{x^2-A^2}{1+x^2} \right),
\label{groups_eq:bright}
\end{equation}
where $x \equiv b/r_c$ and $A \equiv r_h/r_c$ are substitutions to
simplify the integration, and $_2F_1$ is the Gauss hypergeometric
function.  Three quantities go into the normalization of $\mathcal{B}$
for each mock--group: its mock--luminosity from
e.g. (\ref{groups_eq:distfit}); its luminosity distance $D_L(z)$; and
its core radius.  Finally, we assumed a foreground column density of
neutral hydrogen of $2\times10^{20}{\rm ~cm^{-2}}$
\citep{dicke+lockman1990}.

It was unclear what core radii our mock--groups should be assigned,
because the $\beta$-model core radii of groups and clusters vary a
great deal.  \citet*{ota+mitsuda2004} report that in a sample of 79
clusters, including 45 ``regular'' clusters and 34 ``irregular''
clusters, the mean of all 79 values of $r_c$ is $(163 \pm
202)~{h_{70}}^{-1}{\rm~kpc}$, midway between the mean $r_c$ of the
regular clusters ($(76 \pm 60)~{h_{70}}^{-1}{\rm~kpc}$) and the mean
$r_c$ of the irregular clusters ($(273 \pm
259)~{h_{70}}^{-1}{\rm~kpc}$).  Because there is no strong correlation
between $r_c$ and group--mass, we think it is reasonable to adopt a
uniform assumed value of $r_c$ for the mock--groups in our
mock--catalogs.  We ultimately decided to set each mock--group's core
radius to $125 {\rm~kpc}$, because this value lies in the middle of
the range of values reported by \citet*{ota+mitsuda2004}.  So long as
the value of $r_c$ that we adopted is physically plausible, the
precise value does not matter much, because the predicted number of
counts from a group, within an aperture of fixed size, is not very
sensitive to the assumed value of $r_c$, as long as the angular--size
of the aperture is larger than that of the core radius.

The SED of each mock--group (and hence the flux in a given energy
band) is calculated with \textsl{XSPEC}, using \textsl{APEC}, with its
mock--temperature calculated from equation~(\ref{groups_eq:temp}).
For $A$, we tried values between $2$ and $\infty$, and for $\beta$, we
tried values between $0.6$ and $1.0$.  The results presented below are
for $A=\infty$ and $\beta=2/3$, and for D--band (0.2--1.0~keV).

In brief, the results of this Monte Carlo indicate that the
probability appears to be small, but not vanishingly so, that
statistical fluctuations in the measured values of $\sigma_v$ explain
our null result.  Figure~\ref{groups_fig:histsim} shows a summary of
the mock--catalog results.  The median and the mode number of groups
seen per catalog are both $3$, and the mean number seen is $3.1$,
which is in good agreement with the prediction of
section~\ref{groups_subsec:limits}.  The finding of our observation --
that no groups were seen -- occurred in only $3.5\%$ of the
mock--catalogs.  If there were no systematic errors in the
measurements of velocity dispersions of the \textsl{CNOC2} groups,
then it is unlikely but not impossible that the only reason we had
expected some of the groups to be visible in the first place was
erroneously high measurements of $\sigma_v$.

\begin{figure}[p]
\plotone{figures/chapter2_groups/f5.eps}
\caption[Histogram of the results of the mock--catalog
analysis.]{Histogram of the results of the mock--catalog analysis.  Of
the 10,000 mock--catalogs, $96.5\%$ contained at least 1 mock--group
visible at the $5\sigma$ level.  In the actual \textsl{CNOC2} survey
no groups were visible at this level.}
\label{groups_fig:histsim}
\end{figure}
\afterpage{\clearpage}

\vspace{0.5 in}
\subsection{Redshifted Mulchaey Groups}
\label{groups_subsec:M03z}
Perhaps we have placed too much emphasis, in sections
\ref{groups_subsec:fits} and \ref{groups_subsec:limits}, on which
best--fit line actually fits the M03 data the best -- when the truth
is that there will be significant scatter to the data around any line
drawn through the cloud of points.  An alternative way to determine
whether it is surprising that no groups in our survey showed up at
$5\sigma$ above the background is to ask whether the \textsl{CNOC2}
groups and the M03 groups were drawn from the same population.  We
addressed this question by investigating how many of the groups in the
M03 atlas would have been visible at the $5\sigma$ level if they had
been in our observation.

The M03 groups are all in the local universe, approximately at
redshift 0.  If they had been in our survey, i.e., between redshifts
0.1 and 0.6, they would have both been dimmer and subtended a smaller
solid angle on the sky.  Furthermore, they would have been observed
with \textsl{XMM--Newton}, which, owing to its greater focal ratio,
has a higher particle background rate than \textsl{ROSAT} (the
instrument with which they were originally observed).

There were 109 groups in the M03 catalog, of which 61 were seen in
X--rays.  We simulated observations of each of these 61 M03 groups at
each of the 21 redshifts of the \textsl{CNOC2} groups, ranging from
$z_1 = 0.165$ to $z_{21} = 0.543$ (ID numbers taken from
Table~\ref{groups_ta:21Groups}).  As before, our criterion was that a
group must outshine the background by at least $5\sigma$ to qualify as
being detected in our simulated observations.  Each group was assumed
to be at a location with detector--averaged characteristics: the mean
exposure map value, the median background value (from
Figure~\ref{groups_fig:histsim}), the mean amount of smearing due to
the point--spread function, and the mean amount of lost usable
detector--area because of bright point sources.  Much like in the
mock--catalog calculations described in section
\ref{groups_subsec:surprise}, we calculated the count--rate of each
group with \textsl{XSPEC}, using \textsl{APEC}, with the group's
temperature calculated from equation~(\ref{groups_eq:temp}).  We
distributed emission according to the isothermal--$\beta$ profile
using equation~(\ref{groups_eq:bright}), with $\beta=2/3$, $A =
\infty$, and $r_c = 125 {\rm~kpc}$.  For our simulated observations,
we used an aperture of $250 {\rm~kpc}$, and looked in D--band.  For
background, we used $\R_{BG} = 3.0\times10^{-6}
{\rm~cts~arcsec^{-2}~s^{-1}}$.  Finally, as before, we assumed a
foreground column density of neutral hydrogen of $2\times10^{20}{\rm
~cm^{-2}}$.  Within our simulation, the number of M03 groups that
exceeded the $5\sigma$ detection criterion varied from 22 of the 61
total groups (or $p_1 = 36\%$) at the lowest redshift ($z_1$), to 2 of
61 ($p_{21} = 3\%$) at the highest redshift ($z_{21}$).

We identify two populations of the Mulchaey groups -- all 109, and the
61 that were seen in X--rays.  In Appendix~II, we present in detail
our estimate of the likelihood that the \textsl{CNOC2} groups and the
M03 groups were drawn from the same population.  The result is that
the probability of obtaining our null result on the assumption that
the \textsl{CNOC2} groups were drawn from the same population as the
whole M03 catalog is 14.7\%; and the analogous probability on the
assumption that the \textsl{CNOC2} groups were drawn from the same
population as the 61 X--ray--bright M03 groups is 2.8\%.  These
probabilities (especially the latter) are low enough to be
interesting, but are certainly not dispositive.


\section{Conclusion}
\label{groups_sec:conc}
The analysis in section \ref{groups_subsec:M03z} is mildly suggestive
that the groups of the \textsl{CNOC2} survey constitute a sample that
is on average less luminous than the sample of groups observed by
Mulchaey and collaborators in the 2003 atlas.  The interesting
comparison, however, is not between two particular samples, but rather
between the population of groups at intermediate redshift and the
population in the local universe.

A first caveat is that the two samples were selected in very different
ways, and there is no {\it a priori} reason to suspect that their
underlying populations have similar luminosity functions.
Furthermore, it is conceivable that for some reason the M03 groups
should be expected to be a more luminous sample of their population
than the \textsl{CNOC2} groups are of theirs.  Perhaps the M03 groups
are on average more massive (although the ranges and distributions of
velocity dispersions for the two samples are nearly identical).  Do
the M03 groups contain more galaxies than the \textsl{CNOC2} groups?
Certainly, Mulchaey reports more galaxies per group on average --
including as many as 63 galaxies in one group.  Of the \textsl{CNOC2}
groups in our field of view, none contained more than 4 galaxies with
identified redshifts.  But most or all of the difference here is
almost certainly due to the difficulty of measuring spectroscopic
redshifts of faint galaxies at large distances; we therefore see no
reason to think that the M03 groups are on average more massive than
the \textsl{CNOC2} groups.

An obvious criticism of the present work is that the \textsl{CNOC2}
groups might not be groups at all, but just chance superpositions of
galaxies.  While this would clearly explain the lack of X--ray
emission, we will argue that we find it unlikely to be the reason why
we detected no hot gas.

\citet{carlberg2001gg} identified as a group any object in the
\textsl{CNOC2} Survey that contains at least 3 galaxies in
sufficiently close redshift--space proximity.  This identification
procedure does leave open the possibility of confusion: if two
galaxies are separated by a large distance, but have appropriately
large peculiar velocities, they could have similar recessional
velocities (redshifts) despite not being part of a gravitationally
bound structure.  Or, even if a few galaxies are close to one another
in physical space, they will not be part of a bound system without a
dark matter halo that is massive enough to bind them.

In searching for groups and clusters, it is often assumed that where
there are associations of galaxies, there is dark matter to bind them,
but this assumption could be wrong.  With rich clusters, there are
multiple independent checks of this assumption: the temperature of
intergalactic gas (which can be detected through its X--ray emission
or through inverse Compton distortions to the cosmic microwave
background) is one indicator of the mass; the velocity--distribution
of galaxies is another indicator of the mass, and in particular the
distribution should be Maxwellian (or Gaussian) for a virialized
structure; and the degree of gravitational lensing of background
galaxies is a third indicator of the mass.

The first check has obviously failed in this case -- for these 21
groups, there is no apparent hot gas component.  The second check is
useless for an association of very few (3--6) galaxies -- nearly any
velocity distribution is consistent with an underlying Gaussian
distribution.  How about lensing?  For structures of $\lsim 10^{13.5}
M_\sun$, noise prevents detection of a lensing signal.  As a result,
in the \textsl{CNOC2} fields, it is only possible to determine a
statistical lensing mass for the groups by stacking them and
determining their combined lensing signal.  This analysis has been
performed by \citet{hoekstra_et_al2001, hoekstra_et_al2003CNOC2}:
their calculated estimate of the average velocity dispersion per group
based on the mass inferred from the average shearing of background
galaxies ($258_{-56}^{+45} {\rm~km~s^{-1}}$ for $\Omega_m=0.2$ and
$\Omega_\Lambda=0.8$) is in good agreement with the average velocity
dispersion of $230 {\rm~km~s^{-1}}$ from the kinematics of the groups'
member--galaxies.  The observed lensing signal was sufficiently strong
that chance projections of unbound galaxies alone would have
difficulty mimicking it.  Some of the detected lensing mass is
therefore almost surely in the form of galaxies orbiting within a
parent dark matter halo -- groups.  It is entirely reasonable to
expect part of this group--component of the lensing mass to consist of
hot gas trapped within the group potential.  On the other hand, it is
not unlikely that a portion of the lensing mass is in the form of
projected, yet physically unassociated, galaxies.  Although there may
be limited amounts of gas associated with the individual galaxies,
such projections will produce no extended group emission.  Therefore,
the presence of a lensing signal does not automatically lead to the
expectation of group--like X--ray emission that is at sufficient
surface--brightness to be detectable in our survey.

It is conceivable that there were several relatively massive groups,
or dark clusters (i.e., cluster--mass objects with very low
mass--to--light ratios), that dominated the mean mass measurement.
Although we cannot rule out the possibility that a large fraction of
the mass is in only a few objects that happened to fall outside the
field of view of our observations, and that the remaining objects are
not gravitationally bound structures but rather chance associations of
galaxies that just happen to be near one another in redshift--space,
this seems somewhat unlikely: Combinatorial statistics dictate that if
there were a few massive halos scattered among the 40 alleged groups
and none of the rest are true groups, then the probability is low that
all the massive ones ended up among the 19 groups that did not fit
into our field of view.  Consider, for instance, if there were only
four massive halos: since 19 is roughly half of 40, if the four
massive halos were randomly distributed the probability that all four
would be out of our field of view is $\lsim\left(1/2\right)^4$.
Furthermore, a calculation in \citet{carlberg2001gg} shows that the
number of alleged groups in the field in the redshift range
$0.1<z<0.6$ (40 in the \textsl{CNOC2} field, 21 in the
\textsl{EPIC--PN} field of view) is roughly consistent with
predictions.  The concordances both of the observed number of groups
with the predicted number, and of the observed velocity dispersions
with the mass inferred from weak lensing, together constitute a strong
indication that the \textsl{CNOC2} group sample does not suffer
significant contamination from chance superpositions of galaxies.  An
important caveat is that the number of groups in the \textsl{CNOC2}
fields in the high $\sigma_v$ range is in excess of the aforementioned
predictions derived from the basic formalism of
\citet{press+schecter1974} by a factor of $\sim 4$.  Based on the
\citet{carlberg2001eg} results, $\sim 80\%$ of such high $\sigma_v$
groups could be false, and our own null results are consistent with
this hypothesis.

If we take seriously the conclusion, for which our analysis has
generated some modest evidence, that galaxy groups at intermediate
redshift are less X--ray luminous than those at redshift zero, we can
think of several potential explanations.  Groups at $z \sim 0.5$ are
younger than their relatives in the local universe, which leads to the
following three possibilities: \pn{i} Since simulations and
observations indicate that groups and clusters at the present are
accreting gas from cosmic filaments, groups at a younger stage of the
universe would presumably have been accreting gas for less time than
the more evolved groups of the present--day universe and might
therefore contain less gas.  \pn{ii} Even if the intragroup gas in
young groups is not lower density than the gas in groups today, if the
young groups are still in the process of heating up -- if they have
not yet reached their virial temperatures -- their luminosities would
be reduced. (We find this possibility to be unlikely, because the
heating timescale is short relative to a Hubble time, even at $z \sim
0.5$.)  \pn{iii} Even if the gas in young groups is on average the
same density and the same temperature as the gas in nearby groups, if
it is of lower metallicity its X--ray emissivity would be reduced.
There are undoubtedly feedback processes that enrich the intragroup
medium, and pristine gas of primordial abundance is also surely
constantly falling onto groups, so it is not clear whether to expect
higher metallicity in the IGM of groups that are at $z \sim 0.5$ or at
$z \sim 0$. For warm gas, observed in a soft X--ray band such as our
D--band, line--emission can be important, depending on abundances.
For example, if a $T = 1{\rm~keV}$ group at redshift 0.5 contained
primordial gas, its D--band X--ray flux would be only two thirds as
great as if its gas were at 10\% solar abundance and only half great
as if its gas were at 20\% solar abundance.  It is possible,
therefore, that the relation between velocity dispersion and X--ray
luminosity ought to be shifted down by a factor of $\sim 2$ in
Figure~\ref{groups_fig:maxlums}.  Such a shift would make it less
surprising that we detected no X--ray emission from the groups.

Future data on high--redshift clusters that make it possible to
examine the time--evolution of the density and metallicity of the
intracluster medium in low--mass clusters, and deeper searches for
X--ray gas in low--mass objects at intermediate redshift, will help to
answer why our survey failed to detect such gas.


\section*{APPENDIX}

\subsection*{I Regression Reconsidered}
\label{appendixI}

In this Appendix, we address the question of which regression represents the
``true'' relation between the velocity dispersion and the X--ray luminosity of
groups of galaxies.  It is instructive first to consider a statistical
effect that is common to all regression analysis.  It has been known at
least since Karl Pearson's seminal paper in 1901 that if there is scatter (say,
due to measurement error) in the variable considered to be the independent one,
this will in general cause the recovered slope from regression analysis to
be shallower than the slope of the true underlying relation between the
variables.  (PT04 demonstrate a nice example of this effect.) An
important consequence of this effect is that when there is scatter in both
variables, the slopes from both the direct and the inverse regression will
be shallower than the respective true slopes. 

For example, suppose two variables $x$ and $y$ are related by
$y = \alpha x$ (or $x = \beta y$, where $\beta = 1/\alpha$).  In an experiment,
a large number of $(x,y)$ pairs are measured, and there are statistical
uncertainties in the measurements of both variables.  If $x$ is considered
to be the independent variable, regression analysis will lead to
$y = \hat{\alpha} x$, where in general $\hat{\alpha} < \alpha$
because of the scatter in $x$.  If $y$ is considered to be the
independent variable, regression (``inverse regression'') analysis
will lead to $x = \tilde{\beta} y$, where in general $\tilde{\beta} < \beta$
because of the scatter in $y$.  Of course, if we wish to represent $y$ as
a function of $x$ even in the case of the inverse regression, we will write
$y = \tilde{\alpha} x$, where $\tilde{\alpha} = 1/\tilde{\beta} > \alpha$.
So, although the recovered slopes are in general less than the respective true
slopes for both regressions
($\hat{\alpha} < \alpha$ and $\tilde{\beta} < \beta$),
when we fix the abscissa and ordinate variables we get a shallower slope
than the true one in the case of the direct regression and and a steeper
slope in the case of the inverse regression.  Note that this effect holds
whether the scatter comes from measurement error or is intrinsic to the
variables.

When there is scatter in both variables, so that neither one is a
proper independent variable, it is sometimes appropriate to use a type
of best--fit line that depends on the geometric orientation of the
points.  Instead of finding the line that minimizes the summed,
squared {\it vertical} deviations of the data from the line, as in the
case of regression, some situations are better analyzed by finding the
line that minimizes the summed, squared {\it perpendicular} distances
of the data from the line.  This is sometimes called ``least squares
orthogonal distance fitting,'' and was the subject of
\citet{pearson1901}.  Because this type of fitting, in contrast to
linear regression, is unfortunately sensitive to the units of
measurement, it is advisable to normalize the data prior to fitting.
The most common normalization is by the estimated errors or
uncertainties ($\sigma_x$ and $\sigma_y$) on the variables
\citep{NR1992, akritas+bershady1996}.  The direct regression line will
have the minimum slope, the inverse regression line will have the
greatest slope, and the ``distance--fit'' line will have an
intermediate slope (closer to the direct regression slope if the
scatter in $x$ is smaller; closer to the inverse regression slope
otherwise).  All three lines intersect at the centroid of the data.

For completeness, we present the equation of the least squares
orthogonal distance fit line.  Let us express the measured data as a
set of $N$ points $\{(x_i, y_i)\}_{i=1}^N$.  To simplify the form of
the line's equation, we define the following sums of data:
\[
S_x \equiv \sum_{i=1}^N x_i \qquad S_y \equiv \sum_{i=1}^N y_i
\]
\begin{equation}
\label{groups_eq:sums} S_{x^2} \equiv \sum_{i=1}^N {x_i}^2
 \quad S_{xy}  \equiv \sum_{i=1}^N (x_i \: y_i)
 \quad S_{y^2} \equiv \sum_{i=1}^N {y_i}^2;
\end{equation}
and we define the following combinations of these sums:
\begin{eqnarray}
\nonumber      A & \equiv & S_x S_y - NS_{xy}\\
\nonumber      B & \equiv & (S_x)^2 - (S_y)^2 - N(S_{x^2}-S_{y^2})\\
\label{groups_eq:ABC} C & \equiv & - S_x S_y + NS_{xy}.
\end{eqnarray}
If the equation of the least squares orthogonal distance fit line is
\begin{equation}
y = m x + b,
\label{groups_eq:basiclineeq}
\end{equation}
then the slope of the line is given by
\begin{equation}
m = \frac{-B \pm \sqrt{B^2 - 4AC}}{2A},
\label{groups_eq:meq}
\end{equation}
and the $y$--intercept is given by
\begin{equation}
b = \frac{S_y - m S_x}{N}.
\label{groups_eq:beq}
\end{equation}
The expression for $m$ in equation~(\ref{groups_eq:meq}) contains a
$\pm$--sign from the solution to a quadratic equation. One solution
for $m$ gives the line that minimizes the summed squared orthogonal
distances to the points, and the other solution gives the line passing
through the centroid of the data that maximizes the summed squared
orthogonal distances to the points.


\subsection*{II Comparison: \textsl{CNOC2} vs. M03 Groups Samples}
\label{appendixII}

In this Appendix, we address how to compute the probability that the
\textsl{CNOC2} groups and the Mulchaey (M03) groups derive from the same
population.  In order to do this, we find the properties of the hypothetical
parent group population $\mathcal{P}$ that maximize the joint probability of
our results and Mulchaey's results.

Let the redshifts of the \textsl{CNOC2} groups be labeled $\{z_k\}_{k=1}^{21}$.
At each redshift $z_k$, let $N_k$ be the number of M03 groups that our
simulation predicts would have been visible in our 110 ksec observation, and
let $p_k$ be the probability that a group drawn from $\mathcal{P}$ will be
visible in our survey if it is in our field of view and at redshift $z_k$.

At redshift $z_k$, the probability of obtaining our result -- of not seeing any
groups -- is
\begin{equation}
q_k \equiv (1-p_k);
\label{groups_eq:qk}
\end{equation}
and the probability of obtaining the M03 result -- of seeing $N_k$ groups, as
predicted by our simulation -- is
\begin{equation}
B(M,N_k,p_k) = {M \choose N_k} {p_k}^{N_k} {q_k}^{M-N_k},
\label{groups_eq:M03prob}
\end{equation}
where $M$ is the total number of groups in the M03 sample, either 109 or 61, as
described above.

The joint probability, then, of obtaining our result and of obtaining the M03
result is
\begin{equation}
P_{\rm joint}(z_k) = q_k B(M,N,p_k) = {M \choose N_k} {p_k}^{N_k} {q_k}^{M-N_k+1}.
\label{groups_eq:joint}
\end{equation}

At each redshift $z_k$, we find the binomial probability $\hat{p}_k$ that
maximizes $P_{\rm joint}(z_k)$.  This set of optimal probabilities
$\{ \hat{p}_k \}$ can be thought of as defining the parent population
$\widehat{\mathcal{P}}$ of groups that maximizes the probability that we would
obtain our null result and that the M03 groups would have their observed
properties.  Each percentage $\hat{p}_k$ in turn can be thought of as an
estimate of the probability that a group would be visible in our survey if it
were randomly selected from $\widehat{\mathcal{P}}$ and placed at redshift
$z_k$.  The value $\hat{q}_k \equiv (1-\hat{p}_k)$, then, is the
probability that such a group would {\it not} be detected in our survey.

If 21 groups were selected at random from $\widehat{\mathcal{P}}$ and placed at
the 21 redshifts $z_k$,
the probability that none would be seen in our survey is the product
of the 21 quantities $\hat{q}_k$.  A maximal estimate of the conditional
probability that none of the \textsl{CNOC2} groups would be detected at the
$5\sigma$ level, given that the \textsl{CNOC2} groups were drawn from the same
population as the M03 groups, may therefore be represented as follows:
\begin{equation}
P(\varnothing ~| {\rm ~same ~pop.}) \leq \prod_{k=1}^{21} \left( 1-\hat{p}_k \right),
\label{groups_eq:condprob}
\end{equation}
where $\varnothing$ denotes our null result of not detecting any
\textsl{CNOC2} groups.  The result of our simulation is that the
product in equation~(\ref{groups_eq:condprob}) is $P \leq 14.7\%$ for
$M=109$ and $P \leq 2.8\%$ for $M=61$; this indicates that the
probability is small, but not vanishing, that the \textsl{CNOC2}
groups in our survey have the same luminosity function as the M03
groups.
